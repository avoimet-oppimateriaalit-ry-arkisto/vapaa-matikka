\newpage
\section*{Lisenssi}

Kirjan lisenssi: Creative Commons Attribution 3.0 Unported (CC BY 3.0) \\
(\url{http://creativecommons.org/licenses/by/3.0/legalcode})

Sinulla on vapaus:
\begin{enumerate}
\item jakaa — kopioida, levittää, näyttää ja esittää teosta
\item remiksata — valmistaa muutettuja teoksia
\item käyttää teosta kaupallisiin tarkoituksiin
\end{enumerate}
Seuraavilla ehdoilla:
\begin{enumerate}
\item nimeä — Teoksen tekijä on ilmoitettava siten kuin tekijä tai teoksen lisensoija on sen määrännyt (mutta ei siten että ilmoitus viittaisi lisenssinantajan tukevan lisenssinsaajaa tai Teoksen käyttötapaa)
\end{enumerate}

Määräys lisäyksenä lisenssiin: kaikkien tekijöiden nimet on ilmoitettava jossakin kohtaa kirjaa.

\subsection*{Kirjoittajat}
\begin{minipage}[t]{0.5\textwidth}
\begin{itemize}
\item Lauri Hellsten
\item Niko Ilomäki
\item Pauliina Kärkkäinen
\item Edvard Majakari
\item Joonas Mäkinen
\item Johanna Rämö
\item Sampo Tiensuu
\end{itemize}
\end{minipage}
\begin{minipage}[t]{0.5\textwidth}
\begin{itemize}
\item Jokke Häsä
\item Tero Keinänen
\item Vesa Linja-aho
\item Ossi Mauno
\item Matti Pajunen
\item Topi Talvitie
\item Ville Tilvis
\end{itemize}
\end{minipage} \\
\\
\\
\\

This work is licensed under the Creative Commons Nimeä 3.0 Muokkaamaton License. To view a copy of this license, visit http://creativecommons.org/licenses/by/3.0/.

\newpage
\section*{Metasivu}

\textbf{Projektin tietoja} \\
Projektin nimi: Oppikirjamaraton \\
Taustataho: Avoimet oppimateriaalit ry \\
Koordinaattori: Vesa Linja-aho, Metropolia AMK \\
Rahoitus: Hankerahoitus Teknologiateollisuuden 100-vuotissäätiöltä

\textbf{Kirjan LaTeX-lähdekoodi} \\
\url{https://github.com/Oppikirjamaraton/oppikirjamaraton-maa2}

\textbf{Kirjan uusin julkaistu pdf-versio} \\
\url{https://github.com/Oppikirjamaraton/oppikirjamaraton-pdf/blob/master/MAA2.pdf?raw=true}

\textbf{Kirjan virallinen kotisivu} \\
\url{http://avoinoppikirja.fi}

\textbf{Oppikirjamaraton Facebookissa} \\
\url{http://facebook.com/oppikirjamaraton}

\textbf{Avoimet oppimateriaalit ry \& Oppikirjamaraton IRCnetissä} \\
\#avoimetoppimateriaalit

Versio: 0.33 \qquad lähdekoodi ajettu \today \\
Kirjan kirjoittaminen aloitettiin viikonlopun 14.--16.12.2012 aikana. \\
Uusi versio julkaistaan aina, kun edistystä on tapahtunut riittävästi.

\subsection*{Kiitokset}
\begin{itemize}
\item Oskari Lehtonen %Red Bullia
\item Pekka Peura %avoinoppikirja.fi
\item Metropolia AMK %tilat
\item Teknologiateollisuuden 100-vuotissäätiö %rahoitus
\end{itemize}

\newpage

%\subsection{Rahoittajia ja tukijoita} Jos tällaiseen kasaisi rahallista tukea antaneet? (Vai kannattaako edes erotella rahaa ja tiloja/matkoja, ...)

\section*{Mikrolahjoituskanavat}

\subsection*{Flattr}
\url{https://flattr.com/profile/oppikirjamaraton}

%\includegraphics[scale=0.2]{MAA1-Flattr.png} \\
%\url{http://flattr.com/t/914482}
%Teen Flattriin uudet entryt Vapaa matikka -PDF:ille T:Joonas

\subsection*{Bitcoin}

\includegraphics[scale=0.5]{content/pictures/Oppikirjamaraton-Bitcoin.png} \\
bitcoin:148pMeTViRMFBqMVZRnMZ6Hwccq9WTubq1?label=Oppikirjamaraton

\todo{Ennen sisällysluetteloa tekstiä kurssin tavoitteista ja aikatauluehdotus}

