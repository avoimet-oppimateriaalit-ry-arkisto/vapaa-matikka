\appendix

\chapter{Harjoituskokeita}

\section{Harjoituskoe 1}

kesken...
% \begin{enumerate}
% \item Ratkaise epäyhtälö $1-\dfrac{1-x}{6}<x$.
% \item Ratkaise yhtälöt. \\ a) 4x^2-1=0 \\ b) x^3=-3x \\ c) 2y^2=y-8
% \item Millä parametrin $k$:n arvoilla yhtälöllä $kx^2-(k+1)x+1=0$ on kaksi erisuurta reaalijuurta?
% \end{enumerate}

\section{Harjoituskoe 2}


\section{Harjoituskoe 3}


\section{Harjoituskoe 4}



\chapter{Todistukset}

\subsection*{Tulon nollasääntö}
\label{tod:tulonolla}

\begin{todistus}
Annettuna joukko nollasta poikkeavia lukuja $x_0, x_1, x_2 ... , x_n$ asetetaan
\begin{align*}
    x_0 \cdot x_1 \cdot x_2 \cdot ... \cdot x_n &= 0 & \ppalkki & : (x_0 \cdot x_1 \cdot x_2 \cdot ... \cdot x_{n-1}) \\
    & & & \text{koska kaikille $x_i$ pätee $x_i \neq 0$} \\
    \\
    \frac{x_0 \cdot x_1 \cdot x_2 \cdot ... \cdot x_n}{x_0 \cdot x_1 \cdot x_2 \cdot ... \cdot x_{n-1}} &=
    \frac{0}{x_0 \cdot x_1 \cdot x_2 \cdot ... \cdot x_{n-1}} & \ppalkki & \frac{0}{x_i} = 0\ \text{kaikilla $x_i,\ x_i \neq 0$} \\
    \\
    x_n &= 0 & &
\end{align*}

mikä on ristiriidassa todistuksessa asetetun vaatimuksen kanssa (kaikkien
lukujen piti olla nollasta poikkeavia). Näin alkuperäinen väite on tosi.

\end{todistus}

\chapter{Toisen asteen polynomifunktio}

\section{Harjoitustehtäviä}

%funktion nollakohtien ja yhtälön juurien yhteys
%paraabelin muodon perustelu: P(x)=ax^2+bx+c=a(x-b/2a)^2+b^2/4a^2, siis pienin/suurin arvo kun x=-b/2a -> ylöspäin ja alaspäin aukeavat paraabelit

\subsection*{Polynomien jakolause}
\label{tod:poljako}

Jakolauseen todistus perustuu polynomien jakoyhtälöön, josta tarkemmin kurssilla 12.

\begin{todistus}
Vaikka lauseke $x-b$ ei olisi polynomin $P(x)$ tekijä, niin lähelle päästään: jos polynomin $Q(x)$ kertoimet valitaan sopivasti, voidaan kirjoittaa
\begin{align*}
P(x)&=(x-b)Q(x)+r,
\end{align*}
missä $r$ on jokin vakio, niin sanottu jakojäännös. Jos nyt $b$ on polynomin $P$ nollakohta, sijoitetaan edelliseen yhtälöön $x=b$, jolloin
\begin{align*}
P(b)&=(b-b)Q(b)+r \quad || \ \ P(b)=0 \\
0&=0+r,
\end{align*}
eli $r=0$, joten $x-b$ on polynomin $P(x)$ tekijä. Jos siis $x=b$ on polynomin $P(x)$ nollakohta, $x-b$ on sen tekijä.
\end{todistus}


\section{Tehtäviä ylioppilaskokeista}

\subsubsection*{Pitkän oppimäärän tehtäviä}

\begin{tehtava}
	Muodosta sen suoran yhtälö, joka kulkee pisteiden $(4, -3)$ ja $(-2,6)$ kautta, blaa. (S07/1b)
\end{tehtava}




\section{Sekalaisia tehtäviä}

Ryhmittele tehtävät luvuittain!
Sijoita nämä tehtävät sopivaan paikkaan, jos ovat hyviä!


\paragraph*{Polynomifunktion kuvaaja}

\begin{tehtava}
  Aukeavatko seuraavat paraabelit ylös- vai alaspäin?
  \begin{enumerate}[a)]
    \item $4x^2 + 100x - 3$
    \item $-x^2 + 1337$
    \item $5x^2 - 7x + 5$
    \item $-6(-3x^2 + 5)$
    \item $-13x(9 - 17x)$
    \item $100(1-x^2)$
  \end{enumerate}

  \begin{vastaus}
    \begin{enumerate}[a)]
      \item Ylös
      \item Alas
      \item Ylös
      \item Ylös
      \item Ylös
      \item Alas
    \end{enumerate}
  \end{vastaus}
\end{tehtava}

\begin{tehtava}
  \begin{enumerate}[a)]
    \item Ratkaise funktion $2x^2 - 5x - 3$ nollakohdat
    \item Millä arvoilla edellisen kohdan funktio $2x^2 - 5x - 3$ saa positiivisia arvoja?
    \item Onko em. funktiolla globaali raja-arvo (minimi tai maksimi), ja jos on, missä kohtaa funktio saa tämän arvon? Mikä on funktion arvo silloin?
  \end{enumerate}

  \begin{vastaus}
    \begin{enumerate}[a)]
      \item $x = 1.2$ tai $x = -0.2$
      \item $x = \frac{12}{10} = 1.2$ tai $x = -\frac{2}{10} = -0.2$
      \item Koska neliötermin kerroin a on positiivinen (2), funktiolla on globaali minimi (mutta ei ylärajaa). Symmetrian vuoksi minimi on nollakohtien puolivälissä kohdassa 0.5, jossa funktio saa siis pienimmän arvonsa -5.
    \end{enumerate}
  \end{vastaus}
\end{tehtava}

\begin{tehtava}
  Tutki, millä muuttujan x arvoilla seuraavat funktiot saavat positiivisia arvoja.
  \begin{enumerate}[a)]
    \item $x^2 - 4$
    \item $-x^2 - 2x + 3$
    \item $x^2 + 2x + 5$
    \item $-x^2 - 1$
  \end{enumerate}

  \begin{vastaus}
    \begin{enumerate}[a)]
      \item $x \leq -2$ tai $x \geq 2$
      \item $-3 \geq x \leq 1$
      \item Kaikilla x:n arvoilla.
      \item Ei millään x:n arvoilla.
    \end{enumerate}
  \end{vastaus}
\end{tehtava}
