\chapter{Toisen asteen epäyhtälö}
\textbf{Esimerkki 1.}
\missingfigure{johdantoesimerkki esim. lämpötilatarkastelu tjsp}
\begin{itemize}
\item{Milloin lämpötila on suurempi kuin nolla?}
\item{Milloin lämpötila on pienempi kuin nolla?}
\item{Milloin lämpötilan merkki voi vaihtua?} 
\end{itemize}
\textbf{Esimerkki 2.} \\
Milloin toisen asteen polynomifunktion $f(x)=x^2-5$ arvot ovat suurempia kuin nolla?  \\
\textbf{Ratkaisu}
Funktion $f$ arvot ovat suurempia kuin nolla niillä muuttujan $x$ arvoilla, joilla pätee $x^2-5>0$. \\
\laatikko{
\begin{itemize}
\item{Funktion arvo on positiivinen, kun funktion kuvaaja on x-akselin yläpuolella.}
\item{Funktion arvo on negatiivinen, kun funktion kuvaaja on x-akselin alapuolella.}
\end{itemize}}
\missingfigure{kuvaaja johon merkitty missä pos ja missä neg}
Ratkaistaan funktion $f$ nollakohdat, koska niissä kohdissa funktion arvojen merkki voi vaihtua.
\begin{align*}
f(x)&=0 \\
x^2-5&=0 \\
x^2&=5 \\
x=\pm \sqrt[]{5}
\end{align*}
Funktion $f$ kuvaaja on ylöspäin aukeava paraabeli, koska kerroin $a>0$.
Funktio $f$ leikkaa x-akselin kohdissa $x=-\sqrt[]{5}$ ja $x=\sqrt[]{5}$.
\missingfigure{kuvaaja, johon merkitty nollakohdat}
Funktion arvot ovat positiivisia niillä muuttujan $x$ arvoilla, joilla funktion kuvaaja on $x$-akselin yläpuolella.
Kuvaajasta huomataan, että $f(x)>0$, kun $x<-\sqrt[]{5}$ tai $x>\sqrt[]{5}$.
\section{Harjoitustehtäviä}
