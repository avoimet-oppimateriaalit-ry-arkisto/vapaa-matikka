% Tämä luku sisältää aiemmin luvussa 2 (Polynomien yhteen- ja vähennyslasku) olleen sisällön.
% Päätimme siirtää sisällön aliluvuksi lukuun 1 (Polynomit), jotta päästään käsittelemään samassa luvussa polynomien sieventämistä perusmuotoon.
% Sisältö on helposti palautettavissa \input-komennolla kumpaan lukuun halutaan.
% T: Jokke ja Johanna

Polynomeja voidaan laskea yhteen summaamalla samanasteiset termit. Esimerkiksi polynomien $5x^2-x+5$ ja $3x^2-1$ summa on
   \begin{align*}
        (\textcolor{red}{5x^2}-x\textcolor{blue}{{}+5})+(\textcolor{red}{3x^2}\textcolor{blue}{-1}) &=\textcolor{red}{(5+3)x^2}-x\textcolor{blue}{{}+(5-1)}=\textcolor{red}{8x^2}-x\textcolor{blue}{{}+4}.
    \end{align*}
Samalla tavalla polynomeja voidaan vähentää toisistaan. Polynomien
$14x^3+69$ ja $3x^3+2x^2+x$ erotus on
    \begin{align*}
        (\textcolor{green}{14x^3} + 69) - (\textcolor{green}{3x^3} \textcolor{blue}{{}+ 2x^2} \textcolor{red}{{}+x})
        &= \textcolor{green}{14x^3} + 69 \textcolor{green}{{}-3x^3} - 
            \textcolor{blue}{2x^2} \textcolor{red}{{}-x} \\
        &= \textcolor{green}{(14-3)x^3} \textcolor{blue}{{}-2x^2} \textcolor{red}{{}-x} + 69 \\
        &= \textcolor{green}{11x^3} \textcolor{blue}{{}-2x^2} \textcolor{red}{{}-x} + 69
    \end{align*}
    
Samanasteisten termien yhteen- ja vähennyslasku perustuu siihen, että vakiokerroin voidaan
ottaa yhteiseksi tekijäksi:
\[
ax^n+bx^n=(a+b)x^n.
\]
    
\begin{esimerkki}
Lasketaan polynomit
$x+1$ ja $3x^2-2x+5$ yhteen:
   \begin{align*}
        (x+1)+(3x^2-2x+5) =3x^2-x+6.
    \end{align*}
Lasketaan polynomit $-4x^4+3x^3-x$ ja $-3x^3+5x^2+2x$ yhteen:
   \begin{align*}
        (-4x^4+3x^3-x)+(-3x^3+5x^2+2x) =-4x^4+5x^2+x.
    \end{align*}
Vähennetään polynomit $x+1$ ja $-4x^4+3x^3-x$ toisistaan:
   \begin{align*}
        (x+1)-(-4x^4+3x^3-x) =x+1+4x^4-3x^3+x=4x^4-3x^3+2x+1.
    \end{align*}
\end{esimerkki}

%EN AIVAN YMMÄRTÄNYT TÄTÄ
%Huomaa, että sulkeet voidaan poistaa ja termit yhdistää silloin, kun ne ovat
%samaa astetta, ihan samaan tapaan kuin sulkeet voidaan poistaa lukujen
%yhteenlaskussa silloin, kun ne eivät ole tulon tekijöinä.

Yhteenlasketut ja vähennetyt polynomit täytyy sieventää perusmuotoon, jos halutaan tarkistaa polynomin aste.
Esimerkiksi polynomin
\[
P(x)=(x^2+2x+1)-(x^2+2)
\]
aste ei ole kaksi vaan yksi, sillä perusmuodossaan polynomi $P(x)$ on
\[
P(x)=x^2+2x+1-x^2-2=2x-1.
\]
Tässä tapauksessa siis toisen asteen termit häviävät sievennettäessä.