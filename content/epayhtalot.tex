\chapter{Epäyhtälöt}

\section{Ensimmäisen asteen epäyhtälö}

\textbf{Esimerkki 1}
\begin{align*}
ax^2+bx+c<0 \\
9 \cdot x+15\geq 0 \\
1>0>-1 
\end{align*}

Lauseketta, jossa yhtälön $=$-merkki on korvattu
''$<$''-, ''$\leq$''-, ''$>$''-, ''$\geq$''- tai ''$\neq$''-merkillä sanotaan epäyhtälöiksi.

Samoin kuin yhtälöiden kohdalla, myös epäyhtälöistä halutaan usein selvittää
tuntemattoman muuttujan arvot. Epäyhtälöllä ratkaisuja on tietysti paljon ja ratkaistaessa
pyritään löytämään ne kaikki. Käytännössä ratkaiseminen tarkoittaa sitä, että epäyhtälö
muutetaan niin yksinkertaiseen muotoon kuin mahdollista siten, että yksinkertaisesta epäyhtälöstä
on helppoa nähdä, mitkä luvut kuuluvat ratkaisuun ja mitkä eivät.

Kokeilemalla tai vähän miettimällä on helppoa huomata, että epäyhtälöiden ratkaisemisessa
voi käyttää suurelta osin samanlaisia menetelmiä kuin yhtälön ratkaisemisessa. Helposti huomataan,
että myös epäyhtälöön voidaan lisätä puolittain jotakin tai siitä voidaan vähentää puolittain jotakin.
Samoin epäyhtälö voidaan kertoa tai jakaa puolittain positiivisella luvulla. Negatiivisella luvulla
jakamisessa ja kertomisessa täytyy kuitenkin huomata jotain erityistä.

\textbf{Esimerkki 2}
\begin{align*}
4&<6  \ \ \ \ \ &&|| +4 \\
8&<10 &&|| \cdot 5 \\
40&<50 &&|| :2 \\
20&<25 &&|| \cdot (-3) \\
-60&>-75
\end{align*}

Yllä oleva esimerkki havainnollistaa sitä, että epäyhtälö säilyy aivan kuin yhtälökin, kun sama toimitus tehdään
epäyhtälön kummallekin puolelle. Viimeisessä vaiheessa huomataan kuitenkin, että negatiivisella luvulla kerrottaessa
epähtälön keskellä oleva merkki pitää kääntää toisin päin, että epäyhtälö olisi edelleen voimassa. $20<25$, mutta kun molemmat
kerrotaan $-3$:lla, saadaan $-60$ ja $-75$, jolloin epäyhälö kääntyykin toisin päin.

Myös silloin, kun epäyhtälö korotetaan potenssiin, täytyy merkin suuntaan kiinnittää erityistä huomiota.
Tästä kerrotaan tarkemmin toisen asteen epäyhtölön yhteydessä.

\textbf{Esimerkki 2}
Millä $w$:n arvoilla pätee $-8w-(8-w)\geq w:2+5$?

\textbf{Ratkaisu}
\begin{align*}
-8w-(8-w)&\geq w:2+5 \\
-8w-8+w&\geq w:2+5 \\
-7w-8&\geq \frac12 w+5  \ \ \ \ \ &&|| -\frac12 w \\
-7\frac12 w-8&\geq 5  \ \ \ \ \ &&|| +8 \\
-7\frac12 w&\geq 13  \ \ \ \ \ &&|| :(-7\frac12) \\
w&\leq 13:(-7\frac12) \\
w&\leq 13:(-\frac{15}{2}) \\
w&\leq -13\cdot \frac{2}{15} \\
w&\leq -\frac{26}{15} \\
w&\leq -1\frac{11}{15}
\end{align*}

Vastaus: $w\leq -1\frac{11}{15}$

Epäyhtälö ratkaistaan siis aivan kuten vastaava yhtälö, mutta negatiivisella luvulla
jaettaessa tai kerrottaessa merkki kääntyy toisin päin.

\textbf{Esimerkki 3}
Ratkaise yhtälö $1\leq q+7<-5q+4$.

\textbf{Ratkaisu}
Tässä on itse asiassa kaksi epäyhtälöä $1\leq q+7$ ja $q+7<-5q+4$. Haluamme siis löytää ne $q$:n arvot, joilla molemmat epäyhtälöt pätevät.
\begin{align*}
1&\leq q+7 \ \ \ \ \ &&|| -7 \\
-6&\leq q
\end{align*}
Vastaavasti toiselle yhtälölle:
\begin{align*}
q+7&<-5q+4  \ \ \ \ \ &&|| +5q \\
6q+7&<4 &&|| -7 \\
6q&<-3 &&|| :6 \\
q&< -\frac12 \\
\end{align*}

Nämä yhdistämällä saadaan $-6\leq q$ ja $q< -\frac12$ eli $-6\leq q < -\frac12$ eli $q\in [-6, -\frac12[$.

\textbf{Esimerkki 4}
Ratkaise yhtälö $|-5q+4|<5$.

\textbf{Ratkaisu}
Luvun itseisarvo on pienempi kuin $5$ jos ja vain jos se on $-5$:n ja $5$:n välillä.
Itseisarvoepäyhtälö voidaan siis jakaa kahdeksi epäyhtälöksi $-5<-5q+4<5$ eli $-5<-5q+4$ ja $-5q+4<5$.

\begin{align*}
-5&<-5q+4 \ \ \ \ \ &&|| -4 \\
-9&<-5q &&|| :(-5) \\
\frac95&>q
\end{align*}
Vastaavasti toiselle yhtälölle:
\begin{align*}
-5q+4&<5  \ \ \ \ \ &&|| -4 \\
-5q&<1 &&|| :(-5) \\
q&>-\frac15 \\
\end{align*}

(Huomaa missä kohtaa epäyhtälön merkki kääntyi, kun kerrottiin puolittain negatiivisella luvulla.)

Vastaus: $-\frac15<q<\frac95$ eli $q\in ]-\frac15,\frac95[$

\section{Harjoitustehtäviä}
\begin{tehtava}
    Ratkaise seuraavat yhtälöt tai epäyhtälöt.
    \begin{enumerate}
        \item $-2r+6=0$
        \item $-2r+6\leq 0$
        \item $5y-2<y+6$
        \item $8(x+2)\geq 5(5-x)\cdot (-1)+3$
        \item $\frac{x+3}{2}+\frac{-2x+1}{3}>\frac{x-9}{3+1}$
    \end{enumerate}
    \begin{vastaus}
        \begin{enumerate}
            \item $r=3$
            \item $r\geq 3$
            \item $y<2$
            \item $x=-12\frac{2}{3}$
            \item $x<9\frac{4}{5}$
        \end{enumerate}
    \end{vastaus}
\end{tehtava}

\begin{tehtava}
    Ratkaise seuraavat epäyhtälöt.
    \begin{enumerate}
        \item $3x+6<2x\leq 9-x$
        \item $3x+6<2x\leq 1+3x$
        \item $|5y-2|<12$
        \item $3\leq |x+9|$
        \item $-\smiley{}-5\geq-888$
    \end{enumerate}
    \begin{vastaus}
        \begin{enumerate}
            \item $x<-6$
            \item ei ratkaisua
            \item $-2<y<2\frac{4}{5}$
            \item $x\geq -6$ tai $x\leq -12$
            \item $\smiley{}\leq 883$
        \end{enumerate}
    \end{vastaus}
\end{tehtava}