\chapter{Toisen asteen yhtälön ratkaisukaava}

Yleensä toisen asteen yhtälöitä ei ratkaista täydentämällä neliöksi, sillä menetelmän voi ilmaista valmiina kaavana.
Johdetaan seuraavaksi toisen asteen yhtälön ratkaisukaava.

\begin{enumerate}
    \item Yleinen toisen asteen yhtälö on muotoa $ax^2+bx+c=0$.
    \item Kerrotaan yhtälön molemmat puolet vakiolla $4a$: $4a^2x^2+4abx+4ac=0$.
    \item Siirretään termi $4ac$ toiselle puolelle: $4a^2x^2+4abx=-4ac$.
    \item Pyritään täydentämään vasen puoli neliöksi.
    \item Lisätään puolittain termi $b^2$: $4a^2x^2+4abx+b^2=b^2-4ac$.
    \item Havaitaan vasemmalla puolella neliö: $(2ax+b)^2=b^2-4ac$.
    \item Otetaan puolittain neliöjuuri: $2ax+b=\pm\sqrt{b^2-4ac}$.
    \item Vähennetään puolittain termi $b$: $2ax=-b\pm\sqrt{b^2-4ac}$.
    \item Jaetaan puolittain vakiolla $2a$: $x=\frac{-b\pm\sqrt{b^2-4ac}}{2a}$.
\end{enumerate}

\section{Harjoitustehtäviä}
