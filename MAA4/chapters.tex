\chapter{Esitietoja}
	\section{Itseisarvo}


\laatikko{
KIRJOITA TÄHÄN LUKUUN

\begin{itemize}
\item itseisarvon määritelmä (paloittain määritelty muoto)
\item geometrinen tulkinta lukusuoralla
\item itseisarvoyhtälöistä tyyppiä $|f(x)|=a$, $|f(x)|=|g(x)|$,
$|f(x)|=g(x)$
\end{itemize}

KIITOS!}

Itseisarvo kuvaa lukusuoralla luvun etäisyyttä nollasta. Positiivisen luvun itseisarvo on luku itse ja negatiivisen luvun itseisarvo on luvun vastaluku. Nollan itseisarvo on nolla.


\laatikko{$|x|= \left\{ \begin{array}{rcl}
		x & , kun & x \geq 0 \\
		-x & , kun & x < 0
		\end{array}\right.$}

\begin{esimerkki}
Laske
	\begin{alakohdat}
		\alakohta{$|3-\pi|$}
		\alakohta{$|x-3|$}
	\end{alakohdat}
	\textbf{Ratkaisu}
	\begin{alakohdat}
		\alakohta{Koska $3-\pi\approx-0,14<0$, niin $|3-\pi|=-(3-\pi)=\pi-3$}
		\alakohta{Koska $x-3\geq 0$, kun $x\geq3$, niin\\
			$|x-3|= \left \{ \begin{array}{rcl}
			x-3 & , kun & x\geq3 \\
			3-x & , kun & x<3
			\end{array}\right.$
		}
	\end{alakohdat}
\end{esimerkki}


	\section{Itseisarvoepäyhtälöt}

\laatikko{
KIRJOITA TÄHÄN LUKUUN

\begin{itemize}
\item miten ratkaistaan epäyhtälöitä tyyliin
\item $|x|<a$,  $|x|>a$
\item $|f(x)|<a$, $|f(x)|<|g(x)|$, $|f(x)|<g(x)$
\end{itemize}

KIITOS!}

Itseisarvoepäyhtälön ratkaiseminen riippuu hyvin paljon epäyhtälön muodosta. Helpointa on hahmotella kutakin tilannetta erikseen lukusuoran avulla ja yrittää palauttaa itseisarvoepäyhtälö yhdeksi tai useammaksi tavalliseksi epäyhtälöksi. Tutustutaan erilaisiin tilanteisiin esimerkkien avulla.

\begin{esimerkki}
Ratkaise epäyhtälö $|x|<3$.

\begin{esimratk}
On löydettävä kaikki sellaiset luvut, joiden itseisarvo on pienempi kuin $3$. Esimerkiksi luvut $2$ ja $-1$ käyvät, mutta luvut $4$ ja $-5,5$ eivät käy. Epäyhtälön ratkaisuja ovat täsmälleen ne luvut, joiden etäisyys nollasta on pienempi kuin $3$.

(Tähän kuva.)

Siten ratkaisu on $-3<x<3$.
\end{esimratk}

\begin{esimvast}
$-3<x<3$.
\end{esimvast}

\end{esimerkki}

\begin{esimerkki}
Ratkaise epäyhtälö $|x|>5$.

\begin{esimratk}
On löydettävä kaikki sellaiset luvut, joiden itseisarvo on suurempi kuin $5$. Esimerkiksi luvut $2$ ja $-4,5$ eivät ole epäyhtälön ratkaisuja, mutta luvut $5,5$ ja $-6$ ovat. Epäyhtälön ratkaisuja ovat täsmälleen ne luvut, joiden etäisyys nollasta on suurempi kuin $5$.
 
(Tähän kuva.)

Nyt ratkaisu on ilmoitettava kahdessa osassa: $x<-5$ tai $x>5$.
\end{esimratk}

\begin{esimvast}
$x<-5$ tai $x>5$
\end{esimvast}
\end{esimerkki}

\begin{esimerkki}
Ratkaise epäyhtälö $|x+4|<2$.

\begin{esimratk}
Nyt luvun $x+4$ itseisarvo on pienempi kuin kaksi, joten aikaisemman esimerkin perusteella tiedetään, että $-2<x+4<2$. Näin saadaan ratkaistaviksi epäyhtälöt $-2<x+4$ ja $x+4<2$. Ratkaistaan nämä kaksi epäyhtälöä erikseen:
\begin{align*}
-2&<x+4 & \ppalkki{+2} \\
0&<x+6 & \ppalkki{-6} \\
-6&<x &
\end{align*}
ja
\begin{align*}
x+4&<2 & \ppalkki{-4} \\
x&<-2 & \ppalkki{-6}
\end{align*}

Kun vastaukset yhdistetään, saadaan ratkaisuksi $-6<x<-2$.

(Tähän kuva.)

\end{esimratk}

\begin{esimvast}
$-6<x<-2$.
\end{esimvast}
\end{esimerkki}

\begin{esimerkki} Ratkaise epäyhtälö $|2x-1|>|x+2|$.

\begin{esimratk} Koska molemmat puolet ovat epänegatiivisia, itseisarvomerkit voitaisiin poistaa korottamalla epäyhtälö puolittain toiseen potenssiin. Mutta miten käy epäyhtälön merkille? Jos kaksi epänegatiivista lukua korotetaan toiseen potenssiin, niiden keskinäinen järjestys säilyy. Voimme siis korottaa tehtävän epäyhtälön puolittain toiseen potenssiin, ja epäyhtälömerkin suunta säilyy. Tämän jälkeen epäyhtälöä käsitellään toisen asteen epäyhtälönä.

\begin{align*}
|2x-1|^2 & >|x+2|^2 \\
(2x-1)^2 & >(x+2)^2 \\
4x^2-4x+1 & >x^2+4x+4 \\
3x^2-8x-3 & >0
\end{align*}

Toisen asteen epäyhtälö ratkaistaan etsimällä lausekkeen nollakohdat ja päättelemällä merkkikaavion tai kuvaajan avulla, millä alueilla epäyhtälö toteutuu.
\begin{align*}
3x^2-8x-3 & =0 \\
x & =\frac{8\pm\sqrt{8^2-4\cdot3\cdot(-3)}}{2\cdot 3} \\
x & =\frac{8\pm\sqrt{64+36}}{6} \\
x & =\frac{8\pm 10}{6} \\
x=3 \quad & \text{tai} \quad x=-\frac{1}{3}
\end{align*}

TÄHÄN MERKKIKAAVIO JA KUVAAJA

Epäyhtälö toteutuu lausekkeen $3x^2-8x-3$ nollakohtien välisen alueen ulkopuolella eli silloin, kun $x<-\dfrac{1}{3}$ tai $x>3$.
\end{esimratk}

\begin{esimvast}
$x<-\dfrac{1}{3}$ tai $x>3$.
\end{esimvast}
\end{esimerkki}

\subsection*{Tehtäviä}

\begin{tehtavasivu}

\subsubsection*{Opi perusteet}

\begin{tehtava}
Ratkaise seuraavat epäyhtälöt.
% vai: Esitä ilman itseisarvomekkejä.
	\begin{alakohdat}
		\alakohta{$|x|<6$}
		\alakohta{$|x|>10$}
		\alakohta{$|x|<1,6$}
	\end{alakohdat}
	\begin{vastaus}
		\begin{alakohdat}
			\alakohta{$-6<x<6$}
			\alakohta{$x<-10$ tai $x>10$}
			\alakohta{$-1,6<x<1,6$}
		\end{alakohdat}
	\end{vastaus}
\end{tehtava}

\begin{tehtava}
Ratkaise seuraavat epäyhtälöt.
	\begin{alakohdat}
		\alakohta{$|x+6|>3$}
		\alakohta{$|x-5|<2$}
	\end{alakohdat}
	\begin{vastaus}
		\begin{alakohdat}
			\alakohta{$x<-9$ tai $x>-3$}
			\alakohta{$3<x<7$}
		\end{alakohdat}
	\end{vastaus}
\end{tehtava}

\subsubsection*{Hallitse kokonaisuus}

\subsubsection*{Sekalaisia tehtäviä}

TÄHÄN TEHTÄVIÄ SIJOITTAMISTA ODOTTAMAAN

\begin{tehtava}
Ratkaise seuraavat epäyhtälöt.
	\begin{alakohdat}
		\alakohta{$|x|\le 6$}
		\alakohta{$|x|>-3$}
	\end{alakohdat}
	\begin{vastaus}
		\begin{alakohdat}
			\alakohta{$-6 \le x \le 6$}
			\alakohta{ei ratkaisua}
		\end{alakohdat}
	\end{vastaus}
\end{tehtava}


\begin{tehtava}
Ratkaise epäyhtälö $|x^2+1| \ge 3$.
	\begin{vastaus}
          $x<\sqrt{-2}$ tai $x>\sqrt{2}$
	\end{vastaus}
\end{tehtava}

\end{tehtavasivu}

	% itseisarvoepäyhtälöt
	\section{Yhtälöryhmät} % FIXME: siirrä

\laatikko{
KIRJOITA TÄHÄN LUKUUN

\begin{itemize}
\item mikä yhtälöryhmä on
\item miten ratkaistaan yhtälöpari (sijoitus, yhteenlaskumenetelmä)
\item että ratkaisuja voi olla yksi, nolla tai äärettömän monta
\item miten useamman tuntemattoman yhtälöryhmä ratkaistaan
\end{itemize}

KIITOS!}

\subsection*{Lineaarinen yhtälöpari}

Yksinkertaisin tapaus yhtälöryhmistä on lineaarinen yhtälöpari, jossa on
kaksi ensimmäisen asteen yhtälöä. Käyttäen normaalimuotoa:

\laatikko{
Lineaarisen yhtälöparin yleinen muoto:

\begin{align*}
a_1x+b_1y+c_1 &= 0 \\
a_2x+b_2x+c_2 &= 0
\end{align*}

$a_1, a_2, b_1, b_2, c_1, c_2 \in \mathbb{R}$

Lineaarisen yhtälöparin ratkaisu on pari $(x, y) \in \mathbb{R}^2$, joka toteuttaa molemmat yhtälöt.
}

Aiemmin on todettu, että normaalimuotoinen ensimmäisen asteen yhtälö voidaan tulkita suorana
$(x, y)$-tasossa. Näin ollen lineaariselle yhtälöparille on geometrinen tulkinta: sen
ratkaisut ovat ne tason pisteet, joissa yhtälöitä vastaavat
suorat leikkaavat. Näitä voi olla $0$ (suorat ovat yhdensuuntaiset,
mutta eivät sama suora), $1$ (suorat eivät ole yhdensuuntaiset) tai äärettömän monta (suorat ovat sama suora).

\laatikko{
Lineaarisella yhtälöparilla voi olla joko $0$, $1$ tai äärettömän monta ratkaisua.
}

\subsection*{Lineaarinen yhtälöryhmä}

Lineaarisen yhtälöparin ajatus yleistyy mihin tahansa määrään ensimmäisen asteen yhtälöitä.
Tällöin puhutaan lineaarisista yhtälöryhmistä. Myös geometrinen tulkinta yleistyy, mutta se on vaikeampi
visualisoida paperille: lineaarinen kolmen yhtälön ryhmä vastaa kolmen $(x, y)$-tason leikkauspisteitä
kolmiulotteisessa $(x, y, z)$-avaruudessa jne.

	% sijoitusmenetelmä
	% yhtälöiden laskeminen yhteen
	% ratkaisujen määrä
	\section{Koordinaatisto ja yhtälön kuvaaja}

\laatikko{
KIRJOITA TÄHÄN LUKUUN

\begin{itemize}
\item ihan lyhyt koordinaatistokertaus
\item kahden pisteen välinen etäisyys (pysty-tai vaakasuoraan helpolla, Pythagoraan lauseella yleensä)
\item esimerkkejä käyrän yhtälöistä, esim. suora, paraabeli, kartesiuksen lehti
\end{itemize}

KIITOS!}

Analyyttisen geometrian perusajatus on käsitellä geometrisia kuvioita koordinaatistossa.
Koordinaatistossa kuvion jokainen piste voidaan ilmoittaa sen koordinaattien avulla.
Esimerkiksi oheisessa kuvassa on kolmio $ABC$, jonka nurkkapisteiden koordinaatit ovat
\[
A(1, 2), \quad B(-1, -1) \quad \text{ja} \quad C(2, -2).
\]

\begin{kuva}
    kuvaaja.pohja(-2, 3, -3, 3, korkeus = 4, nimiX = "$x$", nimiY = "$y$", ruudukko = True)
    geom.jana((1, 2), (-1, -1))
    geom.jana((-1, -1), (2, -2))
    geom.jana((2, -2), (1, 2))
    kuvaaja.piste((1, 2), "$A$", 45)
    kuvaaja.piste((-1, -1), "$B$", 180)
    kuvaaja.piste((2, -2), "$C$", -45)
\end{kuva}

Koordinaattiakselit leikkaavat toisensa kohtisuoraan pisteessä, jota nimitetään \termi{origo}{origoksi}.
Tuota pistettä voi ajatella koordinaatiston keskuksena.
Vaaka-akselia on yleensä tapana nimittää $x$-akseliksi ja pystyakselia $y$-akseliksi.
Näiden mukaan koko koordinaatistoa kutsutaan toisinaan $xy$-koordinaatistoksi.

Kunkin pisteen koordinaatit määräytyvät siitä, missä kohdassa se on $x$- ja $y$-akselien asteikkoihin verrattuna.
Aivan kuten lukusuoralla kutakin pistettä vastaa tietty reaaliluku $x$ ja päinvastoin, koordinaatistossa kutakin pistettä vastaa yksi yhteen tietty lukupari $(x, y)$.

\subsection{Pisteiden välinen etäisyys}

Geometriassa tärkeää on päästä mittaamaan pituuksia.
Tätä varten on selvitettävä, miten voidaan määrittää kahden koordinaatiston pisteen välinen etäisyys.
Aivan kuten tavallisessa geometriassa, emme voi aina turvautua mittaamiseen, sillä siten ei saada täsmällisiä tuloksia.
Sen sijaan pyritään selvittämään pisteiden väliset etäisyydet niiden koordinaattien perusteella.

\begin{kuva}
    kuvaaja.pohja(-4, 3, 0, 6, korkeus = 4, nimiX = "$x$", nimiY = "$y$", ruudukko = True)
    piste((1, 2), "$A$")
    piste((1, 5), "$B$")
    piste((-3, 2), "$C$")
\end{kuva}

Helpointa etäisyyden määrittäminen on silloin, kun pisteet ovat samalla vaaka- tai pystysuoralla.
Toisin sanoen niillä on sama $x$- tai $y$-koordinaatti.
Tällöin niiden etäisyys saadaan yksinkertaisesti laskemalla toisistaan poikkeavien koordinaattien erotus.

Esimerkiksi yllä olevassa kuvassa pisteiden $A(1, 2)$ ja $B(1, 5)$ välinen etäisyys on $5-2=3$.
Toisaalta pisteiden $A(1, 2)$ ja $C(-3, 2)$ välinen etäisyys on $1-(-3)=4$.

Etäisyyden tulee olla aina positiivinen.
Jos ei ole varma pisteiden järjestyksestä, voi käyttää itseisarvoja.
Esimerkiksi edellä pisteiden $A$ ja $C$ etäisyys voidaan laskea myös järjestyksessä $|-3-1|=|-4|=4$.

% \begin{esimerkki}
% jotkin helpot etäisyydet
% \end{esimerkki}

\begin{kuva}
    kuvaaja.pohja(-2, 3, -1, 5, korkeus = 4, nimiX = "$x$", nimiY = "$y$", ruudukko = True)
    piste((1, 2), "$A$")
    piste((4, 3), "$B$")
    geom.jana((1, 2), (1, 3))
    geom.jana((1, 3), (4, 3))
    geom.jana((4, 3), (1, 2))
\end{kuva}

Kun kahdella pisteellä on sekä eri $x$-koordinaatit että eri $y$-koordinaatit, etäisyys on määritettävä toisella tapaa.
Nyt voidaan turvautua Pythagoraan lauseeseen ja siihen, että koordinaatisto on suorakulmainen.

Edellä olevassa kuvassa pisteiden $A(1, 2)$ ja $B(4, 3)$ etäisyys saadaan piirtämällä kuvan mukainen suorakulmainen kolmio $ABC$.
Kateetin $AC$ pituus on pisteiden $A$ ja $B$ $x$-koordinaattien erotus eli $4-1=3$.
Kateetin $BC$ pituus on puolestaan $A$ ja $B$ $y$-koordinaattien erotus eli $3-2=1$.
Hypotenuusan $AB$ pituus saadaan Pythagoraan lauseesta:
\[
|AB|=\sqrt{3^2+1^2}=\sqrt{9+1}=\sqrt{10}.
\]
Pisteiden $A$ ja $B$ välinen etäisyys on siis $\sqrt{10}$.
Tätä ei olisi voinut selvittää mittaamalla.

Yleisessä tapauksessa kahden pisteen välinen etäisyys saadaan seuraavasta kaavasta.
\laatikko{
\textbf{Pisteiden $(x_1, y_1)$ ja $(x_2, y_2)$ välinen etäisyys.}
\[
\sqrt{(x_1-x_2)^2+(y_1-y_2)^2}
\]
}

\begin{esimerkki}
Mitkä ovat luvun alussa olleen kolmion $ABC$ sivujen pituudet?

\begin{esimratk}
Pisteiden koordinaatit olivat $A(1, 2)$, $B(-1, -1)$ ja $C=(2, -2)$.
Kolmion sivujen pituudet ovat sen kärkipisteiden väliset etäisyydet.
Yllä olevaa kaavaa soveltamalla saadaan sivun $AB$ pituudeksi
\[
\sqrt{\bigl(1-(-1)\bigr)^2+\bigl(2-(-1)\bigr)^2}=\sqrt{2^2+3^2}=\sqrt{4+9}=\sqrt{13}.
\]
Sivun $AC$ pituudeksi tulee
\[
\sqrt{(1-2)^2+\bigl(2-(-2)\bigr)^2}=\sqrt{(-1)^2+4^2}=\sqrt{1+16}=\sqrt{17}.
\]
Sivun $BC$ pituudeksi tulee
\[
\sqrt{\bigl((-1)-2\bigr)^2+\bigl((-1)-(-2)\bigr)^2}=\sqrt{(-3)^2+1^2}=\sqrt{9+1}=\sqrt{10}.
\]
\end{esimratk}

\begin{esimvast}
Sivujen pituudet ovat $|AB|=\sqrt{13}$, $|AC|=\sqrt{17}$ ja $|BC|=\sqrt{10}$.
\end{esimvast}
\end{esimerkki}

\subsection{Käyrät ja niiden yhtälöt}

Koordinaatistoon voidaan pisteiden ja janojen lisäksi piirtää myös suoria ja erilaisia käyriä.
Esimerkiksi polynomien kuvaajia piirrettiin ja tutkittiin pitkän matematiikan kurssissa 2 ja suoriin on tutustuttu jo aiemmin.
Yhteistä kaikille näille on se, että koordinaatistoon piirrettyä käyrää (jollaiseksi myös suora voidaan laskea) vastaa jokin yhtälö, jossa esiintyvät tuntemattomat $x$ ja $y$.

Alla olevan kuvan suoraa vastaa yhtälö $y=2x-1$.
Yhtälö tulkitaan siten, että jokaisen suoralla olevan pisteen $(x, y)$ koordinaatit toteuttavat yhtälön.
Esimerkiksi suoran piste $(2, 3)$ toteuttaa yhtälön, sillä $3=2\cdot 2-1$.
Toisaalta piste $(5, 2)$ ei ole suoralla, sillä $2\cdot 5-1=9\neq 2$.

[TÄHÄN KUVA SUORASTA, JOHON ON MERKITTY PISTE (1,1)]

Kääntäen voidaan sanoa, että jos piste toteuttaa yhtälön $y=2x-1$, se on suoralla.
Yhtälö siis täysin määrittää suoran pisteet.
Etsittäessä pisteitä, jotka muodostavat suoran, riittää tutkia sen yhtälöä.

Monilla tutuilla käyrillä on omat yhtälönsä.
Alle on piirretty paraabeli, jonka yhtälö on $y=x^2-2x+1$.
Aivan kuten suoran tapauksessa, paraabeli koostuu täsmälleen niistä pisteistä, jotka toteuttavat tämän yhtälön.
Toisinaan sanotaan, että paraabeli on \emph{niiden pisteiden joukko, jotka toteuttavat yhtälön $y=x^2-2x+1$}.

[TÄHÄN PARAABELI]

Tällä kurssilla opitaan käsittelemään erityisesti suoria ja ympyröitä niiden yhtälöiden avulla.
Ympyrän yhtälö muodostetaan vaatimuksesta, että jokainen ympyrän piste on yhtä kaukana ympyrän keskipisteestä.
Tällöin tulee käyttöön edellä opittu kahden pisteen etäisyyden kaava.
Esimerkiksi alla olevassa kuvassa on piirretty ympyrä, jonka yhtälö on $(x-1)^2+(y-2)^2=4$.

[TÄHÄN YMPYRÄ]

Yllä olevasta ympyrän yhtälöstä nähdään, että käyrien yhtälöt eivät ole välttämättä muotoa $y=f(x)$.
Toinen esimerkki on niin sanottu Cartesiuksen lehti, jonka yhtälö on $x^3+y^3-3xy=0$.
Kartesiuksen lehti on piirretty alla olevaan kuvaan.

[TÄHÄN CARTESIUKSEN LEHTI]

Esimerkiksi piste $(3/2, 3/2)$ on käyrällä, sillä
\[
\left(\frac{3}{2}\right)^3+\left(\frac{3}{2}\right)^3-3\cdot\frac{3}{2}\cdot\frac{3}{2}
=\frac{27}{8}+\frac{27}{8}-\frac{27}{4}=\frac{27+27-54}{8}=0.
\]

\begin{tehtavasivu}

\subsubsection*{Opi perusteet}

\subsubsection*{Hallitse kokonaisuus}

\subsubsection*{Sekalaisia tehtäviä}

TÄHÄN TEHTÄVIÄ SIJOITTAMISTA ODOTTAMAAN

\end{tehtavasivu}
	% yleistä käyristä, esim. Kartesiuksen lehdestä jotain
	% kahden pisteen välinen etäisyys (Pythagoraan lauseella)

\chapter{Suorat}
	\section{Suoran yhtälö}

\laatikko{
KIRJOITA TÄHÄN LUKUUN

\begin{itemize}
\item suoran yhtälö muodossa $y=kx+b$
\item kulmakertoimen ja vakiotermin merkitys
\item pysty- ja vaakasuoran suoran yhtälö
\item suorien leikkauspisteen/suoran ja $x$-akselin leikkauspisteen ratkaiseminen
\end{itemize}

KIITOS!}

Alla on kolme kuvaa, joihin on seuraavia yhtälöitä vastaavat kuvaajat:
\begin{align*}
y & =x-1 \\
y & =2x \\
\text{ja} \quad y & =-x+2.
\end{align*}
Kuten huomataan, kaikki kolme kuvaajaa ovat suoria.

\begin{kuva}
    kuvaaja.pohja(-3.5, 3.5, -3.5, 3.5, korkeus = 4, nimiX = "$x$", nimiY = "$y$", ruudukko = True)
    kuvaaja.piirra("x-1", nimi = "$y=x-1$")
    kuvaaja.piirra("2*x", nimi = "$y=2x$", suunta = 45)
    kuvaaja.piirra("-x+2", nimi = "$y=-x+2$", kohta = -0.6, suunta = -135)
\end{kuva}

\laatikko{
Yhtälön
\[
y=kx+b
\]
määräämä kuvaaja on suora. Lukua $k$ nimitetään suoran \termi{kulmakerroin}{kulmakertoimeksi} ja lukua $b$ \termi{vakiotermi}{vakiotermiksi}.
}


%%%%%%FIXME Pitäisikö tässä olla kolme pistettä, kun käytännössä aika usein kolmas olisi hyvä, koska huolimattomuusvirhe

\begin{esimerkki} Piirretään koordinaatistoon yhtälön $y=2x-3$ kuvaaja. Koska kyseessä on suoran yhtälö, riittää löytää kaksi pistettä, joiden kautta suora kulkee.
Valitaan esimerkiksi pisteet, joiden $x$-koordinaatit ovat $0$ ja $2$. Ensimmäisen $y$-koordinaatti on
\[
y=2\cdot 0-3=-3
\]
ja toisen
\[
y=2\cdot 2-3=4-3=1.
\]
Suoran pisteet ovat siis $(0, -3)$ ja $(2, 1)$. Piirretään nämä koordinaatistoon ja vedetään niiden kautta suora.

\begin{kuva}
    kuvaaja.pohja(-1.5, 3.5, -3.5, 3.5, korkeus = 4, nimiX = "$x$", nimiY = "$y$", ruudukko = True)
    kuvaaja.piirra("2*x-3", nimi = "$y=2x-3$")
    piste((0, -3))
    piste((2, 1))
\end{kuva}

\end{esimerkki}

\subsubsection*{Kulmakertoimen tulkinta}

Suoran kulmakerroin kertoo, miten jyrkästi suora nousee tai laskee. Tarkastellaan alla olevaa suoraa, jonka yhtälö on $y=2x$.

\begin{kuva}
    kuvaaja.pohja(-1, 3, -1, 5, korkeus = 4, nimiX = "$x$", nimiY = "$y$", ruudukko = True)
    kuvaaja.piirra("2*x", nimi = "$y=2x$")
    piste((1, 2))
    piste((2, 4))
\end{kuva}


Valitaan suoralta kaksi pistettä,
$A=(1, 2)$ ja $B=(2, 4)$. Siirryttäessä pisteestä $A$ pisteeseen $B$ $x$-koordinaatin arvo kasvaa yhdellä ja $y$-koordinaatin arvo kahdella. Saadaan suhde
\[
\frac{\text{$y$-koordinaatin muutos}}{\text{$x$-koordinaatin muutos}}=\frac{2}{1}=2.
\]
Jos nyt valitaan suoralta jotkin toiset pisteet, esimerkiksi $D=(-1, -2)$ ja $E=(5, 10)$, voidaan laskea samalla tavalla
\[
\frac{\text{$y$-koordinaatin muutos}}{\text{$x$-koordinaatin muutos}}=\frac{10-(-2)}{5-(-1)}=\frac{12}{6}=2.
\]
Huomataan, että yllä laskettu suhde on aina sama pisteistä riippumatta. Tämä johtuu siitä, että kuvan kolmiot $ABC$ ja $DEF$ ovat yhdenmuotoisia.
Suhde on lisäksi sama kuin suoran yhtälössä esiintyvä kulmakerroin.

\begin{esimerkki} Määritä alla olevien suorien kulmakertoimet.

%%% pitäisikö pisteet olla piirrettyinä kuvaan?

\begin{kuva}
    kuvaaja.pohja(-2.5, 3, -1, 5, korkeus = 4, nimiX = "$x$", nimiY = "$y$", ruudukko = True)
    kuvaaja.piirra("3*x+1")
    kuvaaja.piirra("-0.5*x+3")
    piste((0, 1))
    piste((1, 4))
    piste((-2, 4))
    piste((2, 2))
\end{kuva}

\begin{esimratk} Valitaan suorilta kaksi mielivaltaista pistettä ja lasketaan $y$-koordinaatin muutoksen suhde $x$-koordinaatin muutokseen.
Ensimmäiseltä suoralta valitaan vaikkapa pisteet $(0, 2)$ ja $(1, 4)$. Nyt kulmakertoimeksi tulee
\[
\frac{\text{$y$-koordinaatin muutos}}{\text{$x$-koordinaatin muutos}}=\frac{4-1}{1-0}=\frac{3}{1}=3.
\]
Toiselta suoralta valitaan pisteet $(-2, 4)$ ja $(2, 2)$. Nyt täytyy olla tarkkana etumerkkien kanssa:
\[
\frac{\text{$y$-koordinaatin muutos}}{\text{$x$-koordinaatin muutos}}=\frac{2-4}{2-(-2)}=\frac{-2}{4}=-\frac{1}{2}.
\]
\end{esimratk}

\begin{esimvast}
Ensimmäisen suoran kulmakerroin on $3$ ja toisen $-\frac{1}{2}$.
\end{esimvast}
\end{esimerkki}

Kulmakerroin kertoo suoran suunnasta: mitä suurempi kulmakerroin, sitä jyrkemmin suora nousee koordinaatistossa oikealle päin.
Jos kulmakerroin on negatiivinen, suora on laskeva. Vaakasuoran suoran kulmakerroin on 0.

\subsubsection*{Vakiotermin tulkinta}

Yllä nähtiin, että suora $y=2x$ kulkee origon kautta. Seuraavassa kuvassa on suoran $y=2x+1$ kuvaaja. Se saadaan nostamalla suoraa $y=2x$ yhden yksikön verran ylöspäin.

\begin{kuva}
    kuvaaja.pohja(-1, 3, -1, 5, korkeus = 4, nimiX = "$x$", nimiY = "$y$", ruudukko = True)
    kuvaaja.piirra("2*x+1", nimi = "$y=2x+1$", suunta = 45)
    vari("lightgray")
    kuvaaja.piirra("2*x")
\end{kuva}

Tarkastellaan suoralla $y=kx+b$ olevaa pistettä, jonka $x$-koordinaatti on 0.
Tämä piste sijaitsee $y$-akselilla. Toisin sanoen se on suoran ja $y$-akselin leikkauspisteessä.
Sen $y$-koordinaatti saadaan laskemalla
\[
y=k\cdot 0+b=b.
\]
Pisteen $y$-koordinaatti on siis $b$, eli suoran yhtälön vakiotermi. Vakiotermi siis ilmaisee, missä kohtaa suora leikkaa $y$-akselin.
Alla on esimerkkejä erilaisista vakiotermeistä.

%%%% Nyt on samat kulmakertoimet kuin aiemmissa esimerkeissä.
%%%% Jos kulmakerroin olisi esim. 0.5, niin nimet saisi oikealle puolelle nätisti

\begin{kuva}
    kuvaaja.pohja(-2, 3, -1, 5, korkeus = 4, nimiX = "$x$", nimiY = "$y$", ruudukko = True)
    kuvaaja.piirra("2*x+1", nimi = "$y=2x+1$", suunta = 45)
    kuvaaja.piirra("2*x+3", nimi = "$y=2x+3$", kohta = -2, suunta = -135)
    kuvaaja.piirra("2*x-1", nimi = "$y=2x-1$", suunta = -45)
    kuvaaja.piirra("2*x-5", nimi = "$y=2x-5$", suunta = 0)
\end{kuva}

\subsubsection*{Suoran yhtälön määrittäminen}

Suoran yhtälö voidaan määrittää kuvasta laskemalla suoran kulmakerroin sekä vakiotermi.

\begin{esimerkki} Mikä on alla olevan kuvan suoran yhtälö?

\begin{kuva}
    kuvaaja.pohja(-1, 4.5, -0.5, 3.5, korkeus = 4, nimiX = "$x$", nimiY = "$y$", ruudukko = True)
    kuvaaja.piirra("0.5*x+1")
    piste((0, 1), "(0, 1)", 135)
    piste((4, 3), "(4, 3)", 135)
\end{kuva}


\begin{esimratk}
Aloitetaan määrittämällä kulmakerroin. Valitaan suoralta pisteet $(0, 1)$ ja $(4, 3)$. Kulmakertoimeksi tulee
\[
k=\frac{3-1}{4-0}=\frac{2}{4}=\frac{1}{2}\text{.}
\]
\end{esimratk}
Vakiotermi saadaan kohdasta, jossa suora leikkaa $y$-akselin. Tuossa kohdassa $y$-koordinaatti on 1. Vakiotermi on siis $b=1$.
\begin{esimvast}
Suoran yhtälö on $y=\frac{1}{2}x+1$.
\end{esimvast}
\end{esimerkki}

Kun kulmakerroin on 0, suora on vaakasuora. Sen yhtälö on siis muotoa
\laatikko[Vaakasuora suora]{
\[
y=b.
\]
}
Toisaalta pystysuoralla suoralla ei ole kulmakerrointa lainkaan. Sen yhtälöä ei voi ilmaista muodossa $y=\dots$, vaan sillä on yhtälö
\laatikko[Pystysuora suora]{
\[
x=a.
\]
}
Tässä $a$ on sen pisteen $x$-koordinaatti, jossa suora leikkaa $x$-akselin.

\begin{tehtavasivu}

\subsubsection*{Opi perusteet}

\begin{tehtava}
Ratkaise suoran $y=3x+1$ nollakohta.
\begin{vastaus}
$x=-\frac{1}{3}$
\end{vastaus}
\end{tehtava}

\begin{tehtava}
Ratkaise suorien $y=-5x+3$ ja $y=2x-17$ leikkauspiste.
\begin{vastaus}
% http://www.wolframalpha.com/input/?i=y%3D-5x%2B3%2C+y%3D2x-17
$(\frac{20}{7}, -\frac{79}{70})$
\end{vastaus}
\end{tehtava}

\begin{tehtava}
Mikä on $x$-akselin suuntaisen suoran, joka kulkee pisteen $(1, 3)$ kautta, yhtälö?
\begin{vastaus}
$y=3$
\end{vastaus}
\end{tehtava}

\begin{tehtava}
Mikä on suoran $y=3,14x-10$ yhtälön
\begin{enumerate}[a)]
\item vakiotermi,
\item kulmakerroin?
\end{enumerate}
\begin{vastaus}
a)$-10$ b) $3,14$
\end{vastaus}
\end{tehtava}

\begin{tehtava}
Suora kulkee pisteiden $(2, 1)$ ja $(5, 9)$ kautta. Määritä kulmakerroin.
\begin{vastaus}
Kulmakerroin on $\frac{8}{3}$
\end{vastaus}
\end{tehtava}

\begin{tehtava}
Piirrä suora $y=9x-1$.
\begin{vastaus}
puuttuu
\end{vastaus}
\end{tehtava}

\subsubsection*{Hallitse kokonaisuus}

\begin{tehtava}
Ratkaise suorien $y=-x+2$ ja $y=2x-4$ leikkauspiste.
\begin{vastaus}
$(2, 0)$
\end{vastaus}
\end{tehtava}

\begin{tehtava}
Määritä
\begin{enumerate}[a)]
\item $x$-akselin suuntaisen suoran,
\item $y$-akselin suuntaisen suoran kulmakerroin?
\end{enumerate}
\begin{vastaus}
a) $0$ b) ei määritelty %(ääretön)
\end{vastaus}
\end{tehtava}

\begin{tehtava}
Piirrä suora $y=-2x+3$.
\begin{vastaus}
puuttuu
\end{vastaus}
\end{tehtava}

\begin{tehtava}
Ratkaise $\frac{y}{2}=\frac{x}{2}+2$ nollakohta.
\begin{vastaus}
$(-4, 0)$
\end{vastaus}
\end{tehtava}

\begin{tehtava}
Ratkaise suoran $6=-60x+600y$ nollakohta.
\begin{vastaus}
$x=-\frac{1}{10}$
\end{vastaus}
\end{tehtava}

\begin{tehtava}
Missä pistessä suora $y=\frac{16x}{25}+\frac{36}{49}$
\begin{enumerate}[a)]
\item leikkaa $x$-akselin,
\item leikkaa $y$-akselin?
\end{enumerate}
\begin{vastaus}
a)$(-\frac{225}{196}, 0)$ b) $(0, \frac{36}{49})$
\end{vastaus}
\end{tehtava}

\begin{tehtava}
Ratkaise suorien $y=-\frac{2}{5}$ ja $3y=18x+20$ leikkauspiste.
\begin{vastaus}
$(-\frac{53}{45}, -\frac{2}{5})$
\end{vastaus}
\end{tehtava}

\begin{tehtava}
Ratkaise suoran $16y-9x=-5y-11x+27$ nollakohta.
\begin{vastaus}
$(11, 0)$
\end{vastaus}
\end{tehtava}

\subsubsection*{Sekalaisia tehtäviä}

LAITA TEHTÄVÄT TÄHÄN, JOS ET OLE VARMA VAIKEUSASTEESTA TAI TEHTÄVÄ
EI TÄLLÄ HETKELLÄ SOVI MUKAAN

\end{tehtavasivu}

	% ratkaistu muoto y = kx + b, kulmakerroin ja vakiotermi
	% nollakohdat ja leikkauspisteet
	% vaaka- ja pystysuorat
	\section{Suoran yhtälön muut muodot}

\laatikko{
KIRJOITA TÄHÄN LUKUUN

\begin{itemize}
\item suoran yhtälö normaalimuodossa $ax+by+c=0$
\item suoran yhtälö muodossa $y-y_0=k(x-x_0)$, suoran yhtälön muodostaminen pisteiden avulla
\end{itemize}

KIITOS!}

\subsection*{Valeorigomuoto} % tämä ei ole vakiintunut termi!!!

Toisinaan suoran yhtälöä on helpompaa tarkastella muodossa $y-y_0=k(x-x_0)$.
Tämän voidaan ajatella olevan origon kautta kulkeva suora, jos origo olisi
pisteessä $(x_0, y_0)$. Valeorigomuoto on kätevin silloin, kun tiedämme suoran
kulmakertoimen ja yhden pisteen, jonka kautta suora kulkee.

\begin{esimerkki}
    Esimerkkejä valeorigomuodon käytöstä:
    \begin{enumerate}[a)]
        \item Suoran kulmakerroin on $5$ ja suora kulkee pisteen $(3,0)$ kautta.
        \[y-y_0 = k(x-x_0) \ekvi y-0 = 5(x-3) \ekvi y = 5x-15\]
        \item Suoran kulmakerroin on $4$ ja suora kulkee pisteen $(5,7)$ kautta.
        \[y-y_0 = k(x-x_0) \ekvi y-7 = 4(x-5) \ekvi y-7 = 4x-20 \ekvi y = 4x-13\]
    \end{enumerate}
\end{esimerkki}

\subsection*{Normaalimuoto}

Suoran yhtälön kanonisin muoto on normaalimuoto tai yleinen muoto $ax+by+c=0$.

\begin{tehtavasivu}

\subsubsection*{Opi perusteet}

\begin{tehtava}
Mikä on suoran yhtälö normaalimuodossa?
\begin{enumerate}[a)]
\item $y=-15x+2$
\item $2y=11x+7$
\item $2y+5x-8=13y-6x-8$
\end{enumerate}
\begin{vastaus}
a)$15x+y-2=0$ b) $11x-2y+7=0$ c) $-11x+11y=0$
\end{vastaus}
\end{tehtava}


\begin{tehtava}
Suoran kulmakerroin on $\frac{1}{2}$ ja suora kulkee pisteen
\begin{enumerate}[a)]
\item $(-12,4)$
\item $(3,9)$. Mikä on suoran yhtälö?
\end{enumerate}
\begin{vastaus}
a)$y=\frac{1}{2}x+10$ b) $y=\frac{1}{2}x+\frac{15}{2}$
\end{vastaus}
\end{tehtava}

\begin{tehtava}
Mikä on pisteiden
\begin{enumerate}[a)]
\item $(1,-2)$ ja $(3,1)$
\item $(0,0)$ ja $(-4,4)$ kautta kulkevan suoran yhtälö?
\end{enumerate}
\begin{vastaus}
a)$y=\frac{3}{2}x-\frac{7}{2}$ b) $y=-x$
\end{vastaus}
\end{tehtava}

\subsubsection*{Hallitse kokonaisuus}

\begin{tehtava}
Tutki ovatko pisteet  
\begin{enumerate}[a)]
\item $(1,-5)$, $(4,-23)$ja $(4,-239)$
\item $(7,3)$, $(-2,10)$ ja $(-3,90)$ samalla suoralla?
\end{enumerate}
\begin{vastaus}
a) kyllä b) ei
\end{vastaus}
\end{tehtava}

\begin{tehtava}
Määritä luku $t$ niin, että pisteet $(-t+3,-4)$, $(6,t-5)$ ja $(5,-4)$ ovat samalla suoralla.
\begin{vastaus}
$t=-2$ tai $t=1$
\end{vastaus}
\end{tehtava}

\subsubsection*{Sekalaisia tehtäviä}

TÄHÄN TEHTÄVIÄ SIJOITTAMISTA ODOTTAMAAN

\begin {tehtava}
Suora kulkee pisteiden $(3,4)$ ja $(\sqrt{3},1)$ kautta. Määritä suoran kulmakerroin.
\begin {vastaus}
$\frac{\sqrt{3}-1}{\sqrt{3}}$
\end {vastaus}
\end {tehtava}

\end{tehtavasivu}

	% esitys y-y_0=k(x-x_0)
	% esitys ax + by + c = 0 (normaalimuoto)
	\section{Suorien keskinäinen asema}

\laatikko{
KIRJOITA TÄHÄN LUKUUN

\begin{itemize}
\item sama kulmakerroin --> yhdensuuntaiset tai sama suora
\item eri kulmakerroin --> yksi leikkauspiste, kytkentä yhtälöpareihin
\item suoran ja normaalin kulmakertoimet, $k_1k_2=-1$.
\end{itemize}

KIITOS!}
	% suorien keskinäinen asema, yhdensuuntaiset suorat
	% suoralle ja sen normaalille k_1 * k_2 = -1
	\section{Pisteen etäisyys suorasta}

\laatikko{
KIRJOITA TÄHÄN LUKUUN

\begin{itemize}
\item pisteen etäisyyden suorasta laskeminen yhdenmuotoisilla
kolmioilla
\item pisteen etäisyys suorasta -kaava: $d=\frac{|ax_0+by_0+c|}{\sqrt{a^2+b^2}}$
\item sovelluksia
\item kaavan todistuksen voi laittaa tähän osioon tai liitteeksi,
käytetään yhdenmuotoisia kolmioita
\end{itemize}

KIITOS!}

Pisteen etäisyydellä suorasta tarkoitetaan pisteen ja mielivaltaisen suoran pisteen pienintä mahdollista etäisyyttä.
Jos tunnetaan jokin vaaka- tai pystysuora suora ja jokin koordinaatiston piste, kyseisen pisteen etäisyys annetusta suorasta on helppo määrittää.

[TÄHÄN SELITYS JA KUVA]

Jos suora on kalteva, etäisyyden määrittäminen ei ole näin suoraviivaisesti.
Seuraavaksi tutustutaan kahteen tapaan tämän pulman ratkaisemiseksi.

\subsection*{Pisteen etäisyys suorasta yhdenmuotoisten kolmioiden avulla}

Tarkastellaan esimerkkinä suoraa $l$, jonka normaalimuotoinen yhtälö on $3x-4y=12$. [TÄMÄ EI MUKA OLE NORMAALIMUOTO. KAI KORJATTAVA]
Selvitetään pisteen $P=(8, 5)$ etäisyys suorasta $l$.

[TÄHÄN KUVA]

Kuvaan on merkitty suorakulmaiset kolmiot $OAB$ ja $PQR$.
Etäisyys, jonka haluamme selvittää, on kolmion $PQR$ sivun $r$ pituus.
Tehtävä ratkeaa, kun huomataan, että kolmiot $OAB$ ja $PQR$ ovat yhdenmuotoisia.
Tämä johtuu siitä, että molemmat ovat suorakulmaisia ja lisäksi kulmat $OAB$ ja $PRQ$ ovat samankokoiset (ks. kuva alla).

[TÄHÄN KUVA]

Kolmion $OAB$ sivut selviävät, kun ratkaistaan, missä pisteissä suora leikkaa $x$- ja $y$-akselit.
Asettamalla suoran yhtälössä $x=0$ suoran yhtälössä
\[
3x-4\cdot 0=12, \quad \text{josta} \quad x=\frac{12}{3}=4.
\]
Pisteen $A$ koordinaatit ovat siis $(4, 0)$. Toisaalta kun $x=0$, saadaan
\[
3\cdot 0-4\cdot y=12, \quad \text{josta} \quad y=-\frac{12}{4}=3.
\]
Pisteen $B$ koordinaatit ovat siis $(0, 3)$. Nyt tunnetaan sivut $a=4$ ja $b=3$, ja lisäksi Pythagoraan lauseen perusteella
\[
c=\sqrt{a^2+b^2}=\sqrt{4^2+3^2}=\sqrt{25}=5.
\]

Koska kolmiot $OAB$ ja $PQR$ ovat yhdenmuotoiset, saadaan verranto
\[
\frac{r}{q}=\frac{a}{c}.
\]
Tunnemme jo sivut $a$ ja $c$, joten enää on selvitettävä sivu $q$. Tämä on sama kuin pisteiden $P$ ja $R$ välinen etäisyys.

Pisteen $R$ $x$-koordinaatti on sama kuin pisteen $P$, eli 8. Koska $R$ on suoralla $l$, sen $y$-koordinaatti saadaan suoran yhtälöstä:
\begin{align*}
3\cdot 8-4y & =12 \\
-4y & =12-3\cdot 8 \\
-4y & =-12 \\
y & =3. \\
\end{align*}
Nyt siis $R=(6, 3)$. Pisteiden $P$ ja $R$ välinen etäisyys on siis $5-3=2$, ja tämä on sivun $q$ pituus.

Kun verrantoon $\dfrac{r}{q}=\dfrac{a}{c}$ sijoitetaan tunnetut sivujen pituudet, saadaan
\begin{align*}
\frac{r}{2} & =\frac{4}{5} \quad \ppalkki \cdot 2 \\[3pt]
r & =\frac{8}{5}.
\end{align*}
Siispä pisteen $P$ etäisyys suorasta $l$ on $\dfrac{8}{5}$.

\subsection*{Pisteen etäisyys suorasta kaavan avulla}

Edellä esitetystä tavasta laskea pisteen etäisyys suorasta voidaan johtaa myös kaava.
Jos suoran yhtälö on annettu normaalimuodossa $Ax+By+C=0$ ja pisteen koordinaatit ovat $(x_0, y_0)$, etäisyys $d$ saadaan seuraavasta kaavasta.
\laatikko[pisteen etäisyys suorasta]{
\[
d=\frac{|Ax_0+By_0+C|}{\sqrt{A^2+B^2}}
\]
}
Kaavan johtaminen esitetään liitteessä. (VAI TÄSSÄ?)

\begin{esimerkki} Lasketaan aiemman esimerkin pisteen $P=(8, 5)$ etäisyys suorasta $l$, jonka normaalimuotoinen yhtälö on $3x-4y-12=0$.
\begin{esimratk}
Käytetään kaavaa, jolloin $A=3$, $B=-4$ ja $C=12$, sekä $x_0=8$ ja $y_0=5$. Kaavan mukaan etäisyys on
\[
d=\frac{|Ax_0+By_0+C|}{\sqrt{A^2+B^2}}
=\frac{|3\cdot 8-4\cdot 5-12|}{\sqrt{3^2+(-4)^2}}
=\frac{|24-20-12|}{\sqrt{9+16}}=\frac{|-8|}{\sqrt{25}}
=\frac{8}{5}.
\]
\end{esimratk}
\begin{esimvast}
Etäisyys on $\dfrac{8}{5}$.
\end{esimvast}
\end{esimerkki}

\begin{esimerkki} Etsitään ne pisteet, joiden $x$-koordinaatti on 6 ja joiden etäisyys suorasta $-6x+8y+3=0$.
\begin{esimratk}
Piste, jonka $x$-koordinaatti on 6, on muotoa $(6, y)$. Sijoitetaan tämä etäisyyden kaavaan ja sievennetään.
Nyt $A=-6$, $B=8$, $C=3$, $x_0=6$ ja $y_0=y$.
\[
d=\frac{|-6\cdot 6+8y+3|}{\sqrt{(-6)^2+8^2}}
=\frac{|-36+8y+3|}{\sqrt{36+64}}
=\frac{|8y-33|}{\sqrt{100}}
=\frac{|8y-33|}{10}.
\]
Tehtävänannon mukaan etäisyyden pitäisi olla $d=6$. Tästä saadaan yhtälö
\[
\frac{|8y-33|}{10}=6 \quad \text{eli} \quad |8y-33|=60.
\]
Tämä itseisarvoyhtälö ratkeaa jakautumalla kahteen tapaukseen:
\begin{align*}
8y-33 & =60 & &\text{tai} & 8y-33 & =-60 \\
8y & =99 & & & 8y & =-27 \\
y & =\frac{99}{8} & & & y & =-\frac{27}{8}.
\end{align*}
\end{esimratk}
\begin{esimvast}
Pisteet ovat $\bigl(6, \frac{99}{8}\bigr)$ ja $\bigl(6, -\frac{27}{8}\bigr)$.
\end{esimvast}
\end{esimerkki}



\subsection*{Pisteen etäisyys suorasta, kaavan todistus}

Lasketaan pisteen $P = (x_0, y_0)$ etäisyys suorasta $l$: $ax+by+c=0$.

Olkoon $Q$ suoralla $l$ siten, että $l$ ja $PQ$ ovat kohtisuorassa. Huomataan, että jos piste $R$ on suoralla $l$, Pythagoraan lauseen mukaan
\[
PR^2 = PQ^2+QR^2 \geq PQ^2,
\]
jolloin myös $PR \geq PQ$. Siis $PQ$ on määritelmän nojalla pisteen $P$ etäisyys suorasta $l$.

Suorat $PQ$ ja $l$ ovat kohtisuorassa. Jos $l$ ei ole $x$-, eikä $y$-akselin suuntainen, eli $a,b \neq 0$ sen kulmakerroin on $-\frac{a}{b}$. Koska suoran ja normaalin kulmakerrointen tulo on $-1$, suoran $PQ$ kulmakerroin on
\[
-\frac{1}{\frac{-a}{b}} = \frac{b}{a}.
\]
Se kulkee lisäksi pisteen $(x_0,y_0)$ kautta, joten sen yhtälö on
\[
y-y_0 = \frac{b}{a}(x-x0)
\]
tai normaalimuodossa
\[
bx-ay+ay_0-bx_0 = 0.
\]
Toisaalta, jos $a = 0$, suora $PQ$ on muotoa $x+C$, jollain reaaliluvulla $C$; koska se lisäksi kulkee pisteen $P$ kautta, sen on oltava edellistä muotoa. Sama päättely voidaan toistaa kun $b = 0$, jolloin nähdään, että kaava normaalille pätee myös jos $a = 0$ tai $b = 0$.

Koska $Q$ kuuluu suoralle $l$ ja sen normaalille, sen koordinaattien on toteutettava yhtälöpari
\[
\left\{    
    \begin{array}{rcl}
        ax_q + by_q + c &=&0 \\
        bx_q-ay_q+ay_0-bx_0 &=& 0 \\
    \end{array}
    \right.
\]
Yhtälöpari voidaan ratkaista yhtäänlaskumenetelmällä kertomalla ylempi yhtälö $a$:lla ja alempi $b$:llä.
\[
\left\{    
    \begin{array}{rcl}
        a^2x_q + aby_q + ac &=&0 \\
        b^2x_q-aby_q+aby_0-b^2x_0 &=& 0 \\
    \end{array}
    \right.
\]
ja laskemalle yhtälöt yhtälöt yhteen
\begin{align*}
 a^2x_q + aby_q + ac + b^2x_q-aby_q+aby_0-b^2x_0 &= 0 \\
 (a^2+b^2)x_q &= -ac-aby_0+b^2x0 \\
 x_q = \frac{-ac-aby_0+b^2x0}{(a^2+b^2)}.
\end{align*}
Vastaavasti
\[
y_q = \frac{-bc+a^2y_0-abx_0}{(a^2+b^2)}.
\]
Mutta nythän
\begin{align*}
PQ &= \sqrt{(x_q-x_0)^2+(y_q-y0)^2} \\
&= \sqrt{\Big(\frac{-ac-aby_0+b^2x_0-(a^2+b^2)x_0}{a^2+b^2}\Big)^2+\Big(\frac{-bc+a^2y_0-abx_0-(a^2+b^2)y_0}{a^2+b^2}\Big)^2} \\
& = \frac{\sqrt{(-ac-aby_0-a^2x_0)^2+(-bc-abx_0-b^2)y_0)^2}}{a^2+b^2} \\
& = \frac{\sqrt{(a(-c-by_0-ax_0))^2+(b(-c-ax_0-by_0))^2}}{a^2+b^2} \\
& = \frac{\sqrt{(a^2+b^2)((-c-by_0-ax_0))^2}}{a^2+b^2} \\
& = \frac{\sqrt{(a^2+b^2)}}{a^2+b^2}\sqrt{(-c-by_0-ax_0))^2} \\
& = \frac{|ax_0+by_0+c|}{\sqrt{a^2+b^2}}.
\end{align*}
\begin{tehtavasivu}

\subsubsection*{Opi perusteet}

\subsubsection*{Hallitse kokonaisuus}

\subsubsection*{Sekalaisia tehtäviä}

TÄHÄN TEHTÄVIÄ SIJOITTAMISTA ODOTTAMAAN

\end{tehtavasivu}
	% kaava pisteen etäisyydelle suorasta

\chapter{Toisen asteen käyrät}
	\chapter{Diskriminantti}

%
%Marginaaliin tai kuvaksi 2. asteen yhtälön ratkaisukaava (tai tekstin sekaan)
%
Toisen asteen yhtälön ratkaisukaavassa esiintyy neliöjuuri. Tämän neliöjuuren sisällä oleva lauseke $b^2-4ac$ määrää, kuinka monta ratkaisua yhtälöllä on. Joskus riittää pelkkä tieto ratkaisujen olemassaolosta tai lukumäärästä. Tälläisissä tapauksissa siis ei tarvitse ratkaista yhtälöä, vaan pelkkä edellä mainitun lausekkeen tarkastelu riittää. Lausekkeesta käytetään nimeä \emph{diskriminantti} ja sitä merkitään kirjaimella $D$.

\laatikko{Toisen asteen yhtälön $ax^2+bx+c=0$ ratkaisujen lukumäärän näkee diskriminantin, $D=b^2-4ac$ avulla seuraavasti.
\begin{itemize}
\item
Jos $D<0$, ei ole ratkaisuja.
\item
Jos $D=0$, on tasan yksi ratkaisu.
\item
Jos $D>0$, on 2 ratkaisua.
\end{itemize}
}

\begin{esimerkki}
Selvitä onko yhtälöllä $x^2+x+2=0$ ratkaisuja.

Tutkitaan diskriminanttia.
\[D=1^2-4\cdot 1 \cdot 2 = 1-8 = -7\]
Koska $D<0$, yhtälöllä ei ole ratkaisuja.

Jos yhtälön ratkaisua yrittäisi ratkaisukaavan avulla, tulisi neliöjuuren alle negatiivinen luku.
\end{esimerkki}

\begin{esimerkki}
Millä $a$:n arvolla yhtälöllä $9x^2+ax+1$ on tasan yksi ratkaisu.

Jotta ratkaisuja olisi tasan yksi, on diskriminantin oltava 0.
\begin{align*}
D &=0\\
a^2-4\cdot 9\cdot 1 &= 0\\
a^2-36&=0\\
a^2&=36\\
a=\pm6
\end{align*}
Yhtälöllä on täsmälleen yksi ratkaisu, jos $a=-6$ tai $a=6$.
\end{esimerkki}



\section{Harjoitustehtäviä}

	% diskriminantin pikakertaus
	\section{Ympyrä}

\laatikko{
KIRJOITA TÄHÄN LUKUUN

\begin{itemize}
\item ympyrän määritelmä ja siitä seuraava yhtälö,
origokeskinen ensin
\item muodon $x^2 + y^2 +ax +by +c=0$ täydentäminen neliöksi
ja ympyrän keskipisteen ja säteen selvittäminen siitä
\end{itemize}

KIITOS!}

\begin{tehtavasivu}

\subsubsection*{Opi perusteet}

\subsubsection*{Hallitse kokonaisuus}



\subsubsection*{Sekalaisia tehtäviä}

TÄHÄN TEHTÄVIÄ SIJOITTAMISTA ODOTTAMAAN

\end{tehtavasivu}
	% ympyrän yhtälö määritelmästä
	% ensin origokeskeinen
	% keskipiste ja säde muissa tapauksissa neliöksi täydentämällä
	\section{Ympyrä ja suora}

\laatikko{
KIRJOITA TÄHÄN LUKUUN

\begin{itemize}
\item ympyrän ja suoran leikkauspisteiden ratkaiseminen
\item ympyrän tangetin määrittäminen sekä kehällä olevan (kohtisuorassa sädettä vastaan) että
sen ulkopuolisen pisteen kautta (kaksi tapaa: pisteen etäisyys suorasta -kaava tai diskriminantti = 0)
\item kahden ympyrän leikkauspisteiden ratkaiseminen yhtälöparilla
\end{itemize}

KIITOS!}

\begin{esimerkki}
Määritä suorien $x+2y-3=0$ ja ympyrän $(x-1)^2+(y+1)^2=4 $ leikkauspisteet.

\begin{esimratk}

Ympyrän ja suoran leikkauspisteet ovat ne pisteet $(x,y)$, jotka ovat sekä suoralla että ympyrällä, eli toteuttavat molempien yhtälöt, eli yhtälöryhmän
$$\left\{    
    \begin{array}{rcl}
        x+2y-3 &=&0 \\
        (x-1)^2+(y+1)^2 &=&4 \\
    \end{array}
    \right.$$
Vaikka yhtälö ei olekaan tutun lineaarinen, myös sitä voi lähestyä sijoitusmenetelmällä. Ratkaistaan ensimmäisestä yhtälöstä $x$ ja saadaan
\[
x = -2y+3
\]
Kun tämä sijoitetaan toiseen yhtälöön ja kerrotaan auki syntyy toisen asteen yhtälö $y$:n suhteen:
\begin{align*}
(-2y+3-1)^2+(y+1)^2=4 \\
(-2y+2)^2+(y+1)^2=4 \\
(-2y)^2-2\cdot 2y\cdot 2 +2^2+y^2+2\cdot y+1^2=4 \\
4y^2-8y+4+y^2+2y+1-4 = 0 \\
5y^2-6y+1 = 0
\end{align*}
Ratkaisukaavalla
\[
y = \frac{-(-6)\pm\sqrt{(-6)^2-4\cdot 5\cdot 1}}{2\cdot5}
\]
eli
\begin{align*}
y = \frac{6\pm\sqrt{16}}{10} \\
y = 1 \vee y = \frac{1}{5}
\end{align*}
Kun nämä $y$:n arvot sijoitetaan suoran yhtälöön saadaan vastaavast $x$:n arvot:
\begin{align*}
x = -2\cdot 1+3 \vee x = -2\cdot\frac{1}{5}+3 \\
x = 1 \vee x = 2\frac{3}{5}
\end{align*}
eli saatiin 2 ratkaisua: $(x,y) = (1,1)$ ja $(x,y) = (2\frac{3}{5},\frac{1}{5})$. Tarkistamalla on hyvä vielä todeta, että pisteet todella ovat leikkauspisteitä.

Tähän kuva tilanteesta.
\begin{esimvast}
Leikkauspisteet ovat $(1,1)$ ja $(2\frac{3}{5},\frac{1}{5})$.
\end{esimvast}

\end{esimratk}
\end{esimerkki}

Esimerkin avulla huomattiin, että suoralla ja ympyrällä voi olla kaksi leikkauspistettä. Suorasta ja ympyrästä riippuen päädytään toisen asteen yhtälöön, jolla on joko 0, 1 tai 2 ratkaisua. Jos leikkauspisteitä on tasan yksi, sanotaan, että suora sivuaa ympyrää tai suora on ympyrän \termi{tangentti}{tangentti}.

Kuva ympyrästä ja kolmesta suorasta; yksi tangentti, yksi leikkaa kahdessa pisteessä ja yksi ei leikkaa ympyrää.

\laatikko{Suoralla ja ympyrällä on nolla, yksi tai kaksi leikkauspisteitä.

Suora on ympyrän \emph{tangentti}, jos suoralla ja ympyrällä on tasan yksi yhteinen piste.
}

\begin{esimerkki}

Määritä $(2,2)$-keskisen 5-säteisen ympyrän pisteen $(-5,3)$ kautta kulkevat tangentit.

\begin{esimratk}
Suora on ympyrän tangentti, jos sillä ja ympyrällä on tasan yksi yhteinen piste. Jos pisteen $(-5,3)$ suora ei ole $y$-akselin suuntainen, se voidaan esittää muodossa
\[
y-3 = k(x-(-5))
\]
eli
\[
y = kx+5k+3.
\]
Ympyrän yhtälö muistaen suoran ja ympyrän leikkauspisteille saadaan siis yhtälöpari
$$\left\{    
    \begin{array}{rcl}
        y &=&kx+5k+3\\
        (x-2)^2+(y-2)^2 &=& 25 \\
    \end{array}
    \right.$$
    
Sijoitetaan ensimmäisen yhtälön lauseke $y$:lle toiseen yhtälöön ja saadaan toisen asteen yhtälö $x$:n suhteen
\begin{align*}
(x-2)^2+(kx+5k+3-2)^2&=25 \\
(x-2)^2+(kx+5k+1)^2&=25 \\
x^2-2\cdot x\cdot 2 +2^2+(kx)^2+kx\cdot 5k+kx&\\
+5k\cdot kx+(5k)^2+5k+kx+5k+1-25& =0 \\
(1+k^2)x^2+(10k^2+2k-4)x+25k^2+10k-20& = 0. \\
\end{align*}
Tällä yhtälöllä on $x$:n suhteen tasan yksi ratkaisu, \\ jos polynomin diskriminantti on nolla, eli
\begin{align*}
& D = (10k^2+2k-4)^2-4\cdot(1+k^2)\cdot(25k^2+10k-20) \\ 
&= 100k^4+10k^2\cdot 2k+10k^2\cdot (-4) +2k\cdot 10k^2+(2k)^2+2k\cdot (-4) \\
& +(-4)\cdot 10k^2+(-4)\cdot 2k+(-4)^2+(-4)\cdot 25k^2+(-4)\cdot 10k+(-4)\cdot (-20) \\
& +(-4k^2)(25k^2)+(-4k^2)(10k)+(-4k^2)(-20) \\
& = 100k^4+20k^3-40k^2+20k^3+4k^2-8k-40k^2-8k+16-100k^2-40k+80 \\
& -100k^4-40k^3+80k^2 \\
& = (100-100)k^4+(20+20-40)k^3+(-40+4-40-100+80)k^2+(-8-8-40)k+(16+80) \\
& = -96k^2-56k+96 \\
&=-8(12k^2+7k-12) = 0.
\end{align*}
Toisen asteen yhtälön ratkaisukaavalla
\begin{align*}
k &= \frac{-7\pm \sqrt{7^2-4\cdot 12\cdot (-12)}}{2\cdot 12} \\
k &= \frac{-7\pm \sqrt{625}}{24} \\
k &= \frac{-7\pm 25 }{24}, \\
\end{align*}
joten
\[
k =  \frac{3}{4} \vee k = -\frac{4}{3}
\]
Kulmakertoimia vastaa siis tangenttisuorat
\begin{align*}
y = \frac{3}{4}x+5\cdot \frac{3}{4}+3 &\textrm{  ja  } y = -\frac{4}{3}x+5\cdot \Big(-\frac{4}{3}\Big)+3 \\
y = \frac{3}{4}x+6\frac{3}{4} &\textrm{  ja  } y = -\frac{4}{3}x-3\frac{2}{3}
\end{align*}

\begin{esimvast}
Suorat $y = \frac{3}{4}x+6\frac{3}{4}$ ja $y = -\frac{4}{3}x-3\frac{2}{3}$
\end{esimvast}
\end{esimratk}
\end{esimerkki}


\begin{tehtavasivu}

\subsubsection*{Opi perusteet}

\subsubsection*{Hallitse kokonaisuus}
\begin{tehtava}
	Määritä annetun ympyrän tangentit, jotka kulkevat annetun pisteen kautta.
	\begin{alakohdat}
		\alakohta{Piste $(1,-1)$, ympyrä $(x-3)^2 + (y+1)^2 = 2$}
		\alakohta{Piste $(-4,-2)$, ympyrä $(x+5)^2 + (y+6)^2 = 17$}
		\alakohta{Piste $(3,1)$, ympyrä $(x-4)^2 + y^2 = 1$}
		\alakohta{Piste $(2,4)$, ympyrä $x^2 + (y-3)^2 = 8$}
	\end{alakohdat}
	\begin{vastaus}
		\begin{alakohdat}
			\alakohta{$y=x-2$ ja $y=-x$}
			\alakohta{$y= -1/4 x -3$}
			\alakohta{$y=1$ ja $x=3$}
			\alakohta{Ei ole.}
		\end{alakohdat}
	\end{vastaus}
\end{tehtava}

\begin{tehtava}
Määritä yksikköympyrän $x^2+y^2= 1$ pisteeseen $(x_{0},y_{0} )$ piirretyn tangentin normaalimuotoinen yhtälö.
\begin{vastaus}
$x_0x+y_0y=1 $
\end{vastaus}
\end{tehtava}

\begin{tehtava}
Yksikköympyrälle $x^2+y^2=1$ piirretään tangentit pisteestä $(0,a)$ . 
\begin{alakohdat}
\alakohta{kaksi}
\alakohta{yksi}
\alakohta{nolla?}
\end{alakohdat}
Määritä myös tangenttien yhtälöt.
\begin{vastaus}
\begin{alakohdat}
\alakohta{$a > 1$ tai $a < -1$}
\alakohta{$a = \pm1$}
\alakohta{$ -1 < a < 1 $} 
\end{alakohdat}
Tangenttien yhtälöt ovat $ y = \pm \sqrt{a^2-1}x+a$
\end{vastaus}
\end{tehtava}

\subsubsection*{Sekalaisia tehtäviä}

TÄHÄN TEHTÄVIÄ SIJOITTAMISTA ODOTTAMAAN

\end{tehtavasivu}
	% suoran ja ympyrän leikkauspisteet
	% tangentit
	\section{Paraabeli}

\laatikko{
KIRJOITA TÄHÄN LUKUUN

\begin{itemize}
\item käyrän $y = ax^2+by+c=0$ kuvaaja on paraabeli
\item mainitaan geometrinen määritelmä
\item paraabelin yhtälön huippumuoto $y-y_0=a(x-x_0)^2$
\end{itemize}

KIITOS!}
	% merkitys toisen asteen polynomin kuvaajana
	% geometrisen määritelmän maininta
	\section{Paraabelin sovelluksia}

\laatikko{
KIRJOITA TÄHÄN LUKUUN

\begin{itemize}
\item paraabelin huippu on kohdassa $x=-b/2a$, todistus
\item paraabelin yhtälön huippumuoto $y-y_0=a(x-x_0)^2$
\item paraabelin yhtälön ratkaiseminen kolmen pisteen avulla
\item soveltavia tehtäviä, ne iänikuiset holvikaaret jne.
\end{itemize}

KIITOS!}

\begin{tehtavasivu}

\subsubsection*{Opi perusteet}

\subsubsection*{Hallitse kokonaisuus}

\subsubsection*{Sekalaisia tehtäviä}

TÄHÄN TEHTÄVIÄ SIJOITTAMISTA ODOTTAMAAN

\end{tehtavasivu}
	% huipun x-koordinaatti on -b/2a
		% todistus liitteeksi -Ville
		% todistus tähän -Niko
	% paraabelin tangentit
	\section{Vasemmalle ja oikealle aukeavat paraabelit}

\laatikko{
KIRJOITA TÄHÄN LUKUUN

\begin{itemize}
\item muotoa $x=ay^2+by+c$ olevat paraabelit aukeavat oikealle tai vasemmalla
\end{itemize}

KIITOS!}

\begin{tehtavasivu}

\subsubsection*{Opi perusteet}

\subsubsection*{Hallitse kokonaisuus}

\subsubsection*{Sekalaisia tehtäviä}

TÄHÄN TEHTÄVIÄ SIJOITTAMISTA ODOTTAMAAN

\end{tehtavasivu}
	% paraabeli x = ay^2  +by + c
	\section{Yleinen toisen asteen tasokäyrä}

\laatikko{
KIRJOITA TÄHÄN LUKUUN

\begin{itemize}
\item Tässä luvussa tarkastellaan lyhyesti yleisesti toisen asteen tasokäyriä, joista ympyrä ja paraabeli on jo käsitelty edellä
\item (kahden muuttujan) toisen asteen yhtälön määritelmä
\item joku tuttu esimerkki, vaikka paraabeli
\item esimerkkeinä yksi piste, kaksi suoraa, tyhjä
\item maininta siitä, että voi tulla myös ellipsi tai hyperbeli,
joista sitten joskus kirjoitetaan liite
\end{itemize}

KIITOS!}
	% esim. piste, kaksi suoraa, tyhjä
	\section{Sekalaista}

\laatikko{
KIRJOITA TÄHÄN LUKUUN

\begin{itemize}
\item kaikkea jännää kurssiin liittyen !
\end{itemize}

KIITOS!}

\begin{tehtavasivu}

\subsubsection*{Opi perusteet}

\subsubsection*{Hallitse kokonaisuus}

\subsubsection*{Sekalaisia tehtäviä}

TÄHÄN TEHTÄVIÄ SIJOITTAMISTA ODOTTAMAAN

\end{tehtavasivu}
	% esimerkkejä ja tehtäviä (erikoisia, vaikeita, yms.)

\chapter{Ulkoasukokeiluja}
	 
\section{Ulkoasukokeiluja}

%\definecolor{Sampo3}{cmyk}{0.2,0.27,0.0,0.0}
\definecolor{ParisGreen}{RGB}{80,200,120}
%\definecolor{PersianGreen}{RGB}{0,166,147}
%\definecolor{darkKhaki}{RGB}{189,183,107}
%\definecolor{cornflowerBlue}{RGB}{154,206,235}
%\definecolor{thulianPink}{RGB}{222,111,161}
%\definecolor{olivine}{RGB}{154,185,115}
%\definecolor{battleshipGray}{RGB}{132,132,130}

\pgfdeclarehorizontalshading{laatikkotaustaC}{100bp}{color(0bp)=(ParisGreen); color(50bp)=(ParisGreen); color(100bp)=(ParisGreen!50)}
%\pgfdeclarehorizontalshading{laatikkotausta}{100bp}{color(0bp)=(white); color(50bp)=(Sampo3); color(100bp)=(Sampo3!50)}
\pgfdeclarehorizontalshading{laatikkotaustaE}{195mm}{color(0mm)=(ParisGreen); color(97mm)=(ParisGreen); color(195mm)=(Sampo3)}
\pgfdeclarehorizontalshading{laatikkotaustaF}{100bp}{color(0bp)=(white); color(25bp)=(white); color(75bp)=(Sampo3); color(100bp)=(Sampo3)}
\colorlet{mycolor}{green}
\pgfdeclarehorizontalshading[mycolor]{myshadingB}{1cm}{rgb(0cm)=(1,0,0); color(1cm)=(white); color(2cm)=(Sampo3)}

\mdfdefinestyle{laatikkotyyliQ}{
  usetwoside=false,
  leftmargin=-46mm,
  rightmargin=-22mm,
  innerleftmargin=0mm,
  innerrightmargin=0mm,
  innertopmargin=0mm,
  innerbottommargin=0mm,
}

\mdfdefinestyle{laatikkotyyliB}{
%  userdefinedwidth=166mm % FIXME: controversial
  usetwoside=true,
  innermargin=15mm,
  outermargin=0mm,
  innerleftmargin=36mm,
  innerrightmargin=12mm,
  middlelinewidth=0pt,
  apptotikzsetting={\tikzset{mdfbackground/.append style = {shading = laatikkotausta}}},
}

\mdfdefinestyle{laatikkotyyliC}{
  userdefinedwidth=127mm,
  leftmargin=0pt,
  innermargin=0pt,
  innerleftmargin=0,
  rightmargin=0pt,
  innerrightmargin=0pt,
  middlelinewidth=0pt,
  apptotikzsetting={\tikzset{mdfbackground/.append style = {shading = laatikkotaustaC}}},
  innertopmargin=0mm,
  innerbottommargin=0mm,
  needspace=3\baselineskip,
}

\mdfdefinestyle{laatikkotyyliW}{
  usetwoside=false,
  leftmargin=-46mm,
  rightmargin=-22mm,
  innerleftmargin=46mm,
  innerrightmargin=22mm,
  innertopmargin=8pt,
  innerbottommargin=8pt,
  hidealllines=true,
  apptotikzsetting={\tikzset{mdfbackground/.append style = {shading = laatikkotaustaF}}},
  needspace=3\baselineskip,
}

\newcommand{\laatikkoB}[1]{%
%\begin{\widepar}
\ifthispageodd
{
  \begin{mdframed}[style=laatikkotyyliB]
    #1
  \end{mdframed}
}
{
  \begin{mdframed}[style=laatikkotyyliC]
    #1
  \end{mdframed}
}
%\end{widepar}
}


\newcommand{\laatikkoE}[1]{%
%\begin{\widepar}
  \begin{mdframed}[style=laatikkotyyliW]
	#1
  \end{mdframed}
%\end{widepar}
}

\laatikko{
	\begin{tabular}{l l}
		$|a|\geq0$ & Itseisarvo on aina ei-negatiivinen \\
		$|a|=|-a|$ & Luvun ja sen vastaluvun itseisarvot ovat yhtäsuuret \\
		$|a|^2=a^2$ & Luvun itseisarvon neliö on yhtäsuuri kuin luvun neliö \\
		$|\frac{a}{b}|=\frac{|a|}{|b|}$ & Osamaarän itseisarvo on itseisarvojen osamäärä

	\end{tabular}
}

\newpage

\section{Ulkoasukokeiluja 2}

\laatikko[Itseisarvon ominaisuuksia]{
	\begin{tabular}{l l}
		$|a|\geq0$ & Itseisarvo on aina ei-negatiivinen \\
		$|a|=|-a|$ & Luvun ja sen vastaluvun itseisarvot ovat yhtäsuuret \\
		$|a|^2=a^2$ & Luvun itseisarvon neliö on yhtäsuuri kuin luvun neliö \\
		$|\frac{a}{b}|=\frac{|a|}{|b|}$ & Osamaarän itseisarvo on itseisarvojen osamäärä

	\end{tabular}
}

\begin{tehtavasivu}
\begin{tehtava}
    Ratkaise
    \begin{enumerate}[a)]
        \item $x^2 - 2x - 3 = 0$
        \item $-x^2 - 6x - 5 = 0$
        \item $x + 2x^2 - 6= 0$
        \item $1 + x + 3x^2= 0$.
    \end{enumerate}
    \begin{vastaus}
        \begin{enumerate}[a)]
            \item $x = 3 tai x = -1$
            \item $x = -5 tai x = -1$
            \item $x = -1 + \sqrt{2} tai x = -1 - \sqrt{2}$
            \item Ei ratkaisuja.
        \end{enumerate}
    \end{vastaus}
\end{tehtava}

\begin{tehtava}
    Ratkaise
    \begin{enumerate}[a)]
        \item $9x^2 - 12x + 4 = 0$
        \item $x^2 + 2x = -4$
        \item $4x^2 = 12x - 8$
        \item $3x^2 - 13x + 50 = -2x^2 + 17x + 5$.
    \end{enumerate}
    \begin{vastaus}
        \begin{enumerate}[a)]
            \item $x = \dfrac{2}{3}$
            \item $x = -2$
            \item $x = 1$ tai $x = 2$
            \item $x = 3$
        \end{enumerate}
    \end{vastaus}
\end{tehtava}

\begin{tehtava}
    Tasaisesti kiihtyvässä liikkeessä on voimassa kaavat $v = v_0 + at$ ja $s = v_0t + \dfrac{1}{2}at^2$, missä $v$ on loppunopeus, $v_0$ alkunopeus, $a$ kiihtyvyys, $t$ aika ja $s$ siirtymä.
		\begin{enumerate}[a)]
            \item Auton nopeus on $72$~km/h. Auto pysäytetään jarruttamalla tasaisesti. Se pysähtyy $10$ sekunnissa. Laske jarrutusmatka.
            \item Kivi heitetään suoraan alas $50$ metriä syvään rotkoon nopeudella $3,0$~m/s. Kuinka monen sekunnin kuluttua se kohtaa rotkon pohjan?
        \end{enumerate}
    \begin{vastaus}
        \begin{enumerate}[a)]
            \item Jarrutusmatka on $100$ metriä.
            \item Noin $2,9$ sekunnin kuluttua.
        \end{enumerate}
    \end{vastaus}
\end{tehtava}

\begin{tehtava}
    Kahden luvun summa on $8$ ja tulo $15$. Määritä luvut.
    \begin{vastaus}
		Luvut ovat $3$ ja $5$.
    \end{vastaus}
\end{tehtava}

\begin{tehtava}
    Suorakulmaisen muotoisen alueen piiri on $34$~m ja pinta-ala $60$~m$^2$. Selvitä alueen mitat.
    \begin{vastaus}
		Alueen toinen sivu on $5$ m ja toinen $12$ m.
    \end{vastaus}
\end{tehtava}

\begin{tehtava}
    Kultaisessa leikkauksessa jana on jaettu siten, että pidemmän osan suhde lyhyempään on sama kuin koko janan suhde pidempään osaan. Tämä suhde ei riipu koko janan pituudesta ja sitä merkitään yleensä kreikkalaisella aakkosella fii eli $\varphi$. Kultaista leikkausta on taiteessa kautta aikojen pidetty ''jumalallisena suhteena''.
		\begin{enumerate}[a)]
            \item Laske kultaiseen leikkauksen suhteen $\varphi$ tarkka arvo ja likiarvo.
            \item Napa jakaa ihmisvartalon pituussuunnassa kultaisen leikkauksen suhteessa. Millä korkeudella napa on $170$~cm pitkällä ihmisellä?
        \end{enumerate}
    \begin{vastaus}
        \begin{enumerate}[a)]
            \item $ \varphi = \dfrac{\sqrt{5}-1}{2} \approx 0,618$
            \item Noin $105,1$ cm korkeudella.
        \end{enumerate}
    \end{vastaus}
\end{tehtava}

\begin{tehtava}
(K93/T5) Ratkaise yhtälö
        $\frac{2x+a^2-3a}{x-1}=a$ vakion $a$ kaikilla reaaliarvoilla.
\begin{vastaus}
        \begin{enumerate}
         \item{$x=a$, jos $a \neq 2$ ja $a \neq 1$}
         \item{$x\neq 1$, jos $a=2$}
         \item{ei ratkaisua, jos $a=1$}
        \end{enumerate}
    \end{vastaus}
\end{tehtava}

\begin{tehtava}
(K94/T2a) Polynomin $P(x)=ax^3-31x^2+1$ eräs nollakohta on $x=1$. Määritä $a$ ja ratkaise tämän jälkeen $P(x)=0$.
\begin{vastaus}
      $a=30$ yhtälön ratkaisut ovat $1$, $\frac{1}{5}$ ja $-\frac{1}{6}$.
    \end{vastaus}
\end{tehtava}

\begin{tehtava}
(K96/T2b) Yhtälössä $x^2-2ax+2a-1=0$ korvataan luku $a$ luvulla $a+1$. Miten muuttuvat yhtälön juuret?
\begin{vastaus}
     Toinen kasvaa kahdella ja toinen ei muutu.
    \end{vastaus}
\end{tehtava}

\begin{tehtava} % HANKALA!
	Ratkaise yhtälö $(x^2-2)^6=(x^2+4x+4)^3$.
	\begin{vastaus}
		$x=-1$, $x=0$ tai $x=\frac{1 \pm \sqrt{17}}{2}$
	\end{vastaus}
\end{tehtava}

\begin{tehtava}
    Ratkaise
    \begin{enumerate}[a)]
        \item $9x^2 - 12x + 4 = 0$
        \item $x^2 + 2x = -4$
        \item $4x^2 = 12x - 8$
        \item $3x^2 - 13x + 50 = -2x^2 + 17x + 5$.
    \end{enumerate}
    \begin{vastaus}
        \begin{enumerate}[a)]
            \item $x = \dfrac{2}{3}$
            \item $x = -2$
            \item $x = 1$ tai $x = 2$
            \item $x = 3$
        \end{enumerate}
    \end{vastaus}
\end{tehtava}

\begin{tehtava}
    Tasaisesti kiihtyvässä liikkeessä on voimassa kaavat $v = v_0 + at$ ja $s = v_0t + \dfrac{1}{2}at^2$, missä $v$ on loppunopeus, $v_0$ alkunopeus, $a$ kiihtyvyys, $t$ aika ja $s$ siirtymä.
		\begin{enumerate}[a)]
            \item Auton nopeus on $72$~km/h. Auto pysäytetään jarruttamalla tasaisesti. Se pysähtyy $10$ sekunnissa. Laske jarrutusmatka.
            \item Kivi heitetään suoraan alas $50$ metriä syvään rotkoon nopeudella $3,0$~m/s. Kuinka monen sekunnin kuluttua se kohtaa rotkon pohjan?
        \end{enumerate}
    \begin{vastaus}
        \begin{enumerate}[a)]
            \item Jarrutusmatka on $100$ metriä.
            \item Noin $2,9$ sekunnin kuluttua.
        \end{enumerate}
    \end{vastaus}
\end{tehtava}

\begin{tehtava}
    Kahden luvun summa on $8$ ja tulo $15$. Määritä luvut.
    \begin{vastaus}
		Luvut ovat $3$ ja $5$.
    \end{vastaus}
\end{tehtava}

\begin{tehtava}
    Suorakulmaisen muotoisen alueen piiri on $34$~m ja pinta-ala $60$~m$^2$. Selvitä alueen mitat.
    \begin{vastaus}
		Alueen toinen sivu on $5$ m ja toinen $12$ m.
    \end{vastaus}
\end{tehtava}

\begin{tehtava}
    Kultaisessa leikkauksessa jana on jaettu siten, että pidemmän osan suhde lyhyempään on sama kuin koko janan suhde pidempään osaan. Tämä suhde ei riipu koko janan pituudesta ja sitä merkitään yleensä kreikkalaisella aakkosella fii eli $\varphi$. Kultaista leikkausta on taiteessa kautta aikojen pidetty ''jumalallisena suhteena''.
		\begin{enumerate}[a)]
            \item Laske kultaiseen leikkauksen suhteen $\varphi$ tarkka arvo ja likiarvo.
            \item Napa jakaa ihmisvartalon pituussuunnassa kultaisen leikkauksen suhteessa. Millä korkeudella napa on $170$~cm pitkällä ihmisellä?
        \end{enumerate}
    \begin{vastaus}
        \begin{enumerate}[a)]
            \item $ \varphi = \dfrac{\sqrt{5}-1}{2} \approx 0,618$
            \item Noin $105,1$ cm korkeudella.
        \end{enumerate}
    \end{vastaus}
\end{tehtava}

\begin{tehtava}
(K93/T5) Ratkaise yhtälö
        $\frac{2x+a^2-3a}{x-1}=a$ vakion $a$ kaikilla reaaliarvoilla.
\begin{vastaus}
        \begin{enumerate}
         \item{$x=a$, jos $a \neq 2$ ja $a \neq 1$}
         \item{$x\neq 1$, jos $a=2$}
         \item{ei ratkaisua, jos $a=1$}
        \end{enumerate}
    \end{vastaus}
\end{tehtava}

\begin{tehtava}
(K94/T2a) Polynomin $P(x)=ax^3-31x^2+1$ eräs nollakohta on $x=1$. Määritä $a$ ja ratkaise tämän jälkeen $P(x)=0$.
\begin{vastaus}
      $a=30$ yhtälön ratkaisut ovat $1$, $\frac{1}{5}$ ja $-\frac{1}{6}$.
    \end{vastaus}
\end{tehtava}

\end{tehtavasivu}
