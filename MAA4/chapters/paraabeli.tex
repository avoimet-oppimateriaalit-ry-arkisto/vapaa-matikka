\section{Paraabeli}

\laatikko{
KIRJOITA TÄHÄN LUKUUN

\begin{itemize}
\item käyrän $y = ax^2+by+c$ kuvaaja on paraabeli
%%%terminologia? kuvaaja - funktion kuvaaja - käyrä ovatko samoja eivät?
\item mainitaan geometrinen määritelmä
\item paraabelin yhtälön huippumuoto $y-y_0=a(x-x_0)^2$
\end{itemize}

KIITOS!}

\emph{Paraabeli} on tason niiden pisteiden joukko, joiden etäisyys kiinteästä pisteestä, \emph{polttopisteestä} on sama kuin etäisyys kiinteästä suorasta, \emph{johtosuorasta}.

%%%%%%MAA2, luku 3.1 Toisen asteen polynomifunktio
Kussilla 2 mainittiin, että toisen asteen polynomifunktion kuvaaja on paraabeli. Nämä kuvaajat olivat muotoa $y=ax^2+bx+c$ olevia käyriä, joissa $a$ määräsi paraabelin aukeamissuunnan. Jos $a<0$ paraabeli aukeaa alaspäin ja jos $a>0$ ylöspäin.

\begin{kuva}
    kuvaaja.pohja(-1.5, 3.5, -0.5, 2.5, korkeus = 4, nimiX = "$x$", nimiY = "$y$", ruudukko = True)
    kuvaaja.piirra("0.5*x**2-x+0.25", a = -1.5, b = 3.5, nimi = "$y= 0,5x^2-x+0,25$", kohta = (3.2,2.1), suunta = (-1, 1))
\end{kuva}

\begin{kuva}
    kuvaaja.pohja(-1.5, 3.5, -0.5, 2.5, korkeus = 4, nimiX = "$x$", nimiY = "$y$", ruudukko = True)
    kuvaaja.piirra("-0.5*x**2+x+1.75", a = -1.5, b = 3.5, nimi = "$y= -0,5x^2+x+1,75$", kohta = (3.2,-0.5), suunta = (-1, 1))
\end{kuva}




\begin{tehtavasivu}

\subsubsection*{Opi perusteet}

\subsubsection*{Hallitse kokonaisuus}

\subsubsection*{Sekalaisia tehtäviä}

TÄHÄN TEHTÄVIÄ SIJOITTAMISTA ODOTTAMAAN

\end{tehtavasivu}