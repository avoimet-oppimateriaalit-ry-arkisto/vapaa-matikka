\section{Ympyrä}

\laatikko{
KIRJOITA TÄHÄN LUKUUN

\begin{itemize}
\item ympyrän määritelmä ja siitä seuraava yhtälö,
origokeskinen ensin
\item muodon $x^2 + y^2 +ax +by +c=0$ täydentäminen neliöksi
ja ympyrän keskipisteen ja säteen selvittäminen siitä
\end{itemize}

KIITOS!}

Ympyrä muodostuu pisteistä, jotka ovat kaikki jonkin saman etäisyyden päässä jostakin annetusta pisteestä. Tutkitaan vaikkapa niiden pisteiden joukkoa, joiden etäisyys origosta on 3. Toisin sanoen tutkitaan ympyrää, jonka keskipiste on origo ja säde 3. Pisteen $(x,y)$ etäisyys origosta on $\sqrt{x^2+y^2}$. (Miksi?) Jotta etäisyys olisi 3, täytyy päteä $\sqrt{x^2+y^2}=3$. Korotetaan vielä yhtälön molemmat puolet toiseen potenssiin, jolloin saadaan $x^2+y^2=9$. Tutkitun ympyrän yhtälö on siis $x^2+y^2=9$.

TÄHÄN KUVA.

Samalla tavalla voidaan johtaa yhtälö ympyrälle, jonka keskispiste on origo ja säde $r$.
\laatikko{Jos ympyrän keskipiste on origo ja säde $r$, ympyrän yhtälö on $x^2+y^2=r^2$.}

Jos ympyrän keskipiste ei ole origo, tulee yhtälöstä hieman monimutkaisempi. Oletetaan, että ympyrän keskipiste on $(x_0,y_0)$ ja säde $r$. Määritetään ympyrän yhtälö. Piste $(x,y)$ on ympyrälllä täsmälleen silloin, jos se etäisyys pisteestä $(x_0,y_0)$ on $r$. Luvun ?? perusteella pisteiden $(x,y)$ ja $(x_0,y_0)$ välinen etäisyys on $\sqrt{(x-x_0)^2+(y-y_0)^2}$. Tuloksena on siis yhtälö
\[
\sqrt{(x-x_0)^2+(y-y_0)^2}=r.
\]
Kun tämän yhtälön molemmat puolet korotetaan toiseen potenssiin, saadaan
\[
(x-x_0)^2+(y-y_0)^2=r^2.
\]

\laatikko{Jos ympyrän keskipiste on $(x_0,y_0)$ ja säde $r$, ympyrän yhtälö on
\[
(x-x_0)^2+(y-y_0)^2=r^2.
\]
}

\begin{esimerkki}
Ympyrän keskipiste on $(-4,1)$ ja säde $5$. Määritä ympyrän yhtälö.
\begin{esimratk}
Ympyrän yhtälö saadaan käyttämällä edellä annettua kaavaa. Nyt $x_0=-4$, $y_0=1$ ja $r=5$. Ympyrän yhtälöksi saadaan
\[
(x+4)^2+(y-1)^2=25.
\]
\end{esimratk}
\begin{esimvast}
Ympyrän yhtälö on $(x+4)^2+(y-1)^2=25$.
\end{esimvast}
\end{esimerkki}

Edellisen esimerkin ympyrän yhtälö voidaan kirjoittaa myös toisenlaisessa muodossa.
\begin{align*}
(x+4)^2+(y-1)^2&=25 \\
x^2+8x+16+y^2-2y+1&=25 \\
x^2+y^2+8x-2y-8&=0.
\end{align*}

\begin{tehtavasivu}

\paragraph*{Opi perusteet}

\paragraph*{Hallitse kokonaisuus}

\paragraph*{Sekalaisia tehtäviä}

TÄHÄN TEHTÄVIÄ SIJOITTAMISTA ODOTTAMAAN

\begin{tehtava}
Ympyrän kespiste on $(0,0)$ ja säde $5$. Muodosta ympyrän yhtälö.
\begin{vastaus}
$x^2+y^2=5$
\end{vastaus}
\end{tehtava}

\begin{tehtava}
Märitä keskipiste ja säde.
\begin{alakohdat}
  \alakohta{$(x-3)^2+(y+7)^2=12$}
	\alakohta{$x^2+y^2=49$}
\end{alakohdat}
\begin{vastaus}
\begin{alakohdat}
	\alakohta{keskpiste $(3,-7)$, säde $2\sqrt{3}$}
	\alakohta{keskipiste $(0,0)$, säde $7$}
\end{alakohdat}
\end{vastaus}
\end{tehtava}

\begin{tehtava}
Märitä keskipiste ja säde.
\begin{alakohdat}
	\alakohta{$x^2+y^2-10x+16y+72=0$}
	\alakohta{$x^2+y^2+8x-22y+129=0$}
\end{alakohdat}
\begin{vastaus}
\begin{alakohdat}
	\alakohta{keskpiste $(5,-8)$, säde $\sqrt{17}$}
	\alakohta{keskipiste $(-4,11)$, säde $2\sqrt{2}$}
\end{alakohdat}
\end{vastaus}
\end{tehtava}

\begin{tehtava}
Määritä ympyrän $(x+10)^2+y^2=2$ keskipiste ja säde ja ratkaise ympyrän yhtälöstä $y$. 
\begin{vastaus}
keskipiste $(-10,0)$, säde $\sqrt{2}$, $y=\pm\sqrt{2-(x+10)^2}$ 
\end{vastaus}
\end{tehtava}

\begin{tehtava}
Ympyrän keskipiste on origo ja säde $3$. Onko piste 
\begin{alakohdat}
	\alakohta{$(10,-2)$}
	\alakohta{$(-3,0)$}
	\alakohta{$(2,\sqrt{5})$ ympyrän kehällä?}
\end{alakohdat}
\begin{vastaus}
\begin{alakohdat}
	\alakohta{ei!}
	\alakohta{joo!}
	\alakohta{joo!}
\end{alakohdat}
\end{vastaus}
\end{tehtava}

\begin{tehtava}
Määritä $k$ niin, että lauseke $(x-3)^2+(y+3)^2=k$ on
\begin{alakohdat}
	\alakohta{ympyrä}
	\alakohta{$\sqrt{7}$-säteinen ympyrä}
	\alakohta{origon kautta kulkeva ympyrä?}
\end{alakohdat}
\begin{vastaus}
\begin{alakohdat}
	\alakohta{$k>0$}
	\alakohta{$k=7$}
	\alakohta{$k=18$}
\end{alakohdat}
\end{vastaus}
\end{tehtava}

\begin{tehtava}
Tutki, mitä yhtälöiden kuvaajat esittävät.
\begin{alakohdat}
	\alakohta{$x^2+y^2-6x+4y+4=0$}
	\alakohta{$x^2+y^2+14x-6y+10=0$}
\end{alakohdat}
\begin{vastaus}
\begin{alakohdat}
	\alakohta{ympyrä}
	\alakohta{piste}
\end{alakohdat}
\end{vastaus}
\end{tehtava}

\begin{tehtava}
Määritä ympyrän keskipiste ja säde.
\begin{alakohdat}
	\alakohta{$(x+t)^2+(y+u)^2=k, k>0$}
	\alakohta{$(x+2)^2+(y-7)^2=-8$}
\end{alakohdat}
\begin{vastaus}
\begin{alakohdat}
	\alakohta{keskipiste $(-t,-u)$, säde  $\sqrt{k}$}
	\alakohta{ei ole ympyrä}
\end{alakohdat}
\end{vastaus}
\end{tehtava}

\begin{tehtava}
Ympyrä sivuaa $y$-akselia pisteessä $(0,-1)$ ja kulkee pisteen $(3,2)$ kautta. Mikä on ympyrän yhtälö?
\begin{vastaus}
$(x-3)^2+(y+1)^2=9$
\end{vastaus}
\end{tehtava}

\begin{tehtava}
Ympyrä kulkee pisteiden $(1,6), (-2,5)$ ja $(5,4)$ kautta. Mikä on ympyrän yhtälö?
\begin{vastaus}
$(x-1)^2+(y-1)^2=16$
\end{vastaus}
\end{tehtava}

\begin{tehtava}
Jana, jonka pituus on $t$ liikkuu koordinaatistossa siten, että sen toinen pää on $x$-akselilla ja toinen $y$-akselilla. Mitä käyrää pitkin liikkuu janan keskipiste?
\begin{vastaus}
$x^2+y^2=\frac{1}{4}t^2$
\end{vastaus}
\end{tehtava}

\end{tehtavasivu}