\section{Lineaariset yhtälöryhmät} % FIXME: siirrä?

\laatikko{
KIRJOITA TÄHÄN LUKUUN

\begin{itemize}
\item mikä yhtälöryhmä on
\item miten ratkaistaan yhtälöpari (sijoitus, yhteenlaskumenetelmä)
\item että ratkaisuja voi olla yksi, nolla tai äärettömän monta
\item miten useamman tuntemattoman yhtälöryhmä ratkaistaan
\end{itemize}

KIITOS!}

\subsection*{Lineaariset yhtälöparit}

Yksinkertaisin tapaus yhtälöryhmistä on lineaarinen yhtälöpari, jossa on
kaksi ensimmäisen asteen yhtälöä. Käyttäen normaalimuotoa:

\laatikko{
Lineaarisen yhtälöparin yleinen muoto:

\begin{align*}
a_1x+b_1y+c_1 &= 0 \\
a_2x+b_2x+c_2 &= 0
\end{align*}

$a_1, a_2, b_1, b_2, c_1, c_2 \in \mathbb{R}$

Lineaarisen yhtälöparin ratkaisu on pari $(x, y) \in \mathbb{R}^2$, joka toteuttaa molemmat yhtälöt.
}

Aiemmin on todettu, että normaalimuotoinen ensimmäisen asteen yhtälö voidaan tulkita suorana
$(x, y)$-tasossa. Näin ollen lineaariselle yhtälöparille on geometrinen tulkinta: sen
ratkaisut ovat ne tason pisteet, joissa yhtälöitä vastaavat
suorat leikkaavat. Näitä voi olla $0$ (suorat ovat yhdensuuntaiset,
mutta eivät sama suora), $1$ (suorat eivät ole yhdensuuntaiset) tai äärettömän monta (suorat ovat sama suora).

\laatikko{
Lineaarisella yhtälöparilla voi olla joko $0$, $1$ tai äärettömän monta ratkaisua.
}

\subsection*{Lineaariset yhtälöryhmät}

Lineaarisen yhtälöparin ajatus yleistyy mihin tahansa määrään ensimmäisen asteen yhtälöitä.
Tällöin puhutaan lineaarisista yhtälöryhmistä.\footnote{Myös lineaarinen yhtälöpari on lineaarinen yhtälöryhmä.}
Myös geometrinen tulkinta yleistyy, mutta se on vaikeampi visualisoida paperille:
lineaarinen kolmen yhtälön ryhmä vastaa kolmen $(x, y)$-tason leikkauspisteitä
kolmiulotteisessa $(x, y, z)$-avaruudessa jne.

Tällä kurssilla tarkastellaan lähinnä kolmen yhtälön lineaarisia yhtälöryhmiä. Neljän yhtälön
lineaarisista yhtälöryhmistä esitetään joitakin helppoja esimerkkejä. Yleisesti ottaen yhtälöryhmiä
ei ratkaista tällä kurssilla esitetyin keinoin, vaan likimääräisesti tietokoneella käyttäen numeerista 
matriisilaskentaa (ei käsitellä lukiossa).

\subsection*{Tehtäviä}

\begin{tehtava}
    Ratkaise yhtälöryhmä.
    \begin{align*}
        2x+3y+5z+8 &= 0 \\
        3x+5y+8z &=0 \\
        x+y-1 &=0
    \end{align*}
    \begin{vastaus}
        $x = -\frac{63}{2}, \, y = \frac{65}{2}, \, z = -\frac{17}{2}$
    \end{vastaus}
\end{tehtava}

\begin{tehtava}
    Ratkaise yhtälöryhmä.
    \begin{align*}
        x+y+2z+12 &= 0 \\
        2x+2y+3z+1 &=0 \\
        3x-4 &=0
    \end{align*}
    \begin{vastaus}
        $x = \frac{4}{3}, \, y = \frac{98}{3}, \, z = -23$
    \end{vastaus}
\end{tehtava}
