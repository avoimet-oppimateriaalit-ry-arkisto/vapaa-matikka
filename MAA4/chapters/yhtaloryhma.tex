\section{Lineaariset yhtälöryhmät} % FIXME: siirrä?

\laatikko{
KIRJOITA TÄHÄN LUKUUN

\begin{itemize}
\item mikä yhtälöryhmä on
\item miten ratkaistaan yhtälöpari (sijoitus, yhteenlaskumenetelmä)
\item että ratkaisuja voi olla yksi, nolla tai äärettömän monta
\item miten useamman tuntemattoman yhtälöryhmä ratkaistaan
\end{itemize}

KIITOS!}

%\subsection*{Yhtälöryhmä}

% \termi{yhtälöryhmä}{Yhtälöryhmällä} tarkoitetaan useaa yhtälöä, joiden täytyy
% päteä samanaikaisesti. Englannin kielessä yhtälöryhmä tunnetaankin nimellä
% \textit{simultaneous equations} eli samanaikaiset yhtälöt.
% \footnote{Termi \textit{system of equations} on vähintäänkin yhtä yleinen.}
% Yhtälöryhmän (mahdollisten) ratkaisujen tulee siis toteuttaa kaikki yhtälöt.
% 
% Tässä luvussa käsitellään yhtälöryhmiä, joissa kaikki yhtälöt ovat ensimmäistä
% astetta. Tällaisia yhtälöryhmiä kutsutaan \termi{lineaarinen yhtälöryhmä}{lineaarisiksi yhtälöryhmiksi}.
% Lisäksi myöhemmin kirjassa käsitellään erikoistapauksia toisen asteen yhtälöryhmistä
% ratkaistaessa kahden ympyrän tai ympyrän ja suoran leikkauspisteitä.

\subsection*{Yhtälöpari}

% Yksinkertaisin mielenkiintoinen yhtälöryhmä on
% \termi{lineaarinen yhtälöpari}{lineaarinen yhtälöpari}.
% Lineaarisessa yhtälöparissa on kaksi ensimmäisen asteen yhtälöä.
% Lineaarinen yhtälöpari voidaan esittää monella tapaa. Tässä
% kirjassa käytämme pääasiallisesti ns. normaalimuotoa.
% 
% \[
% \left\{
% \begin{aligned}
% a_1x+b_1y+c_1 &= 0 \\
% a_2x+b_2y+c_2 &= 0
% \end{aligned}
% \right.
% \]
% 
% $a_1, a_2, b_1, b_2, c_1, c_2 \in \R$
% 
% Lineaarisen yhtälöparin ratkaisu on pari $(x, y) \in \R^2$, joka toteuttaa molemmat yhtälöt.

\begin{esimerkki}
Ratkaise yhtälöpari
\[
\left\{
\begin{aligned}
3x-2y&= 1 \\
-x+5y&= 2.
\end{aligned}
\right.
\]
\begin{esimratk}
On löydettävä kaikki ne luvut $x$ ja $y$ jotka toteuttavat molemmat yhtälöt.

Käytetään \termi{sijoitusmenetelmä}{sijoitusmenetelmää}. Ratkaistaan tuntematon $x$ alemmasta yhtälöstä ja sijoitetaan se ylempään yhtälöön. Alemmasta yhtälöstä $-x+5y= 2$ saadaan $x=5y-2$. Sijoitetaan tämä ylempään yhtälöön $3x-2y=1$:
\begin{align*}
3x-2y&=1 && \ppalkki \text{Sijoitetaan $x=5y-2$.} \\
3(5y-2)-2y&=1 && \\
15y-6-2y&=1 && \\
13y&=7 && \\
y&=\frac{13}{7} && \\
\end{align*}
Sijoitetaan $y=13/7$ alempaan yhtälöön, jotta voidaan ratkaista $x$:
\begin{align*}
-x+5y&= 2 && \ppalkki \text{Sijoitetaan $y=\frac{13}{7}$.} \\
-x+\frac{65}{7}&= 2 && \\
-x&= -\frac{51}{7}&& \\
x&= \frac{51}{7}&&
\end{align*}
\end{esimratk}
\begin{esimvast}
Yhtälön ratkaisu on $x= 51/7$, $y=13/7$.
\end{esimvast}
\end{esimerkki}

\begin{esimerkki}
Määritä suorien $-2x-y= 4$ ja $3x-2y=1$ leikkauspiste.
\begin{esimratk}
Saadaan ratkaistavaksi yhtälöpari
\[
\left\{
\begin{aligned}
-2x-y&= 4 \\
3x-2y&= 1.
\end{aligned}
\right.
\]
Käytetään \termi{yhteenlaskumenetelmä}{yhteenlaskumenetelmää}, jolla voidaan eliminoida toinen tuntemattomista.
\begin{align*}
&\left\{
\begin{aligned}
2x-y&= 4 && \ppalkki \cdot 3\\
3x-2y&= 1 && \ppalkki \cdot (-2)
\end{aligned}
\right. \\
&\left\{
\begin{aligned}
6x-3y&= 12 && \ppalkki \text{Lasketaan yhtälöt yhteen.}\\
-6x+4y&= -2.&&
\end{aligned}
\right.\\
&y= 10 \\
\end{align*}
Sijoitetaan $y=10$ jompaan kumpaan alkuperäisistä yhtälöistä, jotta saadaan ratkaistua $x$. Käytetään vaikkapa yhtälöä $3x-2y=1$:
\begin{align*}
3x-2y&=1 && \ppalkki \text{Sijoitetaan $x=10$.} \\
3x-20&=1 && \\
3x&=21 && \\
x&=7 &&.
\end{align*}
\end{esimratk}
\begin{esimvast}
Yhtälön ratkaisu on $x=7$, $y=10$.
\end{esimvast}
\end{esimerkki}

EDELLISEN ESIMERKIN ULKOASUA PITÄÄ VIELÄ MUOKATA. TASAUKSET EIVÄT TOIMI JA LISÄKSI YHTÄLÖPARIN ALLE PITÄISI SUMMAUKSEN KOHDALLA SAADA VIIVA. SAMA ONGELMA MYÖHEMMISSÄ ESIMERKEISSÄ.

On mahdollista, että yhtälöryhmällä ei ole lainkaan ratkaisuja.

\begin{esimerkki}
Ratkaise yhtälöpari
\[
\left\{
\begin{aligned}
x-2y&= 1 \\
5x-10y&= 2.
\end{aligned}
\right.
\]
\begin{esimratk}
Käytetään yhteenlaskumenetelmää:
\begin{align*}
&\left\{
\begin{aligned}
-x+2y&= 1 && \ppalkki \cdot 5 \\
5x-10y&= 2. &&
\end{aligned}
\right. \\
&\left\{
\begin{aligned}
-5x-2y&= 5 && \ppalkki \cdot \text{Lasketaan yhtälöt yhteen.} \\
5x-10y&= 2. &&
\end{aligned}
\right. \\
&0=7 \\
\end{align*}
Koska päädytään mahdottomaan yhtälöön, yhtälöparilla ei ole ratkaisua.
\end{esimratk}
\begin{esimvast}
Yhtälöparilla ei ole ratkaisua.
\end{esimvast}
\end{esimerkki}

Edellisessä esimerkissä yhtälöryhmällä ei ollut lainkaan ratkaisuja. Geometrisesti tämä tarkoittaa sitä, että suorilla $x-2y= 1$ ja $5x-10y= 2$ ei ole leikkauspisteitä. Ne ovat siis yhdensuuntaiset.

TÄHÄN KUVA?

Joillakin yhtälöpareilla on äärettömän monta ratkaisua.

\begin{esimerkki}
Ratkaise yhtälöpari
\[
\left\{
\begin{aligned}
-3x+y&= -1 \\
6x-2y&= 2.
\end{aligned}
\right.
\]
\begin{esimratk}
Käytetään sijoitusmenetelmää. Ensimmäisestä yhtälöstä saadaan $y=3x-1$. Sijoitetaan tämä toiseen yhtälöön:
\begin{align*}
6x-2y&= 2 && \ppalkki \text{sijoitetaan $y=3x-1$} \\
6x-2(3x-1)&= 2 && \\
6x-6x+2&= 2 && \\
2&= 2. &&
\end{align*}
\end{esimratk}
Näin saadusta yhtälöstä $2=2$ ei voikaan ratkaista tuntematonta $x$ niin kuin oli tarkoitus. Yhtälö ei anna mitään lisätietoa tuntemattomista. Tämä johtuu siitä, että alkuperäiset yhtälöt $-3x+y= -1$ ja $6x-2y= 2$ ovat yhtäpitäviä. Ensimmäisestä saadaan jälkimmäinen kahdella kertomalla.

Oleellisesti tarkasteltavana onkin vain yksi yhtälö, $-3x+y= -1$. Yhtälöparin ratkaisuja ovat kaikki ne luvut $x$ ja $y$, jotka toteuttavat tämän yhtälön. Ratkaisuja ovat esimerkiksi $x=0$, $y=-1$ ja $x=1$, $y=2$. Ratkaisuja on äärettömän monta.
\begin{esimvast}
Ratkaisuja on äärettömän monta.
\end{esimvast}
\end{esimerkki}

Edellisessä esimerkissä yhtälöparilla oli äärettömän monta ratkaisua. Geometrisesti tämä tarkoittaa sitä, että yhtälöt $-3x+y= -1$ ja $6x-2y= 2$ ovat saman suoran yhtälö. Siten kaikki suoralla $-3x+y= -1$ (tai yhtä hyvin suoralla $6x-2y= 2$) olevat pisteet toteuttavat yhtälöparin.

% Aiemmin on todettu, että normaalimuotoinen ensimmäisen asteen yhtälö voidaan tulkita suorana
% $xy$-tasossa. Näin ollen lineaariselle yhtälöparille on geometrinen tulkinta: sen
% ratkaisut ovat ne tason pisteet, joissa yhtälöitä vastaavat
% suorat leikkaavat. Näitä voi olla
% $0$ (suorat ovat yhdensuuntaiset, mutta eivät sama suora),
% $1$ (suorat eivät ole yhdensuuntaiset) tai
% äärettömän monta (suorat ovat sama suora).
% 
% Lineaarisia yhtälöpareja ratkotaan pääasiallisesti kahdella menetelmällä.
% Nämä menetelmät ovat \termi{sijoitusmenetelmä}{sijoitusmenetelmä} ja
% \termi{yhteenlaskumenetelmä}{yhteenlaskumenetelmä}.


\subsection*{Lineaarinen yhtälöryhmä}

Ratkaistavia yhtälöitä voi olla useampia kuin kaksi, ja tällöin puhutaan \termi{yhtälöryhmä}{yhtälöryhmästä}.

Jos yhtälöt ovat lisäksi ensimmäistä astetta, on kyseessä \termi{lineaarinen yhtälöryhmä}{lineaarinen yhtälöryhmä}.
Esimerkiksi
\[
\left\{
\begin{aligned}
3x-2y-z&= -5 \\
5x+6y+5z&= 1 \\
-x+5y&= 0.
\end{aligned}
\right.
\]
on lineaarinen yhtälöryhmä. Yhtälöpari on yhtälöryhmän erikoistapaus.

% Tässä tarkastellaan lähinnä kolmen yhtälön lineaarisia yhtälöryhmiä. Neljän yhtälön
% lineaarisista yhtälöryhmistä esitetään joitakin helppoja esimerkkejä. Yleisesti ottaen yhtälöryhmiä
% ei ratkaista tällä kurssilla esitetyin keinoin, vaan likimääräisesti tietokoneella käyttäen numeerista 
% matriisilaskentaa, joka ei kuulu lukion oppimäärään.


\begin{esimerkki}
Ratkaise lineaarinen yhtälöryhmä
\[
\left\{
\begin{aligned}
4x+2y+5z&=0 \\
6x-3y-z&= 23 \\
x-2y+3z&= -1.
\end{aligned}
\right.
\]
\begin{esimratk}
Käytetään yhteenlaskumenetelmää kahteen ylimpään yhtälöön ja eliminoidaan niistä tuntematon $x$:
\begin{align*}
&\left\{
\begin{aligned}
4x+2y+5z&=0 && \ppalkki \cdot 3 \\
6x-3y-z&= 23 && \ppalkki \cdot (-2) \\
\end{aligned}
\right. \\
&\left\{
\begin{aligned}
12x+6y+15z&=0 \\
-12x+6y+2z&= -46 \\
\end{aligned}
\right. \\
&12y+17z=-46 \\
\end{align*}

Tehdään sitten sama toiselle ja kolmannelle yhtälölle:
\begin{align*}
&\left\{
\begin{aligned}
6x-3y-z&= 23 && \\
x-2y+3z&= -1 && \ppalkki \cdot (-6)
\end{aligned}
\right. \\
&\left\{
\begin{aligned}
6x-3y-z&= 23 \\
-6x+12y-18z&=6
\end{aligned}
\right. \\
&9y-19z=29. \\
\end{align*}

Nyt on saatu kaksi yhtälöä, joissa ei ole tuntematonta $x$. Ratkaistaan näin muodostuva yhtälöpari:
\begin{align*}
&\left\{
\begin{aligned}
12y+17z&=-46 && \ppalkki \cdot 3 \\
9y-19z&=29 && \ppalkki \cdot (-4) \\
\end{aligned}
\right. \\
&\left\{
\begin{aligned}
36y+51z&=-138  \\
-36y+76z&=-116  \\
\end{aligned}
\right. \\
&127z=-254 \\
&z=-2 \\
\end{align*}

Nyt tiedetään, että $z=-2$. Sijoitetaan tämä aiemmin saatuun kahden tuntemattoman yhtälöön $12y+17z=-46$:
\begin{align*}
12y+17z&=-46 && \ppalkki \text{Sijoitetaan $z=-2$.} \\
12y-34&=-46 && \\
12y&=-12 && \\
y&=-1 && \\
\end{align*}

Lopuksi sijoitetaan $y=-1$ ja $z=-2$ johonkin alkuperäisistä yhtälöistä, vaikkapa ensimmäiseen yhtälöön:
\begin{align*}
4x+2y+5z&=0 && \ppalkki \text{Sijoitetaan $y=-1$ ja $z=-2$.} \\
4x-2-10&=0 && \\
4x&=12 && \\
x&=3 && \\
\end{align*}

Nyt on siis saatu ratkaisu $x=3$, $y=-1$, $z=-2$. Tarkistetaan vielä, että nämä luvut tosiaankin toteuttavat kaikki alkuperäisen yhtälöryhmän
\[
\left\{
\begin{aligned}
4x+2y+5z&=0 \\
6x-3y-z&= 23 \\
x-2y+3z&= -1.
\end{aligned}
\right.
\]
yhtälöt. Tämä tehdään sijoittamalla luvut yhtälöiden vasemmalle puolella ja tarkistamalla, että tulos täsmää yhtälön oikean puolen kanssa.

Koska $4 \cdot 3+2\cdot(-1)+5\cdot(-2)=12-2-10=0$, ne toteuttavat ensimmäisen yhtälöistä. Samalla tavalla $6 \cdot 3-3\cdot(-1)-\cdot(-2)=18+3+2=23$, joten luvut toteuttavat myös toisen yhtälön. Lopuksi todetaan, että $3-2\cdot(-1)+3\cdot(-2)=3+2-6=0-1$, ja siten kolmaskin yhtälöistä toteutuu.

\end{esimratk}
\begin{esimvast}
Yhtälöryhmän ratkaisu on $x=3$, $y=-1$, $z=-2$.
\end{esimvast}
\end{esimerkki}

\begin{tehtavasivu}

\subsubsection*{Opi perusteet}

\subsubsection*{Hallitse kokonaisuus}

\subsubsection*{Sekalaisia tehtäviä}

TÄHÄN TEHTÄVIÄ SIJOITTAMISTA ODOTTAMAAN

\begin{tehtava}
    Ratkaise yhtälöpari.
    \begin{align*}
        x+y+1 &= 0 \\
        x+2y+1 &=0
    \end{align*}
    \begin{vastaus}
        $x = -1, \, y = 0$
    \end{vastaus}
\end{tehtava}

\begin{tehtava}
    Ratkaise yhtälöpari.
    \begin{align*}
        2x+5y+1 &= 0 \\
        2x+2y+7 &=0
    \end{align*}
    \begin{vastaus}
        $x = -\frac{11}{2}, \, y = 2$
    \end{vastaus}
\end{tehtava}

\begin{tehtava}
    Ratkaise yhtälöpari. $t \in \R$ on vapaa parametri, joka saa sisältyä vastaukseen.
    \begin{align*}
        x+2y-t-1 &= 0 \\
        x+y+t^2 &=0
    \end{align*}
    \begin{vastaus}
        $x = -2t^2-t-1, \, y = t^2+t+1$
    \end{vastaus}
\end{tehtava}

\begin{tehtava}
    Ratkaise yhtälöryhmä.
	$$\left\{    
    \begin{array}{rcl}
        x+2y+1 &=&0 \\
        x+2z+3 &=&0 \\
        y+2z+5 &=&0
    \end{array}
    \right.$$
    \begin{vastaus}
        $x = 1, \, y = -1, \, z = -2$
    \end{vastaus}
\end{tehtava}

\begin{tehtava}
    Ratkaise yhtälöryhmä.
    \begin{align*}
        x+y+z+8 &= 0 \\
        x+y+6 &=0 \\
        x+z-70 &=0
    \end{align*}
    \begin{vastaus}
        $x = 72, \, y = -78, \, z = -2$
    \end{vastaus}
\end{tehtava}

\begin{tehtava}
    Ratkaise yhtälöryhmä.
    \begin{align*}
        x+y+2z+12 &= 0 \\
        2x+2y+3z+1 &=0 \\
        3x-4 &=0
    \end{align*}
    \begin{vastaus}
        $x = \frac{4}{3}, \, y = \frac{98}{3}, \, z = -23$
    \end{vastaus}
\end{tehtava}

\begin{tehtava}
    Ratkaise yhtälöryhmä.
    \begin{align*}
        2x+3y+5z+8 &= 0 \\
        3x+5y+8z &=0 \\
        x+y-1 &=0
    \end{align*}
    \begin{vastaus}
        $x = -\frac{63}{2}, \, y = \frac{65}{2}, \, z = -\frac{17}{2}$
    \end{vastaus}
\end{tehtava}

\end{tehtavasivu}
