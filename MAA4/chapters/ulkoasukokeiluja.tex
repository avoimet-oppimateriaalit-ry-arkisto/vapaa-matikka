 
\section{Ulkoasukokeiluja}

%\definecolor{Sampo3}{cmyk}{0.2,0.27,0.0,0.0}
\definecolor{ParisGreen}{RGB}{80,200,120}
%\definecolor{PersianGreen}{RGB}{0,166,147}
%\definecolor{darkKhaki}{RGB}{189,183,107}
%\definecolor{cornflowerBlue}{RGB}{154,206,235}
%\definecolor{thulianPink}{RGB}{222,111,161}
%\definecolor{olivine}{RGB}{154,185,115}
%\definecolor{battleshipGray}{RGB}{132,132,130}

\pgfdeclarehorizontalshading{laatikkotaustaC}{100bp}{color(0bp)=(ParisGreen); color(50bp)=(ParisGreen); color(100bp)=(ParisGreen!50)}
%\pgfdeclarehorizontalshading{laatikkotausta}{100bp}{color(0bp)=(white); color(50bp)=(Sampo3); color(100bp)=(Sampo3!50)}
\pgfdeclarehorizontalshading{laatikkotaustaE}{195mm}{color(0mm)=(ParisGreen); color(97mm)=(ParisGreen); color(195mm)=(Sampo3)}
\pgfdeclarehorizontalshading{laatikkotaustaF}{100bp}{color(0bp)=(white); color(25bp)=(white); color(75bp)=(Sampo3); color(100bp)=(Sampo3)}
\colorlet{mycolor}{green}
\pgfdeclarehorizontalshading[mycolor]{myshadingB}{1cm}{rgb(0cm)=(1,0,0); color(1cm)=(white); color(2cm)=(Sampo3)}

\mdfdefinestyle{laatikkotyyliQ}{
  usetwoside=false,
  leftmargin=-46mm,
  rightmargin=-22mm,
  innerleftmargin=0mm,
  innerrightmargin=0mm,
  innertopmargin=0mm,
  innerbottommargin=0mm,
}

\mdfdefinestyle{laatikkotyyliB}{
%  userdefinedwidth=166mm % FIXME: controversial
  usetwoside=true,
  innermargin=15mm,
  outermargin=0mm,
  innerleftmargin=36mm,
  innerrightmargin=12mm,
  middlelinewidth=0pt,
  apptotikzsetting={\tikzset{mdfbackground/.append style = {shading = laatikkotausta}}},
}

\mdfdefinestyle{laatikkotyyliC}{
  userdefinedwidth=127mm,
  leftmargin=0pt,
  innermargin=0pt,
  innerleftmargin=0,
  rightmargin=0pt,
  innerrightmargin=0pt,
  middlelinewidth=0pt,
  apptotikzsetting={\tikzset{mdfbackground/.append style = {shading = laatikkotaustaC}}},
  innertopmargin=0mm,
  innerbottommargin=0mm,
  needspace=3\baselineskip,
}

\mdfdefinestyle{laatikkotyyliW}{
  usetwoside=false,
  leftmargin=-46mm,
  rightmargin=-22mm,
  innerleftmargin=46mm,
  innerrightmargin=22mm,
  innertopmargin=8pt,
  innerbottommargin=8pt,
  hidealllines=true,
  apptotikzsetting={\tikzset{mdfbackground/.append style = {shading = laatikkotaustaF}}},
  needspace=3\baselineskip,
}

\newcommand{\laatikkoB}[1]{%
%\begin{\widepar}
\ifthispageodd
{
  \begin{mdframed}[style=laatikkotyyliB]
    #1
  \end{mdframed}
}
{
  \begin{mdframed}[style=laatikkotyyliC]
    #1
  \end{mdframed}
}
%\end{widepar}
}


\newcommand{\laatikkoE}[1]{%
%\begin{\widepar}
  \begin{mdframed}[style=laatikkotyyliW]
	#1
  \end{mdframed}
%\end{widepar}
}

\laatikkoE{
\textbf{Itseisarvon ominaisuuksia}\\ \\
	\begin{tabular}{l l}
		$|a|\geq0$ & Itseisarvo on aina ei-negatiivinen \\
		$|a|=|-a|$ & Luvun ja sen vastaluvun itseisarvot ovat yhtäsuuret \\
		$|a|^2=a^2$ & Luvun itseisarvon neliö on yhtäsuuri kuin luvun neliö \\
		$|\frac{a}{b}|=\frac{|a|}{|b|}$ & Osamaarän itseisarvo on itseisarvojen osamäärä

	\end{tabular}
}

\newpage

\section{Ulkoasukokeiluja 2}

\laatikkoE{
\textbf{Itseisarvon ominaisuuksia}\\ \\
	\begin{tabular}{l l}
		$|a|\geq0$ & Itseisarvo on aina ei-negatiivinen \\
		$|a|=|-a|$ & Luvun ja sen vastaluvun itseisarvot ovat yhtäsuuret \\
		$|a|^2=a^2$ & Luvun itseisarvon neliö on yhtäsuuri kuin luvun neliö \\
		$|\frac{a}{b}|=\frac{|a|}{|b|}$ & Osamaarän itseisarvo on itseisarvojen osamäärä

	\end{tabular}
}

\newpage
