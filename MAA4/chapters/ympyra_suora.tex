\section{Ympyrä ja suora}

\laatikko{
KIRJOITA TÄHÄN LUKUUN

\begin{itemize}
\item ympyrän ja suoran leikkauspisteiden ratkaiseminen
\item ympyrän tangetin määrittäminen sekä kehällä olevan (kohtisuorassa sädettä vastaan) että
sen ulkopuolisen pisteen kautta (kaksi tapaa: pisteen etäisyys suorasta -kaava tai diskriminantti = 0)
\item kahden ympyrän leikkauspisteiden ratkaiseminen yhtälöparilla
\end{itemize}

KIITOS!}

\begin{esimerkki}
Määritä suorien $x+2y-3=0$ ja ympyrän $(x-1)^2+(y+1)^2=4 $ leikkauspisteet.

\begin{esimratk}

Ympyrän ja suoran leikkauspisteet ovat ne pisteet $(x,y)$, jotka ovat sekä suoralla että ympyrällä, eli toteuttavat molempien yhtälöt, eli yhtälöryhmän
$$\left\{    
    \begin{array}{rcl}
        x+2y-3 &=&0 \\
        (x-1)^2+(y+1)^2 &=&4 \\
    \end{array}
    \right.$$
Vaikka yhtälö ei olekaan tutun lineaarinen, myös sitä voi lähestyä sijoitusmenetelmällä. Ratkaistaan ensimmäisestä yhtälöstä $x$ ja saadaan
\[
x = -2y+3
\]
Kun tämä sijoitetaan toiseen yhtälöön ja kerrotaan auki syntyy toisen asteen yhtälö $y$:n suhteen:
\begin{align*}
(-2y+3-1)^2+(y+1)^2=4 \\
(-2y+2)^2+(y+1)^2=4 \\
(-2y)^2-2\cdot 2y\cdot 2 +2^2+y^2+2\cdot y+1^2=4 \\
4y^2-8y+4+y^2+2y+1-4 = 0 \\
5y^2-6y+1 = 0
\end{align*}
Ratkaisukaavalla
\[
y = \frac{-(-6)\pm\sqrt{(-6)^2-4\cdot 5\cdot 1}}{2\cdot5}
\]
eli
\begin{align*}
y = \frac{6\pm\sqrt{16}}{10} \\
y = 1 \vee y = \frac{1}{5}
\end{align*}
Kun nämä $y$:n arvot sijoitetaan suoran yhtälöön saadaan vastaavast $x$:n arvot:
\begin{align*}
x = -2\cdot 1+3 \vee x = -2\cdot\frac{1}{5}+3 \\
x = 1 \vee x = 2\frac{3}{5}
\end{align*}
eli saatiin 2 ratkaisua: $(x,y) = (1,1)$ ja $(x,y) = (2\frac{3}{5},\frac{1}{5})$. Tarkistamalla on hyvä vielä todeta, että pisteet todella ovat leikkauspisteitä.

Tähän kuva tilanteesta.
\begin{esimvast}
Leikkauspisteet ovat $(1,1)$ ja $(2\frac{3}{5},\frac{1}{5})$.
\end{esimvast}

\end{esimratk}
\end{esimerkki}

\begin{tehtavasivu}

\subsubsection*{Opi perusteet}

\subsubsection*{Hallitse kokonaisuus}

\subsubsection*{Sekalaisia tehtäviä}

TÄHÄN TEHTÄVIÄ SIJOITTAMISTA ODOTTAMAAN

\end{tehtavasivu}