\section{Korkeamman asteen epäyhtälöt}

\qrlinkki{http://opetus.tv/maa/maa2/n-asteinen-polynomiepayhtalo/}{Opetus.tv: \emph{N-asteinen polynomiepäyhtälö} (16:19 ja 12:00)}

Korkeamman asteen epäyhtälö, kuten epäyhtälö
\[ x^3 -6x \leq x^2 \]
ratkaistaan siirtämällä kaikki termit epäyhtälön toiselle puolelle ja tutkimalla syntyvän polynomin merkkiä:
\begin{align*}
    x^3 -6x &\leq x^2 &\ppalkki -x^2 \\
    \underbrace{x^3-x^2 -6x}_{P(x)} &\leq 0.
\end{align*}
Polynomin $P(x)$ merkin selvittämiseksi ratkaistaan sen nollakohdat:
\begin{align*}
    x^3 - x^2-6x &= 0 &\ppalkki \text{yhteinen tekijä} \\
    x(x^2 -x -6) &= 0 &\ppalkki \text{tulon nollasääntö} \\
    x = 0 \tai x^2 -x -6 &= 0 &\ppalkki \text{ratkaisukaava} \\
    x &=\frac{-(-1) \pm \sqrt{(-1)^2-4\cdot 1 \cdot (-6)}}{2\cdot 1} \\
    x &= -2 \tai x = 3.
\end{align*}
Polynomin $P$ nollakohdat ovat siis $0$, $-2$ ja $3$. Tästä voidaan jatkaa kahdella eri tavalla.

\paragraph{Tapa 1.} Tekijöihin jako

Jaetaan polynomi tekijöihin nollakohtien avulla:
\[ P(x) = x^3 - x^2-6x = x(x^2-x-6) = x(x+2)(x-3). \]
Tutkitaan kunkin tulon tekijän merkkiä:

\begin{align*}
    x+2>0 &\kun x > -2\\
    x-3>0 &\kun x > 3\\
    x>0 &\kun x > 0.
\end{align*}

Kootaan tulokset merkkikaavioon, jonka jokainen ruutu vastaa yhtä reaalilukuväliä lukusuoralla:

\begin{center}
    \begin{merkkikaavio}{3}
        \merkkikaavioKohta{$-2$}
        \merkkikaavioKohta{$0$}
        \merkkikaavioKohta{$3$}

        \merkkikaavioFunktio{$x+2$}
        \merkkikaavioMerkki{$-$}
        \merkkikaavioMerkki{$+$}
        \merkkikaavioMerkki{$+$}
        \merkkikaavioMerkki{$+$}

        \merkkikaavioUusirivi
        \merkkikaavioFunktio{$x-3$}
        \merkkikaavioMerkki{$-$}
        \merkkikaavioMerkki{$-$}
        \merkkikaavioMerkki{$-$}
        \merkkikaavioMerkki{$+$}

        \merkkikaavioUusirivi
        \merkkikaavioFunktio{$x$}
        \merkkikaavioMerkki{$-$}
        \merkkikaavioMerkki{$-$}
        \merkkikaavioMerkki{$+$}
        \merkkikaavioMerkki{$+$}

        \merkkikaavioUusiriviKaksoisviiva
        \merkkikaavioFunktio{$x(x+2)(x-3)$}
        \merkkikaavioMerkki{$-$}
        \merkkikaavioMerkki{$+$}
        \merkkikaavioMerkki{$-$}
        \merkkikaavioMerkki{$+$}
    \end{merkkikaavio}
\end{center}

Merkkikaaviosta voidaan lukea vastaus: $x^3-x^2-6 \leq 0$, kun $x\leq -2$ tai $0\leq x \leq 3$.

\paragraph{Tapa 2.} Testipisteet

Polynomit ovat jatkuvia funktioita. Jatkuvuudesta puhutaan vasta kurssilla MAA7.
Intuitiivisesti jatkuvuudessa on kyse siitä, että funktion kuvaaja on yhtenäinen viiva.
Jatkuvuudesta seuraa, että polynomi ei voi vaihtaa merkkiä kulkematta nollakohdan kautta.
Polynomin merkin nollakohtien välillä saa selville tarkistamalla väliltä otetun testipisteen merkin.

Esimerkissä nollakohdat olivat $-2$, $0$ ja $3$. Valitaan niiden välistä ja
ympäriltä testipisteiksi vaikkapa $x=-3$, $x=-1$, $x=1$ ja $x=4$. Tarkistetaan funktion merkki:

\begin{tabular}{c|l|r|c}
Väli & Testipiste & $f(x)=x^3-x^2-6x$ & Funktion merkki \\
\hline
\ \ \ \ \  $x < -2$ & $x = -3$ & $(-3)^3 -(-3)^2 - 6(-3) = -18$ & $-$ \\
$-2 <x < 0$ & $x = -1$ & $(-1)^3 -(-1)^2 - 6(-1) =4$ & $+$ \\
$0 <x < 3$ & $x = 1$ & $1^3 -1^2 - 6\cdot 1 =  -6$ & $-$ \\
$3 <x $ \ \ \ \ \ & $x = 4$ & $4^3 -4^2 - 6\cdot 4 = 24$ & $+$
\end{tabular}

Vastaukseksi saadaan taas $x\leq -2$ tai $0\leq x \leq 3$.

\begin{tehtavasivu}

\paragraph*{Opi perusteet}

\begin{tehtava}
    Ratkaise
    \begin{alakohdat}
        \alakohta{$(x-1)(x-2)(x-3) \le 0$}
        \alakohta{$(x-1)(x-2)(x-3) > 0$}
        \alakohta{$3(x-1)(x-2)(x-3) > 0$.}
    \end{alakohdat}
    \begin{vastaus}
        \begin{alakohdat}
            \alakohta{$x \le 1$ tai $2 \le x \le 3$}
            \alakohta{$1 < x < 2$ tai $x>3$}
            \alakohta{$1 < x < 2$ tai $x>3$}
        \end{alakohdat}
    \end{vastaus}
\end{tehtava}

\begin{tehtava}
    Ratkaise $x^3-x^2<0$.
    \begin{vastaus}
        $x<0$ tai $0<x<1$
    \end{vastaus}
\end{tehtava}

\begin{tehtava}
    Ratkaise $x^4 \le 1$.
    \begin{vastaus}
        $-1 \le x \le 1$
    \end{vastaus}
\end{tehtava}

\paragraph*{Hallitse kokonaisuus}

%neliö epänegatiivinen
\begin{tehtava}
    Ratkaise $(2x^3+4x^2-5x+7)^2 < 0$.
    \begin{vastaus}
        Ei ratkaisuja.
    \end{vastaus}
\end{tehtava}

%bikvadraattinen
\begin{tehtava}
    Ratkaise $x^4-3x^2-18 \le 0$.
    \begin{vastaus}
        $-\sqrt{3}\le x \le \sqrt{3}$
    \end{vastaus}
\end{tehtava}

% x yhteinen tekijä ja sij. y=x^5
\begin{tehtava}
    Ratkaise $x+2x^6+x^{11}<0$.
    \begin{vastaus}
        $x<-1$ tai $ -1<x<0$
    \end{vastaus}
\end{tehtava}

%yhteinen tekijä x^3, binomikaava käänteisesti
\begin{tehtava}
    Ratkaise $4 x^5+9 x^3 \le 12 x^4$.
    \begin{vastaus}
        $x\le0$ tai $x=\frac{2}{3}$
    \end{vastaus}
\end{tehtava}

\end{tehtavasivu}
