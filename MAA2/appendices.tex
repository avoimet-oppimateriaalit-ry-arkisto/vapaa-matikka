\liitetyyli

\section{Vastaukset}
\input{appendices/vastaukset}

\section{Harjoituskokeita}

\section{Harjoituskoe 1}

kesken...
% \begin{enumerate}
% \item Ratkaise epäyhtälö $1-\dfrac{1-x}{6}<x$.
% \item Ratkaise yhtälöt. \\ a) 4x^2-1=0 \\ b) x^3=-3x \\ c) 2y^2=y-8
% \item Millä parametrin $k$:n arvoilla yhtälöllä $kx^2-(k+1)x+1=0$ on kaksi erisuurta reaalijuurta?
% \end{enumerate}

\section{Harjoituskoe 2}


\section{Harjoituskoe 3}


\section{Harjoituskoe 4}



\section{Todistuksia}

\input{appendices/tulonolla_todistus}

\subsection*{Toisen asteen polynomin kuvaaja}

Tässä liitteessä perustellaan, miksi kaikkien toisen asteen polynomifunktioiden kuvaajat näyttävät samalta. Lisäksi tarkastellaan, mitkä tekijät vaikuttavat kuvaajien muotoon.

\underline{Funktio $P(x)=x^2$}

Aloitetaan funktiosta $P(x)=x^2$. Mitä tiedämme siitä piirtämättä kuvaajaa?
\begin{itemize}
\item Algebrallisen päättelyn avulla voimme todeta, että funktion pienin arvo on $P(0) = 0$, sillä tulon merkkisäännön perusteella lauseke $x^2$ ei voi saada negatiivisia arvoja – ei ole olemassa sellaista reaalilukua $x$, jonka neliö olisi negatiivinen. (Kompleksiluvuilla tällaista rajoitusta ei ole, tästä lisää liitteessä.) Kuvaajan alin kohta sijoittuu siis origoon ja kuvaajan kaikki muut pisteet sen yläpuolelle.
\item Kun $x > 0$, lauseke $x^2$  on sitä
suurempi, mitä suurempi $x$ on. Tästä tiedämme, että nollasta oikealle siirryttäessä funktion kuvaaja nousee.
\item Koska $(-x)^2 = x^2$, kuvaaja on symmetrinen $y$-akselin suhteen.
\end{itemize}

Näiden tietojen avulla voimme päätellä, että funktion kuvaaja koostuu kahdesta
identtisestä haarasta, jotka kohtaavat, kun $x=0$. Mitä kauempana $x$ on nollasta, sitä suurempia ovat funktion arvot. Tämän kaiken voi päätellä jo ennen kuvaajan piirtämistä.

Merkitsemällä koordinaatistoon yhtälön $y=x^2$ toteuttavia pisteitä, muodostuu lopulta funktion kuvaaja:
\begin{center}
\begin{kuvaajapohja}{2}{-2}{2}{-1}{4}
  \kuvaaja{x**2}{$f(x)=x^2$}{blue}
\end{kuvaajapohja}
\end{center}

% FIXME: ongelmallista: ''alaspäin aukeava'' vasta seuraavaksi
Kuvaajan muoto on geometriselta nimeltään \emph{paraabeli}. Paraabeleja esiintyy monessa yhteydessä: esimerksi polttopeilin ja radioteleskoopin pinta kaareutuu paraabelin muotoisena. Samoin ilmaan heitetyn kappaleen rata on likimain alaspäin aukeava paraabeli, kun ilmanvastus on pieni.

\underline{Funktio $P(x)=ax^2, \quad a\neq 0$}

Polynomien $P(x)=ax^2$ arvot ovat ovat lausekkeeseen $x^2$ nähden $a$-kertaisia. Muuten paraabelin symmetrisyys ja muut keskeiset ominaisuudet säilyvät.

\begin{itemize}
\item Kun $a > 0$, myös $ax^2\geq 0$, joten pienin arvo on yhä $0$.\\
Paraabeli on \termi{ylöspäin aukeava}{ylöspäin aukeava}.
\item Kun $a < 0$, tulon merkkisäännön nojalla $ax^2 \leq 0$. \\
 Nyt $P(0)=0$ onkin suurin arvo, ja funktion arvot ovat sitä pienempiä,
mitä enemmän $x$ poikkeaa nollasta. \\
Paraabeli on \termi{alaspäin aukeava}{alaspäin aukeava}.
\item Mitä enemmän $a$ poikkeaa nollasta, sitä nopeammin funktion arvot
muuttuvat $x$:n muuttuessa ja sitä kapeampi paraabeli on.
\end{itemize}

\begin{center}
$a>0$, paraabeli aukeaa ylöspäin:\\
\begin{tabular}{cc}
$f(x)=\frac{1}{2}x^2$& $f(x)=2x^2$ \\
\begin{kuvaajapohja}{1}{-2}{2}{-1}{4}
  \kuvaaja{0.5*x**2}{}{blue}
\end{kuvaajapohja} &
\begin{kuvaajapohja}{1}{-2}{2}{-1}{4}
  \kuvaaja{2*x**2}{}{blue}
\end{kuvaajapohja}
\end{tabular}

$a<0$, paraabeli aukeaa alaspäin:\\
\begin{tabular}{cc}
$f(x)=-\frac{1}{2}x^2$ & $f(x)=-2x^2$ \\
\begin{kuvaajapohja}{1}{-2}{2}{-4}{1}
  \kuvaaja{-0.5*x**2}{}{blue}
\end{kuvaajapohja} &
\begin{kuvaajapohja}{1}{-2}{2}{-4}{1}
  \kuvaaja{-2*x**2}{}{blue}
\end{kuvaajapohja}
\end{tabular}
\end{center}

\underline{Funktio $P(x)=ax^2+c$}

Lisäämällä vakiotermi $c$ saadaan $P(x)=ax^2+c$. Vakion lisääminen nostaa tai laskee funktion kuvaajaa (riippuen siitä, onko $c > 0$ tai $c<0$), joten kuvaajan muoto ei muutu.

\underline{Funktio $P(x)=ax^2+bx+c$}

Miksi sitten täydellisen toisen asteen polynomin $P(x)=ax^2+bx+c$ kuvaaja on myös paraabeli? Muokataan lauseketta, aloitetaan ottamalla yhteinen tekijä:
\begin{align*}
P(x) &=ax^2+bx+c \\
&= a\left(x^2 +\frac{b}{a}x\right) + c  \quad &
\text{ lavennetaan } \frac{b}{a} \text{ kahdella} \\
&= a\left(x^2 +2\cdot x \cdot \frac{b}{2a} \quad\right) + c  &
\text{ täydennetään neliöksi lisäämällä} \left( \frac{b}{2a} \right)^2 \\
&= a \Bigg( \underbrace{x^2 +2\cdot x \cdot \frac{b}{2a}+\left(\frac{b}{2a} \right)^2}_{\left( x+\frac{b}{2a} \right)^2}
- \left(\frac{b}{2a}\right)^2 \Bigg)  + c \\
&= a \left( \left( x + \frac{b}{2a} \right )^2-\frac{b^2}{4a^2} \right) + c &
\text{ kerrotaan sulut auki } \\
&= a \underbrace{\left(  x + \frac{b}{2a} \right)^2}_{\text{neliö}}-
\underbrace{\frac{b^2}{4a} + c}_{\text{vakio}}
\end{align*}

Tästä neliöksi täydennetyksi muodosta nähdään, että $P(x)$ on muotoa
$a\cdot$neliö + vakio. Kuvaaja on siis samanlainen kuin tapauksessa
$ax^2+c$, huippu on vain siirtynyt.
Koska neliö $\geq 0$, saadaan

\begin{itemize}
\item Kun $a>0$, kyseinen vakio on polynomin pienin arvo ja kuvaaja on
ylöspäin aukeava paraabeli.
\item Kun $a<0$, kyseinen vakio on suurin arvo ja kuvaaja on alaspäin
aukeava paraabeli.
\end{itemize}

Paraabelin \termi{huippu}{huippu} (eli kuvaajan piste, jossa suurin tai pienin arvo
saavutetaan) on aina kohdassa
$x=-\frac{b}{2a}$, koska silloin neliö on nolla.

%Toisen asteen polynomifunktio on muotoa $f(x)=ax^2+bx+c$, jossa $a,b,c \in \mathbb{R}$ ja $a \neq 0$. Toisen asteen polynomifunktion kuvaaja on paraabeli. Toisen asteen polynomifunktioita käytetään matemaattisessa mallinnuksessa talouden, tieteen ja tekniikan aloilla. Esimerkiksi heittoliikkeessä olevan kappaleen lentorata on aina paraabelin muotoinen. \\
%\textbf{Esimerkki 1.}
%a) Piirrä funktion $f(x)=x^2-2$ kuvaaja. \\
%b) Ratkaise funktion $f$ nollakohdat. \\ \\
%
%\begin{kuvaajapohja}{1}{-2}{2}{-3}{1}
%  \kuvaaja{x**2-2}{$f(x)=x^2-2$}{blue}
%\end{kuvaajapohja}
%
%Funtkion kuvaaja on ylöspäin aukeava paraabeli, joka leikkaa x-akselin kohdissa joissa $f(x)=0$. \\
%Graafisesti funktion nollakohdat saadaan ratkaistua kuvaajasta. Kuvaajasta nähdään, että funktion nollakohdat ovat $x_1 \approx 1,4$ ja $x_2 \approx -1,4$. \\
%Algebrallisesti saadaan ratkaistua, että funktion nollakohdat ovat
%\begin{align*}
%f(x)&=0 \\
%x^2-2&=0 \\
%x^2&=2 \\
%x&= \pm \sqrt[]{2}
%\end{align*}
%Funktion $f$ kuvaajasta huomataan, että se on symmetrinen y-akselin suhteen.
%
%\textbf{Esimerkki 2.} \\
%Piirrä funktioiden $f(x)=x^2+1$, $g(x)=2x^2$ ja $h(x)=\frac{1}{3}x^2$ kuvaajat. \\ \\
%
%\begin{kuvaajapohja}{1}{-2}{2}{-1}{3}
%  \kuvaaja{x**2+1}{$f(x)=x^2+1$}{blue}
%\end{kuvaajapohja}
%
%
%\begin{kuvaajapohja}{1}{-2}{2}{-1}{3}
%  \kuvaaja{2*x**2}{$g(x)=2x^2$}{blue}
%\end{kuvaajapohja}
%
%\begin{kuvaajapohja}{1}{-2}{2}{-1}{3}
%  \kuvaaja{(1 / 3.0)*(x**2)}{$h(x)=\frac{1}{3}x^2$}{blue}
%\end{kuvaajapohja}
%
%Mitä tapahtuu funktion kuvaajan muodolle, kun termin $ax^2$ kerroin $a$ muuttuu? \\ \\
%
%\textbf{Esimerkki 3.} \\
%Piirrä funktioiden $f(x)=-x^2+x$, $g(x)=-x^2+2x+1$ ja $h(x)=-x^2+\frac{1}{2}x-1$ kuvaajat. \\
%\missingfigure \\
%Mitä tapahtuu funktion kuvaajan muodolle, kun termin $bx$ kerroin $b$ muuttuu? \\ \\
%
%\textbf{Esimerkki 4.} \\
%Piirrä funktioiden $f(x)=x^2$, $g(x)=x^2-2$ ja $h(x)=x^2+\frac{3}{2}$ kuvaajat. \\ \\
%Mitä tapahtuu funktion kuvaajan muodolle, kun vakiotermi $c$ muuttuu? \\ \\

Koottuna:

\laatikko{Toisen asteen polynomifunktion $f(x)=ax^2+bx+c$ kuvaaja on
\begin{itemize}
\item ylöspäin aukeava paraabeli, kun $a>0$
\item alaspäin aukeava paraabeli, kun $a<0$
\item sitä kapeampi, mitä suurempi $|a|$ on.
\end{itemize}
}

% FIXME: selitys tulisi hoitaa ilman itseisarvoa

%\textbf{Esimerkki 5.} \\
%Ratkaise funktion $f(x)=4x^2-13x+8$ nollakohdat.
%\begin{align*}
%f(x)&=0 \\
%4x^2-13x+8&=0 \\
%x&=\frac{-(-13) \pm \sqrt[]{(-13)^2-4 \cdot 4 \cdot 8}}{2 \cdot 4} \\
%x&=\frac{13 \pm \sqrt[]{169-128}}{8} \\
%x&=\frac{13 \pm \sqrt[]{41}}{8} \\
%x&=\frac{13 \pm \sqrt[]{41}}{8}
%\end{align*}




%Kuvaajaan transformaatioihin esimerkkiä myös muuttujan x muuttamisesta. Piirrä f(x+1):n kuvaaja jne.
%funktion nollakohtien ja yhtälön juurien yhteys
%paraabelin muodon perustelu: P(x)=ax^2+bx+c=a(x-b/2a)^2+b^2/4a^2, siis pienin/suurin arvo kun x=-b/2a
%-> ylöspäin ja alaspäin aukeavat paraabelit

\input{appendices/poljako_todistus}

\section{Tehtäviä ylioppilaskokeista}

\subsubsection*{Pitkän oppimäärän tehtäviä}

\begin{tehtava}
	Muodosta sen suoran yhtälö, joka kulkee pisteiden $(4, -3)$ ja $(-2,6)$ kautta, blaa. (S07/1b)
\end{tehtava}




\section{Sekalaisia tehtäviä}

Ryhmittele tehtävät luvuittain!
Sijoita nämä tehtävät sopivaan paikkaan, jos ovat hyviä!


\paragraph*{Polynomifunktion kuvaaja}

\begin{tehtava}
  Aukeavatko seuraavat paraabelit ylös- vai alaspäin?
  \begin{enumerate}[a)]
    \item $4x^2 + 100x - 3$
    \item $-x^2 + 1337$
    \item $5x^2 - 7x + 5$
    \item $-6(-3x^2 + 5)$
    \item $-13x(9 - 17x)$
    \item $100(1-x^2)$
  \end{enumerate}

  \begin{vastaus}
    \begin{enumerate}[a)]
      \item Ylös
      \item Alas
      \item Ylös
      \item Ylös
      \item Ylös
      \item Alas
    \end{enumerate}
  \end{vastaus}
\end{tehtava}

\begin{tehtava}
  \begin{enumerate}[a)]
    \item Ratkaise funktion $2x^2 - 5x - 3$ nollakohdat
    \item Millä arvoilla edellisen kohdan funktio $2x^2 - 5x - 3$ saa positiivisia arvoja?
    \item Onko em. funktiolla globaali raja-arvo (minimi tai maksimi), ja jos on, missä kohtaa funktio saa tämän arvon? Mikä on funktion arvo silloin?
  \end{enumerate}

  \begin{vastaus}
    \begin{enumerate}[a)]
      \item $x = 1.2$ tai $x = -0.2$
      \item $x = \frac{12}{10} = 1.2$ tai $x = -\frac{2}{10} = -0.2$
      \item Koska neliötermin kerroin a on positiivinen (2), funktiolla on globaali minimi (mutta ei ylärajaa). Symmetrian vuoksi minimi on nollakohtien puolivälissä kohdassa 0.5, jossa funktio saa siis pienimmän arvonsa -5.
    \end{enumerate}
  \end{vastaus}
\end{tehtava}

\begin{tehtava}
  Tutki, millä muuttujan x arvoilla seuraavat funktiot saavat positiivisia arvoja.
  \begin{enumerate}[a)]
    \item $x^2 - 4$
    \item $-x^2 - 2x + 3$
    \item $x^2 + 2x + 5$
    \item $-x^2 - 1$
  \end{enumerate}

  \begin{vastaus}
    \begin{enumerate}[a)]
      \item $x \leq -2$ tai $x \geq 2$
      \item $-3 \geq x \leq 1$
      \item Kaikilla x:n arvoilla.
      \item Ei millään x:n arvoilla.
    \end{enumerate}
  \end{vastaus}
\end{tehtava}

\section{Tavoittele valaistumista}

\Opensolutionfile{ans}[appendices/monsterit_vastaukset]

Tässä on joitakin tehtäviä, jotka on arvioitu hauskoiksi ja hyödyllisiksi kaikkein osaavimmille opiskelijoille. Niiden parissa vietetty aika ei mene hukkaan, vaikkei tehtävä ratkeaisikaan.

\begin{tehtava}
    \begin{enumerate}[a)]
        \item Osoita, että jos polynomi $P$ jaetaan polynomilla $x-1$, niin jakojäännös on polynomin $P$ kertoimien summa.
        \item Päättele tästä, että kokonaisluku $n$ on jaollinen yhdeksällä vain jos sen kymmenjärjestelmäesityksen numeroiden summa on jaollinen yhdeksällä.
    \end{enumerate}
\end{tehtava}

\begin{tehtava} %Ehkä käsitteellisesti vaikea
    $P$ on toisen asteen polynomi, jonka vakiotermi on $1$. Polynomi $Q$ määritellään lausekkeella $Q(x)=P(x+1)-P(x)$ ja siitä tiedetään, että $Q(0)=7$ ja $Q(1)=13$. Määritä polynomin $P$ lauseke.
    \begin{vastaus}
        $P(x) = 3x^2+4x+1$
    \end{vastaus}
\end{tehtava}

\begin{tehtava} %Vaikea!
    Etsi kaikki positiiviset kokonaisluvut $x$ ja $y$, joille pätee $9x^2-y^2=17$.
    \begin{vastaus}
    Opastus: jaa yhtälön vasen puoli tekijöihin muistikaavalla. 
    Ainoa ratkaisu on $x = 3$, $y=8$.
    \end{vastaus}
\end{tehtava}

\begin{tehtava} % Sikavaikea (tällä tasolla)
%täydentyy kahdeksi neliöksi, joiden summa on 0
    Ratkaise $x$ ja $y$ yhtälöstä $y^2+2xy+x^4-3x^2+4=0$.
    \begin{vastaus}
        $x=\sqrt{2}, y=-\sqrt{2}$ tai $x=-\sqrt{2}, y=\sqrt{2}$
    \end{vastaus}
\end{tehtava}

\begin{tehtava} % Kaunis
    Ratkaise $x$ ja $y$ yhtälöstä $2x^4+2y^4=4xy-1$. %lisää tai vähennä kiva termi puolittain
    \begin{vastaus}
        $x=\frac{\sqrt{2}}{2}, y=\frac{\sqrt{2}}{2}$ tai $x=-\frac{\sqrt{2}}{2}, y=-\frac{\sqrt{2}}{2}$
    \end{vastaus}
\end{tehtava}

\begin{tehtava} %Erittäin tärkeä ja hyödyllinen, mutta liekö vielä paikallaan... esitystä voisi ehkä muuttaa
Toisen asteen polynomi $P_1 = (ax-b)^2$ ($a \neq 0 $) on binomin neliönä aina epänegatiivinen. Tulon nollasäännön avulla nähdään, että sillä on tasan yksi nollakohta $x = \frac{b}{a}$. Tällöin polynomin diskriminantin arvo on 0, mikä voidaan nähdä myös suoraan laskemalla.

Polynomi $P_2 = (a_1x-b_1)^2+(a_2x-b_2)^2$ ($a_1,a_2 \neq $) on niin ikään aina epänegatiivinen, mutta sillä ei välttämättä ole nollakohtaa; se on epänegatiivisten termien summa, joka voi olla nolla vain, jos kaikkien termien neliöt ovat nollia. Koska jokainen binomin neliö voi saavuttaa nollan vain yhdessä pisteessä, ne voivat kaikki saavuttaa nollan korkeintaan yhdessä pisteessä. Mutta tällöin polynomin diskriminantin on oltava korkeintaan nolla. Muodosta polynomin $P_2$ diskriminantti, mikä epäyhtälö seuraa?

Saman päättelyn voi yleistää myös $n$:n binomin neliöiden summasta muodostetulle polynomille $P_n = (a_1x-b_1)^2+(a_2x-b_2)^2+\ldots+(a_nx-b_n)^2$. Vastaavalla päättelyllä polynomilla on korkeintaan yksi nollakohta, joten sen diskriminantti on epäpositiivinen. Laske $P_n$:n diskriminantti. Olet todistanut kuuluisan Cauchy-Schwarzin epäyhtälön:

\[
(a_1^2+a_2^2+\ldots+a_n^2)(b_1^2+b_2^2+\ldots+b_n^2) \geq (a_1b_1+a_2b_2+\ldots+a_nb_n)^2
\]




\end{tehtava}

\Closesolutionfile{ans}

\subsubsection*{Vastaukset}

\input{appendices/monsterit_vastaukset}

%\section{Hakemisto}
\newpage
\addcontentsline{toc}{chapter}{Hakemisto}
\printindex


