\section{Potenssifunktio}

Kahden muuttujan välinen riippuvuus ei monesti ole suoraan
tai kääntäen verrannollista. Potenssifunktion käyttäminen on eräs
keino kuvata tällaisia riippuvuuksia.

\laatikko{Potenssifunktioita ovat muotoa
$ f(x) = a x^n $ olevat funktiot, jossa $a \neq 0$. }

Eksponenttia $n$ kutsutaan potenssifunktion \emph{asteluvuksi}
tai \emph{asteeksi}.
Potenssifunktion eksponentti voi olla mikä tahansa reaaliluku, mutta
rajoitumme ensin käsittelemään tapauksia, jossa $n = 1, 2, 3\ldots $.

\begin{esimerkki}
Jos neliön sivun pituus on $x$, neliön pinta-ala voidaan laskea
funktiolla $A(x)=x^2$.
Esimerkiksi jos $x = 3$ cm, saadaan neliön pinta-alaksi $A(x) = 9$ cm$^2$.
Vastaavasti kuutiolle: jos $x$ kuvaa kuution särmän pituutta, funktiolla
$V(x)=x^3$ voidaan laskea kuution tilavuus. Sekä $A(x)$ että $V(x)$ ovat
esimerkkejä potenssifunktioista.
\end{esimerkki}

Potenssifunktion aste vaikuttaa funktion kuvaajan muotoon:
\begin{alakohdat}
  \item
Jos aste on parillinen, kuvaaja on U-kirjaimen muotoinen ja funktio
saa $a$:n merkistä riippuen joko pelkästään positiivisia tai pelkästään
negatiivisia arvoja.
  \item
Jos aste on pariton, kuvaaja muodostaa ''kaksoismutkan'' ja potenssifunktio
saa sekä positiivisia että negatiivisia arvoja.
\end{alakohdat}

%\missingfigure{Potenssifunktioiden kuvaajat -- yksi parillisilla potensseilla ja toinen parittomilla (kerroin a positiivinen).}
\begin{center}
\begin{kuvaajapohja}{0.5}{-6}{6}{-6}{6}
\kuvaaja{0.2*x**2}{$\frac{1}{5}x^2$}{red}
\kuvaaja{0.2*x**3}{$\frac{1}{5}x^3$}{blue}
\end{kuvaajapohja}
\end{center}
Potenssifunktion tärkeitä erikoistapauksia ovat asteluvut $n = 1$ ja $n = -1$.
Kun $n = 1$, saadaan edellisessä luvussa esitelty suoraan verrannollinen
riippuvuus $x$:n ja $f(x)$:n välillä, ja kun $n = -1$, muuttuja $x$ ja
funktion arvo $f(x)$ ovat kääntäen verrannolliset.

Potenssifunktiota voidaan laajentaa sallimalla eksponentille $n$
myös negatiiviset arvot.
Tällöin funktion muoto muuttuu merkittävästi. Funktio ei myöskään ole
enää määritelty kohdassa $x = 0$, vaan funktion arvot
''räjähtävät äärettömyyteen'', kun y-akselia lähestytään:

%\missingfigure{Potenssifunktioiden kuvaajat - yksi potenssilla -1 ja toinen potenssilla -2.}

\begin{center}
\begin{kuvaajapohja}{0.5}{-6}{6}{-6}{6}
%\kuvaaja{x**(-1)}{$\frac{1}{5}x^2$}{red}
%\kuvaaja{x**(-2)}{$\frac{1}{5}x^3$}{blue}
% FIXME: alla oleva puukotus, että saa äärettömyyteen menevät käppyrät piirtymään oikein
\newcommand{\kuvaajaneg}[3]{
\draw[smooth,color=#3,thick,domain=\kuvaajaminx:-0.01,scale=\kuvaajascale,samples=300] plot function{(#1) < \kuvaajamaxy ? ((#1) > \kuvaajaminy ? (#1) : NaN) : NaN} node[right] {#2};
}
\newcommand{\kuvaajapos}[3]{
\draw[smooth,color=#3,thick,domain=0.01:\kuvaajamaxx,scale=\kuvaajascale,samples=300] plot function{(#1) < \kuvaajamaxy ? ((#1) > \kuvaajaminy ? (#1) : NaN) : NaN} node[right] {#2};
}
\kuvaajaneg{x**(-1)}{$x^{-1}$}{red}
\kuvaajaneg{x**(-2)}{}{blue}

\kuvaajapos{x**(-1)}{}{red}
\kuvaajapos{x**(-2)}{$x^{-2}$}{blue}

\end{kuvaajapohja}
\end{center}

%Potenssifunktiota käsitellään samalla lailla riippumatta eksponentin
%etumerkistä. Huomaa kuitenkin, että $\frac{1}{x^n} \neq 0 $ kaikilla $x$:n %arvoilla.

%Tästä alaspäin on potenssiyhtälöitä, jotka varmaan menee jo päälle
%aiemmin käydyn asian kanssa. Sekaannus meikäläisen osalta... -Matti

\begin{tehtavasivu}

\paragraph*{Opi perusteet}

\begin{tehtava}
Mikä on seuraavien potenssifunktioiden aste?
\begin{alakohdat}
\alakohta{$f(x) = x$}
\alakohta{$f(x) = 5x^5$}
\alakohta{$f(x) = \frac{1}{2x}$}
\alakohta{$f(x) = x^{-2}$}
\end{alakohdat}
\begin{vastaus}
\begin{enumerate}
\alakohta{$1$}
\alakohta{$5$}
\alakohta{$-1$}
\alakohta{$-2$}
\end{alakohdat}
\end{vastaus}
\end{tehtava}

\begin{tehtava}
Olkoon $f(x)=x^{-3}$. Laske
\begin{alakohdat}
\alakohta{$f(1)$}
\alakohta{$f(2)$}
\alakohta{$f(-\frac{1}{3})$}
\end{alakohdat}
\begin{vastaus}
\begin{enumerate}
\alakohta{$1$}
\alakohta{$\frac{1}{8}$}
\alakohta{$-27$}
\end{alakohdat}
\end{vastaus}
\end{tehtava}

\paragraph*{Hallitse kokonaisuus}

\begin{tehtava}
Olkoon $f(x)=x^2$. Millä $x$:n arvoilla
\begin{alakohdat}
\alakohta{$f(x)=4$}
\alakohta{$f(x)=25$}
\alakohta{$f(x)=121$}
\end{alakohdat}
\begin{vastaus}
\begin{enumerate}
\alakohta{$2$}
\alakohta{$5$}
\alakohta{$11$}
\end{alakohdat}
\end{vastaus}

\end{tehtava}
\begin{tehtava}
Minkä kahden peräkkäisen kokonaisluvun välissä yhtälön $x^2 = 12$ positiivinen ratkaisu on?
\begin{vastaus}
Ratkaisu on lukujen $3$ ja $4$ välissä.
\end{vastaus}
\end{tehtava}

\paragraph*{Sekalaisia tehtäviä}

\begin{tehtava}
Kuution tilavuus särmän pituuden funktiona on $f(x) = x^3$. Jos kuution tilavuus on $64$, mikä on särmän pituus?
\begin{vastaus}
Särmän pituus on $4$.
\end{vastaus}
\end{tehtava}

\end{tehtavasivu}
