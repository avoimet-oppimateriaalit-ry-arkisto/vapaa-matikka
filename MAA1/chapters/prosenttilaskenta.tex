\section{Prosenttilaskenta}

Sana prosentti tulee latinan kielen ilmaisusta \emph{pro centum},
mikä tarkoittaa kirjaimellisesti sataa kohden. 
Prosentteja käytetään ilmaisemaan suhteellista osuutta.
Prosentin merkki on \%.

\laatikko{$1~\textnormal{prosentti} \; = 1~\% = \frac{1}{100} = 0,01$}

\begin{esimerkki}
    Prosenttiluvut voidaan esittää myös muilla tavoin.
    \begin{alakohdat}
        \alakohta{$6~\% = \frac{6}{100} = 0,06$}
        \alakohta{$48,2~\% = \frac{48,2}{100} = 0,482$}
        \alakohta{$140~\% = \frac{140}{100} = 1,40$ }
    \end{alakohdat}
\end{esimerkki}

% PERUSARVO
\laatikko{Lukua, josta suhde lasketaan, kutsutaan \termi{perusarvo}{perusarvoksi}.}

\begin{esimerkki}
    Jos sadan euron hintaisen tuotteen hintaa on alennettu $25$ prosenttia,
    niin alennettu hinta on $75$ euroa. Jos sen sijaan alkuperäinen
    hinta nousee $15$ prosenttia, niin tuotteen uusi hinta on $115$ euroa.
    Perusarvo on molemmissa tapauksissa $100$ euroa.
    
    \begin{center}
        \includegraphics[scale=.25]{pictures/Kuva13-1-100.pdf}
        \includegraphics[scale=.25]{pictures/Kuva13-2-75.pdf}
        \includegraphics[scale=.25]{pictures/Kuva13-3-115.pdf}
    \end{center}
\end{esimerkki}

% MUUTOSPROSENTTI
\laatikko{
    Prosentteja käytetään usein ilmaisemaan suureiden muutoksia, esimerkiksi luku $a$ kasvaa luvuksi $b$.
    \termi{muutosprosentti}{Muutosprosenttia} laskettaessa muutoksen suuruutta verrataan alkuperäiseen lukuun.
    Perusarvona on siis alkuperäinen arvo, johon nähden muutos on tapahtunut. Muutosta merkitään yleensä symbolilla
    $\Delta$.
    
    \termi{absoluuttinen muutos}{Absoluuttinen muutos} luvusta $a$ lukuun $b$ on $b-a$.
    Suhteellinen muutos saadaan suhteuttamalla muutos absoluuttinen muutos alkuperäiseen lukuun $a$ eli laskemalla
    
    \[ \Delta_{suhteellinen} = \frac{\Delta_{absoluuttinen}}{a} = \frac{b-a}{a} \]
    
    Muutosprosentti saadaan suhteellisesta muutoksesta muuttamalla se prosenttiluvuksi:
    
    \[ \Delta_{prosentti} = \Delta_{suhteellinen} \cdot 100~\% = \frac{b-a}{a} \cdot 100~\% \]
    
    Jos muutosprosenttia lasketaan vastakkaiseen suuntaan, saadaan eri muutosprosentti.
}

\begin{esimerkki}
    Vesan paino on tammikuussa $68$~kg ja kesäkuussa $64$~kg. Kuinka monta prosenttia Vesa on laihtunut?
    
    \textbf{Ratkaisu.}
    
    Halutaan tietää Vesan painon muutos prosentteina tammikuusta kesäkuuhun.
    
    \[
        \Delta_{prosentti}
        = \frac{b-a}{a} \cdot 100~\%
        = \frac{64-68}{68} \cdot 100~\%
        = \frac{-4}{68} \cdot 100~\%
        \approx -0,06 \cdot 100~\%
        = -6~\% 
    \]
    
    Vesan paino on muuttunut kuudella prosentilla negatiiviseen suuntaan,
    eli Vesa on laihtunut kuusi prosenttia.
    
    \textbf{Vastaus.}
    
    Vesa on laihtunut $6~\%$.
\end{esimerkki}

% VERTAILUPROSENTTI
\laatikko{
    Muutosprosentille läheinen käsite on \termi{vertailuprosentti}{vertailuprosentti}.
    Vertailuprosentilla tarkoitetaan sitä, kuinka paljon jokin on jostakin.
    
    Vertailuprosentilla vastataan siis kysymykseen ''kuinka monta prosenttia luku $a$ on luvusta $b$?''
    Vertailuprosentti on tässä tapauksessa $\frac{a}{b} \cdot 100~\%$.
    
    Vertailuprosentista saamme myös uuden esitystavan muutosprosentille, sillä samaan suuntaan laskettujen
    vertailu- ja muutosprosenttien erotus on aina $100~\%$.
    
    Jos vertailuprosenttia lasketaan vastakkaiseen suuntaan, saadaan eri vertailuprosentti.
}

\begin{esimerkki}
    Vesa ansaitsee kuukaudessa ${3~200}$ euroa ja Antero ${2~300}$ euroa.
    Kuinka monta prosenttia Anteron tulot ovat Vesan tuloista? 
    
    \textbf{Ratkaisu.}
    
    Lasketaan vertailuprosentti. Perusarvo on tehtävänannon mukaisesti
    Vesan palkka eli ${3~200}$ euroa.
    
    \[
        \frac{2300}{3200} \cdot 100~\%
        \approx 0,72\cdot 100~\%
        = 72~\%.
    \]
    
    \textbf{Vastaus.}
    
    Anteron tulot ovat $72~\%$ Vesan tuloista.
\end{esimerkki}

% PROSENTTIYKSIKKÖ
\laatikko{
    \termi{prosenttiyksikkö}{Prosenttiyksikkö} mittaa prosenttiosuuksien välisiä eroja.
    Esimerkiksi $4~\%$ on $2$ prosenttiyksikköä suurempi kuin $2~\%$, mutta $100~\%$
    suurempi kuin $2~\%$. Jos prosenttiluku muuttuu, muutos voidaan ilmaista joko
    prosentteina tai prosenttiyksikköinä.
    
    Prosentin ja prosenttiyksikön merkitysero on keskeinen esimerkiksi
    talousuutisten tulkinnassa.
}

\begin{esimerkki}
    Tuotteen markkinaosuus on vuoden tammikuussa $10$~\% ja kesäkuussa $15$~\%.
    \begin{alakohdat}
        \alakohta{Kuinka monta prosenttia tuotteen markkinaosuus on noussut?}
        \alakohta{Kuinka monta prosenttiyksikköä tuotteen markkinaosuus on noussut?}
    \end{alakohdat}
    
    \textbf{Ratkaisu.}
    
    \begin{alakohdat}
        \alakohta{Tuotteen markkinaosuus on noussut}
            \[
                \frac{15-10}{10} \cdot 100~\%= \frac{5}{10}\cdot 100~\% = 50~\%.
            \]
        
        \alakohta{Tuotteen markkinaosuus on noussut $15-10=5$ prosenttiyksikköä.}
    \end{alakohdat}
    
    \textbf{Vastaus.}
    
    \begin{alakohdat}
        \alakohta{$50$ prosenttia}
        \alakohta{$5$ prosenttiyksikköä.}
    \end{alakohdat}
\end{esimerkki}

\begin{tehtavasivu}

% tässä ei käytetä konsistenssin vuoksi euromerkkiä \euro (textmode), vaan sanallista ilmaisua ''euro''

% FIXME: tehtävä: ilmaise desimaalilukuna seuraavat prosentit

% FIXME: tehtävä perusarvosta

% FIXME: laaja tehtävä ansio- ja pääomatulojen verotuksesta

% FIXME: korolle korkoa -laskut esitellään seuraavassa kappaleessa, onko tämä tehtävä tarkoituksella tässä? 
%\begin{tehtava}
%    Erään pankin myöntämä opintolaina kasvaa korkoa $2~\%$ vuodessa. Kuinka monta 
%    prosenttia laina on kasvanut korkoa alkuperäiseen verrattuna kymmenen vuoden kuluttua?
%    \begin{vastaus}
%        $22~\%$.
%    \end{vastaus}
%\end{tehtava}

% FIXME: nykyisessä kappalejärjestyksessä verrannollisuus tulee vasta myöhemmin 
%\begin{tehtava}
%    Kappaleen putoamisen kesto maahan korkeudelta $x$ on kääntäen verrannollinen
%    putoamiskiihtyvyyden $g$ neliöjuureen. Vakio $g$ on kullekin taivaankappaleelle ominainen
%    ja eri puolilla taivaankappaletta likimain sama. Empire State Buildingin katolta (korkeus $381$ metriä)
%    pudotetulla kuulalla kestää noin $6,2$ sekuntia osua maahan. Marsin putoamiskiihtyvyys on $37,6\%$
%    Maan putoamiskiihtyvyydestä. 
%    
%    Jos Empire State Building sijaitsisi Marsissa, kuinka kauan kuluisi kuulan maahan osumiseen?
%    \begin{vastaus}
%        Noin $10$ sekuntia.
%    \end{vastaus}
%\end{tehtava}

\begin{tehtava}
    Oppikirjamaraton-tiimi kävi lounastamassa. Osoita oheisen kuitin tiedoilla vääräksi yleinen virhekäsitys,
    että $13~\%$ arvonlisävero olisi $13~\%$ lopullisesta myyntihinnasta.
    \includegraphics[width=80mm, angle=270]{pictures/alv-kuitti}
    \begin{vastaus}
         $13~\%$ sadasta eurosta on $13$ euroa, ei $11,50$ euroa, niin kuin kuitissa todetaan. Arvonlisävero
         lasketaan suhteessa verottomaan hintaan, ei lopulliseen myyntihintaan.
    \end{vastaus}
\end{tehtava}

\begin{tehtava}
    Laukun normaalihinta on $225$ euroa, ja se on $25$ prosentin alennuksessa.
    Mikä on alennettu hinta?
    \begin{vastaus}
        $168,75$ euroa
    \end{vastaus}
\end{tehtava}

\begin{tehtava}
    Jaakon kuukausipalkka on $1623,52$ euroa. Hän saa $1,3~\%$ palkankorotuksen.
    Mikä on Jaakon kuukausipalkka korotuksen jälkeen?
    \begin{vastaus}
        $1644,63$ euroa
    \end{vastaus}
\end{tehtava}

\begin{tehtava}
    Kirjan myyntihinta, joka sisältää arvolisäveron, on $9~\%$ suurempi kuin kirjan veroton hinta.
    Laske kirjan veroton hinta, kun myyntihinta on $27$ euroa.
    \begin{vastaus}
        Kirjan veroton hinta on $24,77$ euroa.
    \end{vastaus}
\end{tehtava}

\begin{tehtava}
    Sokerijuurikkaassa on $18~\%$ sokeria. Kuinka paljon sokerijuurikkaita tarvitaan valmistettaessa
    $8$ tonnia sokeriliuosta, jonka sokeripitoisuus on $4,5~\%$?
    \begin{vastaus}
        $2$ tonnia
    \end{vastaus}
\end{tehtava}

\begin{tehtava}
    Perussuomalaisten kannatus oli vuoden 2007 eduskuntavaaleissa $4,1~\%$ ja 
    vuoden 2011 eduskuntavaaleissa $19,1~\%$. Kuinka monta prosenttiyksikköä 
    kannatus nousi? Kuinka monta prosenttia kannatus nousi?
    \begin{vastaus}
        Kannatus nousi $15$ prosenttiyksikköä, ja toisaalta $366$ prosenttia.
        Mediassa kannatusmuutokset ilmoitetaan prosenttiyksiköissä.
    \end{vastaus}
\end{tehtava}

\begin{tehtava}
    Askartelukaupassa on alennusviikot, ja kaikki tavarat myydään $60~\%$:n 
    alennuksella. Viimeisenä päivänä kaikista hinnoista annetaan vielä 
    lisäalennus, joka lasketaan aiemmin alennetusta hinnasta. Minkä suuruinen 
    lisäalennus tulee antaa, jos lopullisen kokonaisalennuksen halutaan olevan $80~\%$?
    \begin{vastaus}
        $50~\%$
    \end{vastaus}
\end{tehtava}

\begin{tehtava}
    Hedelmissä on vettä aluksi 60~\%. Kuinka monta prosenttia vedestä on 
    haihdutettava, jotta hedelmissä tämän jälkeen olisi vain 20~\% vettä?
    \begin{vastaus}
        $50~\%$
    \end{vastaus}
\end{tehtava}


\begin{tehtava}
    Samulin pituus on $165$~cm ja Joonaksen $173$~cm.
    \begin{alakohdat}
        \alakohta{Kuinka monta prosenttia Samulin pituus on Joonaksen pituudesta?}
        \alakohta{Kuinka monta prosenttia Samuli on lyhyempi kuin Joonas?}
        \alakohta{Kuinka monta prosenttia Joonas on pidempi kuin Samuli?}
    \end{alakohdat}
    \begin{vastaus}
        \begin{alakohdat}
            \alakohta{$95,4~\%$}
            \alakohta{$4,62~\%$}
            \alakohta{$4,85~\%$}
        \end{alakohdat}
    \end{vastaus}
\end{tehtava}

\begin{tehtava}
    Yleinen arvonlisäveroprosentti oli Suomessa vuonna 2012 $23~\%$ tuotteen
    verottomasta hinnasta. Tuotteen hinta koostuu sen verottomasta hinnasta 
    ja tuotteesta maksettavasta arvonlisäverosta. Kuinka monta 
    prosenttia arvonlisävero on tuotteen myyntihinnasta?
    \begin{vastaus}
        $18,7~\%$
    \end{vastaus}
\end{tehtava}

\begin{tehtava}
    Kun matkalipun hintaa korotettiin $10,0~\%$, matkustajien määrä väheni $10,0~\%$.
    Kuinka monella prosentilla tällöin lisääntyivät tai vähentyivät liikennöitsijän 
    lipputulot?
    \begin{vastaus}
        Vähentyivät $1$ prosentilla.
    \end{vastaus}
\end{tehtava}

\begin{tehtava}
    Tuoreissa omenissa on vettä $80~\%$ ja sokeria $4~\%$. Kuinka monta prosenttia sokeria
    on samoissa omenissa, kun ne on kuivattu siten, että kosteusprosentti on $20$? [K2000, 4]
    \begin{vastaus}
        $16~\%$
    \end{vastaus}
\end{tehtava}

\begin{tehtava}
    Matin ja Iidan duo saa julkisuutta, ja he alkavat myydä CD-levyään keikkojen yhteydessä $10$ euron
    kappalehinnalla. Jonkin ajan päästä he päättävät laskea CD-levyn hintaa $20$ prosenttia. Matti alkaa
    kuitenkin katua päätöstä, ja ehdottaa tämän alennetun hinnan korottamista $20$ prosentilla. Mikä olisi
    tämän toimenpiteen jälkeen CD:n uusi hinta? Montako prosenttia olisi korotuksen oltava, jotta oikeasti
    päästäisiin takaisin alkuperäiseen $10$ euron hintaan?
    \begin{vastaus}
        $25~\%$
    \end{vastaus}
\end{tehtava}

\begin{tehtava}
    Jalkapalloilija Georgios Samaras teki ensimmäisellä kaudellaan Skotlannin valioliigassa (2007-08)
    $5$ maalia Celtic F.C.:n paidassa. Seuraavalla kaudella Samaras teki Celticille liigassa $15$ maalia.
    Kuinka monta prosenttia Samaraksen maalimäärä nousi?
    \begin{vastaus}
        $200~\%$
    \end{vastaus}
\end{tehtava}

\begin{tehtava}
    Kaupungissa tuli jokaisen talonomistajan suorittaa kaupungin kassaan $5~\%$ saadusta
    hyyrymäärästä (vuokrasta, ruotsin sanasta \textit{hyra}). Sittemmin määrättiin, että mainittu
    prosentti oli oleva 10. Monellako prosentilla täytyy talonomistajien korottaa hyyryjä
    saadakseen saman puhtaan säästön kuin ennen? [YO 1877, 4]
    \begin{vastaus}
        $5,6~\%$
    \end{vastaus}
\end{tehtava}

\end{tehtavasivu}
