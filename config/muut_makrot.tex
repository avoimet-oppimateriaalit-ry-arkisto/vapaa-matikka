% muokattava korostus termeille
\newcommand{\termi}[2][] {%
  \ifthenelse{\isempty{#2}}{\korostettuindeksi{#1}{#1}}%
  {\korostettuindeksi{#1}{#2}}%
}

% käytetään kun sovelletaan sama funktio molemmille puolille
% (pystypalkki kaksoisviivalla)
\newcommand{\ppalkki}{\big\|}

% \lowercase poistettu, aiheutti ongelmia makeindexin kanssa
\newcommand{\korostettuindeksi}[2]{\textbf{#1}\index{#2}}

\newenvironment{todistus}[0]
{
  \begin{proof}
}
{
  \end{proof}
}

% Tarvitaan kuvien ja taulukkojen vierekkäin laittamiseen.
\def\vcent#1{\mathsurround0pt$\vcenter{\hbox{#1}}$}


\newcommand{\Harjoitustehtavat}{%
	\newpage
	\subsection*{Harjoitustehtäviä}%
	%\addcontentsline{toc}{section}{Harjoitustehtäviä}
}

% Linkki ja QR-koodi. Parametrit: URL ja kuvaus.
\newcommand{\qrlinkki}[2]{
	
	\begin{minipage}[c]{0.6in}
		\begin{pspicture}(0.6in,0.6in)
			\psscalebox{0.6}{
				\psbarcode{#1}{}{qrcode}
			}
		\end{pspicture}
		\vspace{0.12in}
	\end{minipage}
	\begin{minipage}[l]{\textwidth-0.6in}
		#2
		
		{\scriptsize\url{#1}}
	\end{minipage}
	
}
