% Ympäristö välien, pisteiden ja käyrien käyttäytymisen kuvaamiseen
% lukusuoralla.
% Käyttö:
% \begin{lukusuora}{-1}{1}{10}
% \lukusuoraparaabeli{0.2}{0.8}{1}
% \lukusuoravalisa{0.2}{0.8}{x}{y}
% \lukusuoravaliaa{-0.5}{-0.3}{a}{b}
% \lukusuorapiste{0}{$x^2+y^2$}
% \lukusuoranimi{0.5}{$+$}
% \lukusuoraalanimi{0.05}{$-$}
% \end{lukusuora}
% 
% luo lukusuoran joka piirretään 10 pituisena, käyttää sisäisesti
% koordinaatteja -1..1, ja jolla näytetään väli [0.2, 0.8[ ja
% väli ]-0.5, -0.3[ nimillä [x, y[ ja ]a, b[. Näyttää pisteen
% 0 nimellä $x^2+y^2$ ja piirtää paraabelin jonka nollakohdat ovat
% 0.2 ja 0.8 ja huippu 1. Lisäksi se laittaa +-merkin suoran yläpuolelle
% kohtaan 0.5 ja --merkin suoran alapuolelle kohtaan 0.05.
%
% Mielivaltaisia kuvaajia saa lukusuoralle lukusuorakuvaaja-komennolla.



% lukusuora-ympäristön parametrit:
% alkux loppux viivanpituus
\newenvironment{lukusuora}[3]
{
	\begin{tikzpicture}
		\draw[arrows=-triangle 45,line width=0.2mm] (0, 0) -- (#3, 0);
		\pgfmathsetmacro{\LSax}{#1}
		\pgfmathsetmacro{\LSdx}{#2-#1}
		\pgfmathsetmacro{\LSpituus}{#3}
}
{
	\end{tikzpicture}
}

% lukusuoravali??-komennon parametrit:
% alkux loppux alkunimi loppunimi

% avoimet ja suljetut välit:
% lukusuoravaliss [a, b]
% lukusuoravalisa [a, b[
% lukusuoravalias ]a, b]
% lukusuoravaliaa ]a, b[

\newcommand{\lukusuoravaliss}[4]{
	\lukusuoravali{#1}{#2}{#3}{#4}{black}{black}
}
\newcommand{\lukusuoravalisa}[4]{
	\lukusuoravali{#1}{#2}{#3}{#4}{black}{white}
}
\newcommand{\lukusuoravalias}[4]{
	\lukusuoravali{#1}{#2}{#3}{#4}{white}{black}
}
\newcommand{\lukusuoravaliaa}[4]{
	\lukusuoravali{#1}{#2}{#3}{#4}{white}{white}
}

\newcommand{\lukusuoravali}[6]{
	\pgfmathsetmacro{\LSxa}{\LSpituus*(#1-\LSax)/\LSdx}
	\pgfmathsetmacro{\LSxb}{\LSpituus*(#2-\LSax)/\LSdx}
	\draw[line width=0.6mm] (\LSxa,0) -- (\LSxb,0);
	\fill (\LSxa,0) circle (0.1);
	\fill (\LSxb,0) circle (0.1);
	\fill[color=#5] (\LSxa,0) circle (0.08);
	\fill[color=#6] (\LSxb,0) circle (0.08);
	% Integraalimerkit ovat riittävän korkeita saamaan bounding boxit saman
	% kokoisiksi.
	\draw (\LSxa,0) node[above] {\phantom{$\int$}#3\phantom{$\int$}};
	\draw (\LSxb,0) node[above] {\phantom{$\int$}#4\phantom{$\int$}};
}

% lukusuorapiste-komennon parametrit:
% x nimi
\newcommand{\lukusuorapiste}[2]{
	\pgfmathsetmacro{\LSx}{\LSpituus*(#1-\LSax)/\LSdx}
	\fill (\LSx,0) circle (0.1);
	\draw (\LSx,0) node[above] {\phantom{$\int$}#2\phantom{$\int$}};
}

% lukusuorapienipiste-komennon parametrit:
% x nimi
\newcommand{\lukusuorapienipiste}[2]{
	\pgfmathsetmacro{\LSx}{\LSpituus*(#1-\LSax)/\LSdx}
	\fill (\LSx,0) circle (0.07);
	\draw (\LSx,0) node[above] {\phantom{$\int$}#2\phantom{$\int$}};
}

% lukusuoranimi-komennon parametrit:
% x nimi
\newcommand{\lukusuoranimi}[2]{
	\pgfmathsetmacro{\LSx}{\LSpituus*(#1-\LSax)/\LSdx}
	\draw (\LSx,0) node[above] {\phantom{$\int$}#2\phantom{$\int$}};
}

% lukusuoraalanimi-komennon parametrit:
% x nimi
\newcommand{\lukusuoraalanimi}[2]{
	\pgfmathsetmacro{\LSx}{\LSpituus*(#1-\LSax)/\LSdx}
	\draw (\LSx,0) node[below] {\phantom{$\int$}#2\phantom{$\int$}};
}

% lukusuoraparaabeli-komennon parametrit:
% nollakohta1 nollakohta2 huippuy
% kannattaa olla huippuy = -1 tai 1
\newcommand{\lukusuoraparaabeli}[3]{
	\pgfmathsetmacro{\LSxa}{\LSpituus*(#1-\LSax)/\LSdx}
	\pgfmathsetmacro{\LSxb}{\LSpituus*(#2-\LSax)/\LSdx}
	\providecommand{\LSkaava}{-4*(#3)*(x-\LSxa)*(x-\LSxb)/((\LSxa-\LSxb)**2)}
	\draw[smooth,samples=300,thick,domain=0:\LSpituus,samples=300] plot function{\LSkaava > -1.5 ? (\LSkaava < 1.5 ? \LSkaava : NaN) : NaN};
}

% lukusuorakuvaaja-komennon parametrit:
% kaava
\newcommand{\lukusuorakuvaaja}[1]{
	\pgfmathsetmacro{\LSxa}{-\LSpituus*\LSax/\LSdx}
	\begin{scope}[shift={(\LSxa, 0)}]
		\draw[smooth,samples=300,thick,domain=\LSax:\LSax+\LSdx,xscale=\LSpituus/\LSdx] plot function{(#1) > -1.5 ? ((#1) < 1.5 ? (#1) : NaN) : NaN};
	\end{scope}
}
