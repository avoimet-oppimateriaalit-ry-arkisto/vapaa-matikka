% Infoboksi väreineen

\usepackage[framemethod=TikZ]{mdframed}

\definecolor{ParisGreen}{RGB}{80,200,120}
%\definecolor{PersianGreen}{RGB}{0,166,147}
%\definecolor{darkKhaki}{RGB}{189,183,107}
%\definecolor{cornflowerBlue}{RGB}{154,206,235}
%\definecolor{thulianPink}{RGB}{222,111,161}
%\definecolor{olivine}{RGB}{154,185,115}
%\definecolor{battleshipGray}{RGB}{132,132,130}

\pgfdeclarehorizontalshading{laatikkotausta}{100bp}{color(0bp)=(ParisGreen); color(50bp)=(ParisGreen); color(100bp)=(ParisGreen!50)}
\makeatletter

\mdfdefinestyle{laatikkotyyli}{
  leftmargin=0pt,
  innerleftmargin=5pt,
  rightmargin=0pt,
  innerrightmargin=0pt,
  middlelinewidth=0pt,
  apptotikzsetting={\tikzset{mdfbackground/.append style = {shading = laatikkotausta}}},
  innertopmargin=3mm,
  innerbottommargin=3mm,
  needspace=3\baselineskip,
}
\makeatother

\newcommand{\laatikko}[1]{\begin{minipage}{\textwidth}\begin{mdframed}[style=laatikkotyyli] #1 \end{mdframed}\end{minipage}}
