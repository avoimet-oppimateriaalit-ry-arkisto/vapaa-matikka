\section*{Pythagoraan lause}

Pythagoraan lause on eräs vanhimmista matemaattisista lauseista. Sille tunnetaan monia eri
todistuksia, joista kaksi yksinkertaista esitellään seuraavaksi.

\laatikko{
	\termi{Pythagoraan lause}{Pythagoraan lause}. Olkoot suorakulmaisen kolmion kateettien
	pituudet $a$ ja $b$, ja hypotenuusan pituus $c$. Tällöin pätee
	\[
	a^2 + b^2 = c^2
	\] 
}

\textbf{Pythagoraan lauseen todistus 1}. Olkoot $A$, $B$ ja $C$ suorakulmaisen kolmion
kärjet niin, että kulma $\angle ACB$ on suora. Olkoon piste $D$ sellainen janan $AB$ piste,
jolle $\angle CDA$ on suora.

(KUVA)

Tällöin kolmioilla $ABC$ ja $ACD$ on yhteinen kulma $\angle A$ ja molemmissa on suora kulma, joten ne ovat yhdenmuotoiset (kk). Samoin kolmioilla $ABC$ ja $CBD$ on yhteinen kulma
$\angle B$ ja molemmissa on suora kulma, joten nekin ovat yhdenmuotoiset (kk)

Yhdenmuotoisuuksista saadaan, että
\[
\frac{AB}{AC} = \frac{AC}{AD}
\]
ja
\[
\frac{AB}{CB} = \frac{CB}{DB}.
\]

Ristiinkertomalla saadaan, että
\[
AB \cdot AD = AC^2
\]
ja
\[
AB \cdot DB = CB^2.
\]
Laskemalla nämä saadut yhtälöt yhteen saadaan

\begin{align*}
AC^2 + CB^2  = AB \cdot AD + AB \cdot DB \\
AC^2 + CB^2  = AB(AD + DB) \\
AC^2 + CB^2  = AB^2.
\end{align*}
Tämä on täsmälleen Pythagoraan lause. $\square $

Lauseen voi todistaa myös laskemalla pinta-aloja.

\textbf{Pythagoraan lauseen todistus 2}. Olkoot $a$, $b$ ja $c$ suorakulmaisen kolmion
sivujen pituudet, joista $c$ on hypotenuusan pituus. Tarkastellaan neliötä, jonka sivun
pituus on $a+b$, ja jaetaan se osiin kahdella eri tavalla alla olevien kuvien mukaisesti.

\begin{kuva}
a = 3
b = 1.5
c = sqrt(a**2 + b**2)

A = geom.piste(0, 0, piirra = False)
B = geom.piste(a, 0, piirra = False)
C = geom.piste(a + b, 0, piirra = False)
D = geom.piste(a + b, a, piirra = False)
E = geom.piste(a + b, a + b, piirra = False)
F = geom.piste(b, a + b, piirra = False)
G = geom.piste(0, a + b, piirra = False)
H = geom.piste(0, b, piirra = False)

geom.jana(A, B, "$a$")
geom.jana(B, C, "$b$")
geom.jana(C, D, "$a$")
geom.jana(D, E, "$b$")
geom.jana(E, F, "$a$")
geom.jana(F, G, "$b$")
geom.jana(G, H, "$a$")
geom.jana(H, A, "$b$")

geom.jana(B, H, "$c$")
geom.jana(H, F, "$c$")
geom.jana(F, D, "$c$")
geom.jana(D, B, "$c$")

siirraX(a + b + 2)

A = geom.piste(0, 0, piirra = False)
B = geom.piste(a, 0, piirra = False)
C = geom.piste(a + b, 0, piirra = False)
D = geom.piste(a + b, a, piirra = False)
E = geom.piste(a + b, a + b, piirra = False)
F = geom.piste(a, a + b, piirra = False)
G = geom.piste(0, a + b, piirra = False)
H = geom.piste(0, a, piirra = False)
X = geom.piste(a, a, piirra = False)

geom.jana(A, B, "$a$")
geom.jana(B, C, "$b$")
geom.jana(C, D, "$a$")
geom.jana(D, E, "$b$")
geom.jana(E, F, "$b$")
geom.jana(F, G, "$a$")
geom.jana(G, H, "$b$")
geom.jana(H, A, "$a$")

geom.jana(B, X, "$a$")
geom.jana(D, X, "$b$", puoli = False)
geom.jana(F, X, "$b$")
geom.jana(H, X, "$a$", puoli = False)
\end{kuva}

Ensimmäisessä kuvassa neliön osat ovat neljä yhtenevää suorakulmaista kolmiota, joiden
sivujen pituudet ovat $a$, $b$ ja $c$, ja kaksi neliötä, joiden sivujen pituudet ovat $a$ ja
$b$.

Toisessa kuvassa neliön osat ovat neljä yhtenevää suorakulmaista kolmiota, joiden
sivujen pituudet ovat $a$, $b$ ja $c$, ja neliö, jonka sivun pituus on $c$.
Koska yhteenlaskettujen pinta-alojen on oltava samat, on oltava $a^2 + b^2 = c^2$.
Pythagoraan lause on todistettu. $\square $

\begin{esimerkki}
	Maalari maalaa taloa. Hänellä on tikapuut, joiden pituus on 3,0 metriä. Maalarin on
	noustava 2,5 metrin korkeuteen voidakseen maalata katonrajan sinipunaiseksi. Kuinka
	lähelle seinää hänen on asetettava tikapuiden alapää?
	\begin{esimratk}
		Tikapuut ja seinä muodostavat suorakulmaisen kolmion, jonka hypotenuusa on tikapuut
		ja toinen kateetti on seinä. Merkitään $x$:llä tikapuiden etäisyyttä seinästä
		metreinä ja sovelletaan Pythagoraan lausetta.
		\begin{align*}
		x^2 + 2,5^2 = 3,0^2 \\
		x^2 + 6,25 = 9,0 \\
		x^2 = 2,75 \\
		x = \pm \sqrt{2,75} \approx \pm 1,7
		\end{align*}
		Koska x on etäisyys seinästä, negatiivinen ratkaisu ei kelpaa.
		\begin{esimvast}
		Alle $1,7$ metrin päähän seinästä. 
		\end{esimvast}
	\end{esimratk}
\end{esimerkki}

